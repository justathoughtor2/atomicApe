% \iffalse meta-comment
%
% File: colonequals.dtx
% Version: 2006/08/01 v1.0
% Info: Colon equals symbols
%
% Copyright (C) 2006 by
%    Heiko Oberdiek <heiko.oberdiek at googlemail.com>
%
% This work may be distributed and/or modified under the
% conditions of the LaTeX Project Public License, either
% version 1.3c of this license or (at your option) any later
% version. This version of this license is in
%    http://www.latex-project.org/lppl/lppl-1-3c.txt
% and the latest version of this license is in
%    http://www.latex-project.org/lppl.txt
% and version 1.3 or later is part of all distributions of
% LaTeX version 2005/12/01 or later.
%
% This work has the LPPL maintenance status "maintained".
%
% This Current Maintainer of this work is Heiko Oberdiek.
%
% This work consists of the main source file colonequals.dtx
% and the derived files
%    colonequals.sty, colonequals.pdf, colonequals.ins, colonequals.drv.
%
% Distribution:
%    CTAN:macros/latex/contrib/oberdiek/colonequals.dtx
%    CTAN:macros/latex/contrib/oberdiek/colonequals.pdf
%
% Unpacking:
%    (a) If colonequals.ins is present:
%           tex colonequals.ins
%    (b) Without colonequals.ins:
%           tex colonequals.dtx
%    (c) If you insist on using LaTeX
%           latex \let\install=y% \iffalse meta-comment
%
% File: colonequals.dtx
% Version: 2006/08/01 v1.0
% Info: Colon equals symbols
%
% Copyright (C) 2006 by
%    Heiko Oberdiek <heiko.oberdiek at googlemail.com>
%
% This work may be distributed and/or modified under the
% conditions of the LaTeX Project Public License, either
% version 1.3c of this license or (at your option) any later
% version. This version of this license is in
%    http://www.latex-project.org/lppl/lppl-1-3c.txt
% and the latest version of this license is in
%    http://www.latex-project.org/lppl.txt
% and version 1.3 or later is part of all distributions of
% LaTeX version 2005/12/01 or later.
%
% This work has the LPPL maintenance status "maintained".
%
% This Current Maintainer of this work is Heiko Oberdiek.
%
% This work consists of the main source file colonequals.dtx
% and the derived files
%    colonequals.sty, colonequals.pdf, colonequals.ins, colonequals.drv.
%
% Distribution:
%    CTAN:macros/latex/contrib/oberdiek/colonequals.dtx
%    CTAN:macros/latex/contrib/oberdiek/colonequals.pdf
%
% Unpacking:
%    (a) If colonequals.ins is present:
%           tex colonequals.ins
%    (b) Without colonequals.ins:
%           tex colonequals.dtx
%    (c) If you insist on using LaTeX
%           latex \let\install=y% \iffalse meta-comment
%
% File: colonequals.dtx
% Version: 2006/08/01 v1.0
% Info: Colon equals symbols
%
% Copyright (C) 2006 by
%    Heiko Oberdiek <heiko.oberdiek at googlemail.com>
%
% This work may be distributed and/or modified under the
% conditions of the LaTeX Project Public License, either
% version 1.3c of this license or (at your option) any later
% version. This version of this license is in
%    http://www.latex-project.org/lppl/lppl-1-3c.txt
% and the latest version of this license is in
%    http://www.latex-project.org/lppl.txt
% and version 1.3 or later is part of all distributions of
% LaTeX version 2005/12/01 or later.
%
% This work has the LPPL maintenance status "maintained".
%
% This Current Maintainer of this work is Heiko Oberdiek.
%
% This work consists of the main source file colonequals.dtx
% and the derived files
%    colonequals.sty, colonequals.pdf, colonequals.ins, colonequals.drv.
%
% Distribution:
%    CTAN:macros/latex/contrib/oberdiek/colonequals.dtx
%    CTAN:macros/latex/contrib/oberdiek/colonequals.pdf
%
% Unpacking:
%    (a) If colonequals.ins is present:
%           tex colonequals.ins
%    (b) Without colonequals.ins:
%           tex colonequals.dtx
%    (c) If you insist on using LaTeX
%           latex \let\install=y% \iffalse meta-comment
%
% File: colonequals.dtx
% Version: 2006/08/01 v1.0
% Info: Colon equals symbols
%
% Copyright (C) 2006 by
%    Heiko Oberdiek <heiko.oberdiek at googlemail.com>
%
% This work may be distributed and/or modified under the
% conditions of the LaTeX Project Public License, either
% version 1.3c of this license or (at your option) any later
% version. This version of this license is in
%    http://www.latex-project.org/lppl/lppl-1-3c.txt
% and the latest version of this license is in
%    http://www.latex-project.org/lppl.txt
% and version 1.3 or later is part of all distributions of
% LaTeX version 2005/12/01 or later.
%
% This work has the LPPL maintenance status "maintained".
%
% This Current Maintainer of this work is Heiko Oberdiek.
%
% This work consists of the main source file colonequals.dtx
% and the derived files
%    colonequals.sty, colonequals.pdf, colonequals.ins, colonequals.drv.
%
% Distribution:
%    CTAN:macros/latex/contrib/oberdiek/colonequals.dtx
%    CTAN:macros/latex/contrib/oberdiek/colonequals.pdf
%
% Unpacking:
%    (a) If colonequals.ins is present:
%           tex colonequals.ins
%    (b) Without colonequals.ins:
%           tex colonequals.dtx
%    (c) If you insist on using LaTeX
%           latex \let\install=y\input{colonequals.dtx}
%        (quote the arguments according to the demands of your shell)
%
% Documentation:
%    (a) If colonequals.drv is present:
%           latex colonequals.drv
%    (b) Without colonequals.drv:
%           latex colonequals.dtx; ...
%    The class ltxdoc loads the configuration file ltxdoc.cfg
%    if available. Here you can specify further options, e.g.
%    use A4 as paper format:
%       \PassOptionsToClass{a4paper}{article}
%
%    Programm calls to get the documentation (example):
%       pdflatex colonequals.dtx
%       makeindex -s gind.ist colonequals.idx
%       pdflatex colonequals.dtx
%       makeindex -s gind.ist colonequals.idx
%       pdflatex colonequals.dtx
%
% Installation:
%    TDS:tex/latex/oberdiek/colonequals.sty
%    TDS:doc/latex/oberdiek/colonequals.pdf
%    TDS:source/latex/oberdiek/colonequals.dtx
%
%<*ignore>
\begingroup
  \catcode123=1 %
  \catcode125=2 %
  \def\x{LaTeX2e}%
\expandafter\endgroup
\ifcase 0\ifx\install y1\fi\expandafter
         \ifx\csname processbatchFile\endcsname\relax\else1\fi
         \ifx\fmtname\x\else 1\fi\relax
\else\csname fi\endcsname
%</ignore>
%<*install>
\input docstrip.tex
\Msg{************************************************************************}
\Msg{* Installation}
\Msg{* Package: colonequals 2006/08/01 v1.0 Colon equals symbols (HO)}
\Msg{************************************************************************}

\keepsilent
\askforoverwritefalse

\let\MetaPrefix\relax
\preamble

This is a generated file.

Project: colonequals
Version: 2006/08/01 v1.0

Copyright (C) 2006 by
   Heiko Oberdiek <heiko.oberdiek at googlemail.com>

This work may be distributed and/or modified under the
conditions of the LaTeX Project Public License, either
version 1.3c of this license or (at your option) any later
version. This version of this license is in
   http://www.latex-project.org/lppl/lppl-1-3c.txt
and the latest version of this license is in
   http://www.latex-project.org/lppl.txt
and version 1.3 or later is part of all distributions of
LaTeX version 2005/12/01 or later.

This work has the LPPL maintenance status "maintained".

This Current Maintainer of this work is Heiko Oberdiek.

This work consists of the main source file colonequals.dtx
and the derived files
   colonequals.sty, colonequals.pdf, colonequals.ins, colonequals.drv.

\endpreamble
\let\MetaPrefix\DoubleperCent

\generate{%
  \file{colonequals.ins}{\from{colonequals.dtx}{install}}%
  \file{colonequals.drv}{\from{colonequals.dtx}{driver}}%
  \usedir{tex/latex/oberdiek}%
  \file{colonequals.sty}{\from{colonequals.dtx}{package}}%
  \nopreamble
  \nopostamble
  \usedir{source/latex/oberdiek/catalogue}%
  \file{colonequals.xml}{\from{colonequals.dtx}{catalogue}}%
}

\catcode32=13\relax% active space
\let =\space%
\Msg{************************************************************************}
\Msg{*}
\Msg{* To finish the installation you have to move the following}
\Msg{* file into a directory searched by TeX:}
\Msg{*}
\Msg{*     colonequals.sty}
\Msg{*}
\Msg{* To produce the documentation run the file `colonequals.drv'}
\Msg{* through LaTeX.}
\Msg{*}
\Msg{* Happy TeXing!}
\Msg{*}
\Msg{************************************************************************}

\endbatchfile
%</install>
%<*ignore>
\fi
%</ignore>
%<*driver>
\NeedsTeXFormat{LaTeX2e}
\ProvidesFile{colonequals.drv}%
  [2006/08/01 v1.0 Colon equals symbols (HO)]%
\documentclass{ltxdoc}
\usepackage{holtxdoc}[2011/11/22]
\usepackage{colonequals}
\usepackage{array}
\usepackage{capt-of}
\usepackage{longtable}
\begin{document}
  \DocInput{colonequals.dtx}%
\end{document}
%</driver>
% \fi
%
% \CheckSum{92}
%
% \CharacterTable
%  {Upper-case    \A\B\C\D\E\F\G\H\I\J\K\L\M\N\O\P\Q\R\S\T\U\V\W\X\Y\Z
%   Lower-case    \a\b\c\d\e\f\g\h\i\j\k\l\m\n\o\p\q\r\s\t\u\v\w\x\y\z
%   Digits        \0\1\2\3\4\5\6\7\8\9
%   Exclamation   \!     Double quote  \"     Hash (number) \#
%   Dollar        \$     Percent       \%     Ampersand     \&
%   Acute accent  \'     Left paren    \(     Right paren   \)
%   Asterisk      \*     Plus          \+     Comma         \,
%   Minus         \-     Point         \.     Solidus       \/
%   Colon         \:     Semicolon     \;     Less than     \<
%   Equals        \=     Greater than  \>     Question mark \?
%   Commercial at \@     Left bracket  \[     Backslash     \\
%   Right bracket \]     Circumflex    \^     Underscore    \_
%   Grave accent  \`     Left brace    \{     Vertical bar  \|
%   Right brace   \}     Tilde         \~}
%
% \GetFileInfo{colonequals.drv}
%
% \title{The \xpackage{colonequals} package}
% \date{2006/08/01 v1.0}
% \author{Heiko Oberdiek\\\xemail{heiko.oberdiek at googlemail.com}}
%
% \maketitle
%
% \begin{abstract}
% Package \xpackage{colonequals} defines poor man's symbols
% for math relation symbols such as ``colon equals''.
% The colon is centered around the horizontal math axis.
% \end{abstract}
%
% \tableofcontents
%
% \section{User interface}
%
% \subsection{Introduction}
%
% Math symbols consisting of the colon character can be
% constructed with the colon text character, if the math font
% lacks of the complete symbol. Often, however, the colon text
% character is not centered around the math axis. Especially
% combined with the equals symbol the composed symbol does not
% look symmetrically. Thus this packages defines a colon
% math symbol \cs{ratio} that is centered around the horizontal
% math axis. Also math symbols are provided that consist of the
% colon symbol. The package is not necessary, if the math
% font contains the composed symbols. Examples are \textsf{txfonts}
% (\cite{txfonts}) or \textsf{mathabx} (\cite{mathabx}).
%
% \subsection{Symbols}
%
% All symbols of this package are relation symbols.
% The relation property can be changed by the appropriate
% \TeX\ command \cs{mathbin}, \cs{mathord}, \dots
%
% \begin{center}
% \captionof{table}{Unicode mathematical operators}
% \kern1ex
% \begin{tabular}{l>{\scshape}l>{$}l<{$}l}
%   U+2236 & ratio        & \ratio       & \cs{ratio}       \\
%   U+2237 & proportion   & \coloncolon  & \cs{coloncolon}  \\
%   U+2239 & excess       & \colonminus  & \cs{colonminus}  \\
%   U+2254 & colon equals & \colonequals & \cs{colonequals} \\
%   U+2255 & equals colon & \equalscolon & \cs{equalscolon} \\
% \end{tabular}
% \end{center}
%
% The following grammar generates all symbols that are supported by
% this package:
% \begin{center}
% \captionof{table}{Symbol grammar}
% \kern1ex
% \begin{tabular}{@{}l>{$}r<{$}l@{}}
%   symbols & \coloncolonequals & col \\
%           & \mid & col symbol \\
%           & \mid & symbol col \\
%           & ; & \\[1ex]
%   col     & \coloncolonequals & '\texttt{:}' \\
%           & \mid & '\texttt{::}' \\
%           & ; & \\[1ex]
%   symbol  & \coloncolonequals & '\texttt{=}' \\
%           & \mid & '\texttt{-}' \\
%           & \mid & '$\approx$' \\
%           & \mid & '$\sim$' \\
%           & ; &
% \end{tabular}
% \end{center}
%
% \def\entry#1{\csname #1\endcsname&\cs{#1}\\}
% \def\entryset#1{^^A
%    \entry{colon#1}^^A
%    \entry{coloncolon#1}^^A
%    \entry{#1colon}^^A
%    \entry{#1coloncolon}^^A
% }
% \begin{longtable}{>{$}l<{$}l}
%   \caption{All relation symbols}\\
%   \entry{ratio}
%   \entry{coloncolon}
%   \entryset{equals}
%   \entryset{minus}
%   \entryset{approx}
%   \entryset{sim}
% \end{longtable}
%
% \subsection{Fine tuning}
%
% The distances in composed symbols can be configured:
%
% \begin{declcs}{colonsep}
% \end{declcs}
% Macro \cs{colonsep} is executed between the colon and
% the other symbol.
%
% \begin{declcs}{doublecolonsep}
% \end{declcs}
% Macro \cs{doublecolonsep} controls the distance between
% two colons.
%
% \subsubsection{Example}
% \begin{quote}
%   \verb|\renewcommand*{\colonsep}{\mskip-.5\thinmuskip}|
% \end{quote}
%
%
% \StopEventually{
% }
%
% \section{Implementation}
%
% \subsection{Identification}
%
%    \begin{macrocode}
%<*package>
\NeedsTeXFormat{LaTeX2e}
\ProvidesPackage{colonequals}%
  [2006/08/01 v1.0 Colon equals symbols (HO)]%
%    \end{macrocode}
%
% \subsection{Distance control}
%
%    \begin{macro}{\colonsep}
%    \begin{macrocode}
\newcommand*{\colonsep}{}
%    \end{macrocode}
%    \end{macro}
%
%    \begin{macro}{\doublecolonsep}
%    \begin{macrocode}
\newcommand*{\doublecolonsep}{}
%    \end{macrocode}
%    \end{macro}
%
% \subsection{Centered colons}
%
%    \begin{macrocode}
\def\@center@colon{%
  \mathpalette\@center@math{:}%
}
\def\@center@math#1#2{%
  \vcenter{%
    \m@th
    \hbox{$#1#2$}%
  }%
}
%    \end{macrocode}
%
%    \begin{macro}{\ratio}
%    Because the name \cs{colon} is already in use, the Unicode name
%    \cs{ratio} is used for the centered colon relation symbol.
%    (The \cs{ratio} of package \textsf{calc} is not used outside
%    calc expressions.)
%    \begin{macrocode}
\newcommand*{\ratio}{%
  \ensuremath{%
    \mathrel{%
      \@center@colon
    }%
  }%
}
%    \end{macrocode}
%    \end{macro}
%
%    \begin{macro}{\coloncolon}
%    \begin{macrocode}
\newcommand*{\coloncolon}{%
  \ensuremath{%
    \mathrel{%
      \@center@colon
      \doublecolonsep
      \@center@colon
    }%
  }%
}
%    \end{macrocode}
%    \end{macro}
%
% \subsection{Combined symbols}
%
%    \begin{macrocode}
\def\@make@colon@set#1#2{%
  \begingroup
    \let\@center@colon\relax
    \let\newcommand\relax
    \let\ensuremath\relax
    \let\mathrel\relax
    \let\colonsep\relax
    \let\doublecolonsep\relax
    \def\csx##1{%
      \expandafter\noexpand\csname ##1\endcsname
    }%
    \edef\x{\endgroup
      \newcommand*{\csx{colon#1}}{%
        \ensuremath{%
          \mathrel{%
            \@center@colon
            \colonsep
            {#2}%
          }%
        }%
      }%
      \newcommand*{\csx{coloncolon#1}}{%
        \ensuremath{%
          \mathrel{%
            \@center@colon
            \doublecolonsep
            \@center@colon
            \colonsep
            {#2}%
          }%
        }%
      }%
      \newcommand*{\csx{#1colon}}{%
        \ensuremath{%
          \mathrel{%
            {#2}%
            \colonsep
            \@center@colon
          }%
        }%
      }%
      \newcommand*{\csx{#1coloncolon}}{%
        \ensuremath{%
          \mathrel{%
            {#2}%
            \colonsep
            \@center@colon
            \doublecolonsep
            \@center@colon
          }%
        }%
      }%
    }%
  \x
}
%    \end{macrocode}
%
%    \begin{macrocode}
\@make@colon@set{equals}{=}%
\@make@colon@set{minus}{-}%
\@make@colon@set{approx}{\approx}
\@make@colon@set{sim}{\sim}
%    \end{macrocode}
%
%    \begin{macrocode}
%</package>
%    \end{macrocode}
%
% \section{Installation}
%
% \subsection{Download}
%
% \paragraph{Package.} This package is available on
% CTAN\footnote{\url{ftp://ftp.ctan.org/tex-archive/}}:
% \begin{description}
% \item[\CTAN{macros/latex/contrib/oberdiek/colonequals.dtx}] The source file.
% \item[\CTAN{macros/latex/contrib/oberdiek/colonequals.pdf}] Documentation.
% \end{description}
%
%
% \paragraph{Bundle.} All the packages of the bundle `oberdiek'
% are also available in a TDS compliant ZIP archive. There
% the packages are already unpacked and the documentation files
% are generated. The files and directories obey the TDS standard.
% \begin{description}
% \item[\CTAN{install/macros/latex/contrib/oberdiek.tds.zip}]
% \end{description}
% \emph{TDS} refers to the standard ``A Directory Structure
% for \TeX\ Files'' (\CTAN{tds/tds.pdf}). Directories
% with \xfile{texmf} in their name are usually organized this way.
%
% \subsection{Bundle installation}
%
% \paragraph{Unpacking.} Unpack the \xfile{oberdiek.tds.zip} in the
% TDS tree (also known as \xfile{texmf} tree) of your choice.
% Example (linux):
% \begin{quote}
%   |unzip oberdiek.tds.zip -d ~/texmf|
% \end{quote}
%
% \paragraph{Script installation.}
% Check the directory \xfile{TDS:scripts/oberdiek/} for
% scripts that need further installation steps.
% Package \xpackage{attachfile2} comes with the Perl script
% \xfile{pdfatfi.pl} that should be installed in such a way
% that it can be called as \texttt{pdfatfi}.
% Example (linux):
% \begin{quote}
%   |chmod +x scripts/oberdiek/pdfatfi.pl|\\
%   |cp scripts/oberdiek/pdfatfi.pl /usr/local/bin/|
% \end{quote}
%
% \subsection{Package installation}
%
% \paragraph{Unpacking.} The \xfile{.dtx} file is a self-extracting
% \docstrip\ archive. The files are extracted by running the
% \xfile{.dtx} through \plainTeX:
% \begin{quote}
%   \verb|tex colonequals.dtx|
% \end{quote}
%
% \paragraph{TDS.} Now the different files must be moved into
% the different directories in your installation TDS tree
% (also known as \xfile{texmf} tree):
% \begin{quote}
% \def\t{^^A
% \begin{tabular}{@{}>{\ttfamily}l@{ $\rightarrow$ }>{\ttfamily}l@{}}
%   colonequals.sty & tex/latex/oberdiek/colonequals.sty\\
%   colonequals.pdf & doc/latex/oberdiek/colonequals.pdf\\
%   colonequals.dtx & source/latex/oberdiek/colonequals.dtx\\
% \end{tabular}^^A
% }^^A
% \sbox0{\t}^^A
% \ifdim\wd0>\linewidth
%   \begingroup
%     \advance\linewidth by\leftmargin
%     \advance\linewidth by\rightmargin
%   \edef\x{\endgroup
%     \def\noexpand\lw{\the\linewidth}^^A
%   }\x
%   \def\lwbox{^^A
%     \leavevmode
%     \hbox to \linewidth{^^A
%       \kern-\leftmargin\relax
%       \hss
%       \usebox0
%       \hss
%       \kern-\rightmargin\relax
%     }^^A
%   }^^A
%   \ifdim\wd0>\lw
%     \sbox0{\small\t}^^A
%     \ifdim\wd0>\linewidth
%       \ifdim\wd0>\lw
%         \sbox0{\footnotesize\t}^^A
%         \ifdim\wd0>\linewidth
%           \ifdim\wd0>\lw
%             \sbox0{\scriptsize\t}^^A
%             \ifdim\wd0>\linewidth
%               \ifdim\wd0>\lw
%                 \sbox0{\tiny\t}^^A
%                 \ifdim\wd0>\linewidth
%                   \lwbox
%                 \else
%                   \usebox0
%                 \fi
%               \else
%                 \lwbox
%               \fi
%             \else
%               \usebox0
%             \fi
%           \else
%             \lwbox
%           \fi
%         \else
%           \usebox0
%         \fi
%       \else
%         \lwbox
%       \fi
%     \else
%       \usebox0
%     \fi
%   \else
%     \lwbox
%   \fi
% \else
%   \usebox0
% \fi
% \end{quote}
% If you have a \xfile{docstrip.cfg} that configures and enables \docstrip's
% TDS installing feature, then some files can already be in the right
% place, see the documentation of \docstrip.
%
% \subsection{Refresh file name databases}
%
% If your \TeX~distribution
% (\teTeX, \mikTeX, \dots) relies on file name databases, you must refresh
% these. For example, \teTeX\ users run \verb|texhash| or
% \verb|mktexlsr|.
%
% \subsection{Some details for the interested}
%
% \paragraph{Attached source.}
%
% The PDF documentation on CTAN also includes the
% \xfile{.dtx} source file. It can be extracted by
% AcrobatReader 6 or higher. Another option is \textsf{pdftk},
% e.g. unpack the file into the current directory:
% \begin{quote}
%   \verb|pdftk colonequals.pdf unpack_files output .|
% \end{quote}
%
% \paragraph{Unpacking with \LaTeX.}
% The \xfile{.dtx} chooses its action depending on the format:
% \begin{description}
% \item[\plainTeX:] Run \docstrip\ and extract the files.
% \item[\LaTeX:] Generate the documentation.
% \end{description}
% If you insist on using \LaTeX\ for \docstrip\ (really,
% \docstrip\ does not need \LaTeX), then inform the autodetect routine
% about your intention:
% \begin{quote}
%   \verb|latex \let\install=y\input{colonequals.dtx}|
% \end{quote}
% Do not forget to quote the argument according to the demands
% of your shell.
%
% \paragraph{Generating the documentation.}
% You can use both the \xfile{.dtx} or the \xfile{.drv} to generate
% the documentation. The process can be configured by the
% configuration file \xfile{ltxdoc.cfg}. For instance, put this
% line into this file, if you want to have A4 as paper format:
% \begin{quote}
%   \verb|\PassOptionsToClass{a4paper}{article}|
% \end{quote}
% An example follows how to generate the
% documentation with pdf\LaTeX:
% \begin{quote}
%\begin{verbatim}
%pdflatex colonequals.dtx
%makeindex -s gind.ist colonequals.idx
%pdflatex colonequals.dtx
%makeindex -s gind.ist colonequals.idx
%pdflatex colonequals.dtx
%\end{verbatim}
% \end{quote}
%
% \section{Catalogue}
%
% The following XML file can be used as source for the
% \href{http://mirror.ctan.org/help/Catalogue/catalogue.html}{\TeX\ Catalogue}.
% The elements \texttt{caption} and \texttt{description} are imported
% from the original XML file from the Catalogue.
% The name of the XML file in the Catalogue is \xfile{colonequals.xml}.
%    \begin{macrocode}
%<*catalogue>
<?xml version='1.0' encoding='us-ascii'?>
<!DOCTYPE entry SYSTEM 'catalogue.dtd'>
<entry datestamp='$Date$' modifier='$Author$' id='colonequals'>
  <name>colonequals</name>
  <caption>Colon equals symbols.</caption>
  <authorref id='auth:oberdiek'/>
  <copyright owner='Heiko Oberdiek' year='2006'/>
  <license type='lppl1.3'/>
  <version number='1.0'/>
  <description>
    This package defines poor man&#x2018;s symbols for mathematical
    relation symbols such as &#x201C;colon equals&#x201D;.
    The colon is centered around the horizontal math axis.
    <p/>
    The package is part of the <xref refid='oberdiek'>oberdiek</xref>
    bundle.
  </description>
  <documentation details='Package documentation'
      href='ctan:/macros/latex/contrib/oberdiek/colonequals.pdf'/>
  <ctan file='true' path='/macros/latex/contrib/oberdiek/colonequals.dtx'/>
  <miktex location='oberdiek'/>
  <texlive location='oberdiek'/>
  <install path='/macros/latex/contrib/oberdiek/oberdiek.tds.zip'/>
</entry>
%</catalogue>
%    \end{macrocode}
%
% \begin{thebibliography}{9}
%
% \bibitem{txfonts}
%   Young Ryu: \textit{The TX Fonts};
%   2000/12/15;
%   \CTAN{fonts/txfonts/}.
%
% \bibitem{mathabx}
%   Anthony Phan: \textit{Mathabx font series};
%   2005/05/16;
%   \CTAN{fonts/mathabx/}.
%
% \end{thebibliography}
%
% \begin{History}
%   \begin{Version}{2006/08/01 v1.0}
%   \item
%     First version.
%   \end{Version}
% \end{History}
%
% \PrintIndex
%
% \Finale
\endinput

%        (quote the arguments according to the demands of your shell)
%
% Documentation:
%    (a) If colonequals.drv is present:
%           latex colonequals.drv
%    (b) Without colonequals.drv:
%           latex colonequals.dtx; ...
%    The class ltxdoc loads the configuration file ltxdoc.cfg
%    if available. Here you can specify further options, e.g.
%    use A4 as paper format:
%       \PassOptionsToClass{a4paper}{article}
%
%    Programm calls to get the documentation (example):
%       pdflatex colonequals.dtx
%       makeindex -s gind.ist colonequals.idx
%       pdflatex colonequals.dtx
%       makeindex -s gind.ist colonequals.idx
%       pdflatex colonequals.dtx
%
% Installation:
%    TDS:tex/latex/oberdiek/colonequals.sty
%    TDS:doc/latex/oberdiek/colonequals.pdf
%    TDS:source/latex/oberdiek/colonequals.dtx
%
%<*ignore>
\begingroup
  \catcode123=1 %
  \catcode125=2 %
  \def\x{LaTeX2e}%
\expandafter\endgroup
\ifcase 0\ifx\install y1\fi\expandafter
         \ifx\csname processbatchFile\endcsname\relax\else1\fi
         \ifx\fmtname\x\else 1\fi\relax
\else\csname fi\endcsname
%</ignore>
%<*install>
\input docstrip.tex
\Msg{************************************************************************}
\Msg{* Installation}
\Msg{* Package: colonequals 2006/08/01 v1.0 Colon equals symbols (HO)}
\Msg{************************************************************************}

\keepsilent
\askforoverwritefalse

\let\MetaPrefix\relax
\preamble

This is a generated file.

Project: colonequals
Version: 2006/08/01 v1.0

Copyright (C) 2006 by
   Heiko Oberdiek <heiko.oberdiek at googlemail.com>

This work may be distributed and/or modified under the
conditions of the LaTeX Project Public License, either
version 1.3c of this license or (at your option) any later
version. This version of this license is in
   http://www.latex-project.org/lppl/lppl-1-3c.txt
and the latest version of this license is in
   http://www.latex-project.org/lppl.txt
and version 1.3 or later is part of all distributions of
LaTeX version 2005/12/01 or later.

This work has the LPPL maintenance status "maintained".

This Current Maintainer of this work is Heiko Oberdiek.

This work consists of the main source file colonequals.dtx
and the derived files
   colonequals.sty, colonequals.pdf, colonequals.ins, colonequals.drv.

\endpreamble
\let\MetaPrefix\DoubleperCent

\generate{%
  \file{colonequals.ins}{\from{colonequals.dtx}{install}}%
  \file{colonequals.drv}{\from{colonequals.dtx}{driver}}%
  \usedir{tex/latex/oberdiek}%
  \file{colonequals.sty}{\from{colonequals.dtx}{package}}%
  \nopreamble
  \nopostamble
  \usedir{source/latex/oberdiek/catalogue}%
  \file{colonequals.xml}{\from{colonequals.dtx}{catalogue}}%
}

\catcode32=13\relax% active space
\let =\space%
\Msg{************************************************************************}
\Msg{*}
\Msg{* To finish the installation you have to move the following}
\Msg{* file into a directory searched by TeX:}
\Msg{*}
\Msg{*     colonequals.sty}
\Msg{*}
\Msg{* To produce the documentation run the file `colonequals.drv'}
\Msg{* through LaTeX.}
\Msg{*}
\Msg{* Happy TeXing!}
\Msg{*}
\Msg{************************************************************************}

\endbatchfile
%</install>
%<*ignore>
\fi
%</ignore>
%<*driver>
\NeedsTeXFormat{LaTeX2e}
\ProvidesFile{colonequals.drv}%
  [2006/08/01 v1.0 Colon equals symbols (HO)]%
\documentclass{ltxdoc}
\usepackage{holtxdoc}[2011/11/22]
\usepackage{colonequals}
\usepackage{array}
\usepackage{capt-of}
\usepackage{longtable}
\begin{document}
  \DocInput{colonequals.dtx}%
\end{document}
%</driver>
% \fi
%
% \CheckSum{92}
%
% \CharacterTable
%  {Upper-case    \A\B\C\D\E\F\G\H\I\J\K\L\M\N\O\P\Q\R\S\T\U\V\W\X\Y\Z
%   Lower-case    \a\b\c\d\e\f\g\h\i\j\k\l\m\n\o\p\q\r\s\t\u\v\w\x\y\z
%   Digits        \0\1\2\3\4\5\6\7\8\9
%   Exclamation   \!     Double quote  \"     Hash (number) \#
%   Dollar        \$     Percent       \%     Ampersand     \&
%   Acute accent  \'     Left paren    \(     Right paren   \)
%   Asterisk      \*     Plus          \+     Comma         \,
%   Minus         \-     Point         \.     Solidus       \/
%   Colon         \:     Semicolon     \;     Less than     \<
%   Equals        \=     Greater than  \>     Question mark \?
%   Commercial at \@     Left bracket  \[     Backslash     \\
%   Right bracket \]     Circumflex    \^     Underscore    \_
%   Grave accent  \`     Left brace    \{     Vertical bar  \|
%   Right brace   \}     Tilde         \~}
%
% \GetFileInfo{colonequals.drv}
%
% \title{The \xpackage{colonequals} package}
% \date{2006/08/01 v1.0}
% \author{Heiko Oberdiek\\\xemail{heiko.oberdiek at googlemail.com}}
%
% \maketitle
%
% \begin{abstract}
% Package \xpackage{colonequals} defines poor man's symbols
% for math relation symbols such as ``colon equals''.
% The colon is centered around the horizontal math axis.
% \end{abstract}
%
% \tableofcontents
%
% \section{User interface}
%
% \subsection{Introduction}
%
% Math symbols consisting of the colon character can be
% constructed with the colon text character, if the math font
% lacks of the complete symbol. Often, however, the colon text
% character is not centered around the math axis. Especially
% combined with the equals symbol the composed symbol does not
% look symmetrically. Thus this packages defines a colon
% math symbol \cs{ratio} that is centered around the horizontal
% math axis. Also math symbols are provided that consist of the
% colon symbol. The package is not necessary, if the math
% font contains the composed symbols. Examples are \textsf{txfonts}
% (\cite{txfonts}) or \textsf{mathabx} (\cite{mathabx}).
%
% \subsection{Symbols}
%
% All symbols of this package are relation symbols.
% The relation property can be changed by the appropriate
% \TeX\ command \cs{mathbin}, \cs{mathord}, \dots
%
% \begin{center}
% \captionof{table}{Unicode mathematical operators}
% \kern1ex
% \begin{tabular}{l>{\scshape}l>{$}l<{$}l}
%   U+2236 & ratio        & \ratio       & \cs{ratio}       \\
%   U+2237 & proportion   & \coloncolon  & \cs{coloncolon}  \\
%   U+2239 & excess       & \colonminus  & \cs{colonminus}  \\
%   U+2254 & colon equals & \colonequals & \cs{colonequals} \\
%   U+2255 & equals colon & \equalscolon & \cs{equalscolon} \\
% \end{tabular}
% \end{center}
%
% The following grammar generates all symbols that are supported by
% this package:
% \begin{center}
% \captionof{table}{Symbol grammar}
% \kern1ex
% \begin{tabular}{@{}l>{$}r<{$}l@{}}
%   symbols & \coloncolonequals & col \\
%           & \mid & col symbol \\
%           & \mid & symbol col \\
%           & ; & \\[1ex]
%   col     & \coloncolonequals & '\texttt{:}' \\
%           & \mid & '\texttt{::}' \\
%           & ; & \\[1ex]
%   symbol  & \coloncolonequals & '\texttt{=}' \\
%           & \mid & '\texttt{-}' \\
%           & \mid & '$\approx$' \\
%           & \mid & '$\sim$' \\
%           & ; &
% \end{tabular}
% \end{center}
%
% \def\entry#1{\csname #1\endcsname&\cs{#1}\\}
% \def\entryset#1{^^A
%    \entry{colon#1}^^A
%    \entry{coloncolon#1}^^A
%    \entry{#1colon}^^A
%    \entry{#1coloncolon}^^A
% }
% \begin{longtable}{>{$}l<{$}l}
%   \caption{All relation symbols}\\
%   \entry{ratio}
%   \entry{coloncolon}
%   \entryset{equals}
%   \entryset{minus}
%   \entryset{approx}
%   \entryset{sim}
% \end{longtable}
%
% \subsection{Fine tuning}
%
% The distances in composed symbols can be configured:
%
% \begin{declcs}{colonsep}
% \end{declcs}
% Macro \cs{colonsep} is executed between the colon and
% the other symbol.
%
% \begin{declcs}{doublecolonsep}
% \end{declcs}
% Macro \cs{doublecolonsep} controls the distance between
% two colons.
%
% \subsubsection{Example}
% \begin{quote}
%   \verb|\renewcommand*{\colonsep}{\mskip-.5\thinmuskip}|
% \end{quote}
%
%
% \StopEventually{
% }
%
% \section{Implementation}
%
% \subsection{Identification}
%
%    \begin{macrocode}
%<*package>
\NeedsTeXFormat{LaTeX2e}
\ProvidesPackage{colonequals}%
  [2006/08/01 v1.0 Colon equals symbols (HO)]%
%    \end{macrocode}
%
% \subsection{Distance control}
%
%    \begin{macro}{\colonsep}
%    \begin{macrocode}
\newcommand*{\colonsep}{}
%    \end{macrocode}
%    \end{macro}
%
%    \begin{macro}{\doublecolonsep}
%    \begin{macrocode}
\newcommand*{\doublecolonsep}{}
%    \end{macrocode}
%    \end{macro}
%
% \subsection{Centered colons}
%
%    \begin{macrocode}
\def\@center@colon{%
  \mathpalette\@center@math{:}%
}
\def\@center@math#1#2{%
  \vcenter{%
    \m@th
    \hbox{$#1#2$}%
  }%
}
%    \end{macrocode}
%
%    \begin{macro}{\ratio}
%    Because the name \cs{colon} is already in use, the Unicode name
%    \cs{ratio} is used for the centered colon relation symbol.
%    (The \cs{ratio} of package \textsf{calc} is not used outside
%    calc expressions.)
%    \begin{macrocode}
\newcommand*{\ratio}{%
  \ensuremath{%
    \mathrel{%
      \@center@colon
    }%
  }%
}
%    \end{macrocode}
%    \end{macro}
%
%    \begin{macro}{\coloncolon}
%    \begin{macrocode}
\newcommand*{\coloncolon}{%
  \ensuremath{%
    \mathrel{%
      \@center@colon
      \doublecolonsep
      \@center@colon
    }%
  }%
}
%    \end{macrocode}
%    \end{macro}
%
% \subsection{Combined symbols}
%
%    \begin{macrocode}
\def\@make@colon@set#1#2{%
  \begingroup
    \let\@center@colon\relax
    \let\newcommand\relax
    \let\ensuremath\relax
    \let\mathrel\relax
    \let\colonsep\relax
    \let\doublecolonsep\relax
    \def\csx##1{%
      \expandafter\noexpand\csname ##1\endcsname
    }%
    \edef\x{\endgroup
      \newcommand*{\csx{colon#1}}{%
        \ensuremath{%
          \mathrel{%
            \@center@colon
            \colonsep
            {#2}%
          }%
        }%
      }%
      \newcommand*{\csx{coloncolon#1}}{%
        \ensuremath{%
          \mathrel{%
            \@center@colon
            \doublecolonsep
            \@center@colon
            \colonsep
            {#2}%
          }%
        }%
      }%
      \newcommand*{\csx{#1colon}}{%
        \ensuremath{%
          \mathrel{%
            {#2}%
            \colonsep
            \@center@colon
          }%
        }%
      }%
      \newcommand*{\csx{#1coloncolon}}{%
        \ensuremath{%
          \mathrel{%
            {#2}%
            \colonsep
            \@center@colon
            \doublecolonsep
            \@center@colon
          }%
        }%
      }%
    }%
  \x
}
%    \end{macrocode}
%
%    \begin{macrocode}
\@make@colon@set{equals}{=}%
\@make@colon@set{minus}{-}%
\@make@colon@set{approx}{\approx}
\@make@colon@set{sim}{\sim}
%    \end{macrocode}
%
%    \begin{macrocode}
%</package>
%    \end{macrocode}
%
% \section{Installation}
%
% \subsection{Download}
%
% \paragraph{Package.} This package is available on
% CTAN\footnote{\url{ftp://ftp.ctan.org/tex-archive/}}:
% \begin{description}
% \item[\CTAN{macros/latex/contrib/oberdiek/colonequals.dtx}] The source file.
% \item[\CTAN{macros/latex/contrib/oberdiek/colonequals.pdf}] Documentation.
% \end{description}
%
%
% \paragraph{Bundle.} All the packages of the bundle `oberdiek'
% are also available in a TDS compliant ZIP archive. There
% the packages are already unpacked and the documentation files
% are generated. The files and directories obey the TDS standard.
% \begin{description}
% \item[\CTAN{install/macros/latex/contrib/oberdiek.tds.zip}]
% \end{description}
% \emph{TDS} refers to the standard ``A Directory Structure
% for \TeX\ Files'' (\CTAN{tds/tds.pdf}). Directories
% with \xfile{texmf} in their name are usually organized this way.
%
% \subsection{Bundle installation}
%
% \paragraph{Unpacking.} Unpack the \xfile{oberdiek.tds.zip} in the
% TDS tree (also known as \xfile{texmf} tree) of your choice.
% Example (linux):
% \begin{quote}
%   |unzip oberdiek.tds.zip -d ~/texmf|
% \end{quote}
%
% \paragraph{Script installation.}
% Check the directory \xfile{TDS:scripts/oberdiek/} for
% scripts that need further installation steps.
% Package \xpackage{attachfile2} comes with the Perl script
% \xfile{pdfatfi.pl} that should be installed in such a way
% that it can be called as \texttt{pdfatfi}.
% Example (linux):
% \begin{quote}
%   |chmod +x scripts/oberdiek/pdfatfi.pl|\\
%   |cp scripts/oberdiek/pdfatfi.pl /usr/local/bin/|
% \end{quote}
%
% \subsection{Package installation}
%
% \paragraph{Unpacking.} The \xfile{.dtx} file is a self-extracting
% \docstrip\ archive. The files are extracted by running the
% \xfile{.dtx} through \plainTeX:
% \begin{quote}
%   \verb|tex colonequals.dtx|
% \end{quote}
%
% \paragraph{TDS.} Now the different files must be moved into
% the different directories in your installation TDS tree
% (also known as \xfile{texmf} tree):
% \begin{quote}
% \def\t{^^A
% \begin{tabular}{@{}>{\ttfamily}l@{ $\rightarrow$ }>{\ttfamily}l@{}}
%   colonequals.sty & tex/latex/oberdiek/colonequals.sty\\
%   colonequals.pdf & doc/latex/oberdiek/colonequals.pdf\\
%   colonequals.dtx & source/latex/oberdiek/colonequals.dtx\\
% \end{tabular}^^A
% }^^A
% \sbox0{\t}^^A
% \ifdim\wd0>\linewidth
%   \begingroup
%     \advance\linewidth by\leftmargin
%     \advance\linewidth by\rightmargin
%   \edef\x{\endgroup
%     \def\noexpand\lw{\the\linewidth}^^A
%   }\x
%   \def\lwbox{^^A
%     \leavevmode
%     \hbox to \linewidth{^^A
%       \kern-\leftmargin\relax
%       \hss
%       \usebox0
%       \hss
%       \kern-\rightmargin\relax
%     }^^A
%   }^^A
%   \ifdim\wd0>\lw
%     \sbox0{\small\t}^^A
%     \ifdim\wd0>\linewidth
%       \ifdim\wd0>\lw
%         \sbox0{\footnotesize\t}^^A
%         \ifdim\wd0>\linewidth
%           \ifdim\wd0>\lw
%             \sbox0{\scriptsize\t}^^A
%             \ifdim\wd0>\linewidth
%               \ifdim\wd0>\lw
%                 \sbox0{\tiny\t}^^A
%                 \ifdim\wd0>\linewidth
%                   \lwbox
%                 \else
%                   \usebox0
%                 \fi
%               \else
%                 \lwbox
%               \fi
%             \else
%               \usebox0
%             \fi
%           \else
%             \lwbox
%           \fi
%         \else
%           \usebox0
%         \fi
%       \else
%         \lwbox
%       \fi
%     \else
%       \usebox0
%     \fi
%   \else
%     \lwbox
%   \fi
% \else
%   \usebox0
% \fi
% \end{quote}
% If you have a \xfile{docstrip.cfg} that configures and enables \docstrip's
% TDS installing feature, then some files can already be in the right
% place, see the documentation of \docstrip.
%
% \subsection{Refresh file name databases}
%
% If your \TeX~distribution
% (\teTeX, \mikTeX, \dots) relies on file name databases, you must refresh
% these. For example, \teTeX\ users run \verb|texhash| or
% \verb|mktexlsr|.
%
% \subsection{Some details for the interested}
%
% \paragraph{Attached source.}
%
% The PDF documentation on CTAN also includes the
% \xfile{.dtx} source file. It can be extracted by
% AcrobatReader 6 or higher. Another option is \textsf{pdftk},
% e.g. unpack the file into the current directory:
% \begin{quote}
%   \verb|pdftk colonequals.pdf unpack_files output .|
% \end{quote}
%
% \paragraph{Unpacking with \LaTeX.}
% The \xfile{.dtx} chooses its action depending on the format:
% \begin{description}
% \item[\plainTeX:] Run \docstrip\ and extract the files.
% \item[\LaTeX:] Generate the documentation.
% \end{description}
% If you insist on using \LaTeX\ for \docstrip\ (really,
% \docstrip\ does not need \LaTeX), then inform the autodetect routine
% about your intention:
% \begin{quote}
%   \verb|latex \let\install=y% \iffalse meta-comment
%
% File: colonequals.dtx
% Version: 2006/08/01 v1.0
% Info: Colon equals symbols
%
% Copyright (C) 2006 by
%    Heiko Oberdiek <heiko.oberdiek at googlemail.com>
%
% This work may be distributed and/or modified under the
% conditions of the LaTeX Project Public License, either
% version 1.3c of this license or (at your option) any later
% version. This version of this license is in
%    http://www.latex-project.org/lppl/lppl-1-3c.txt
% and the latest version of this license is in
%    http://www.latex-project.org/lppl.txt
% and version 1.3 or later is part of all distributions of
% LaTeX version 2005/12/01 or later.
%
% This work has the LPPL maintenance status "maintained".
%
% This Current Maintainer of this work is Heiko Oberdiek.
%
% This work consists of the main source file colonequals.dtx
% and the derived files
%    colonequals.sty, colonequals.pdf, colonequals.ins, colonequals.drv.
%
% Distribution:
%    CTAN:macros/latex/contrib/oberdiek/colonequals.dtx
%    CTAN:macros/latex/contrib/oberdiek/colonequals.pdf
%
% Unpacking:
%    (a) If colonequals.ins is present:
%           tex colonequals.ins
%    (b) Without colonequals.ins:
%           tex colonequals.dtx
%    (c) If you insist on using LaTeX
%           latex \let\install=y\input{colonequals.dtx}
%        (quote the arguments according to the demands of your shell)
%
% Documentation:
%    (a) If colonequals.drv is present:
%           latex colonequals.drv
%    (b) Without colonequals.drv:
%           latex colonequals.dtx; ...
%    The class ltxdoc loads the configuration file ltxdoc.cfg
%    if available. Here you can specify further options, e.g.
%    use A4 as paper format:
%       \PassOptionsToClass{a4paper}{article}
%
%    Programm calls to get the documentation (example):
%       pdflatex colonequals.dtx
%       makeindex -s gind.ist colonequals.idx
%       pdflatex colonequals.dtx
%       makeindex -s gind.ist colonequals.idx
%       pdflatex colonequals.dtx
%
% Installation:
%    TDS:tex/latex/oberdiek/colonequals.sty
%    TDS:doc/latex/oberdiek/colonequals.pdf
%    TDS:source/latex/oberdiek/colonequals.dtx
%
%<*ignore>
\begingroup
  \catcode123=1 %
  \catcode125=2 %
  \def\x{LaTeX2e}%
\expandafter\endgroup
\ifcase 0\ifx\install y1\fi\expandafter
         \ifx\csname processbatchFile\endcsname\relax\else1\fi
         \ifx\fmtname\x\else 1\fi\relax
\else\csname fi\endcsname
%</ignore>
%<*install>
\input docstrip.tex
\Msg{************************************************************************}
\Msg{* Installation}
\Msg{* Package: colonequals 2006/08/01 v1.0 Colon equals symbols (HO)}
\Msg{************************************************************************}

\keepsilent
\askforoverwritefalse

\let\MetaPrefix\relax
\preamble

This is a generated file.

Project: colonequals
Version: 2006/08/01 v1.0

Copyright (C) 2006 by
   Heiko Oberdiek <heiko.oberdiek at googlemail.com>

This work may be distributed and/or modified under the
conditions of the LaTeX Project Public License, either
version 1.3c of this license or (at your option) any later
version. This version of this license is in
   http://www.latex-project.org/lppl/lppl-1-3c.txt
and the latest version of this license is in
   http://www.latex-project.org/lppl.txt
and version 1.3 or later is part of all distributions of
LaTeX version 2005/12/01 or later.

This work has the LPPL maintenance status "maintained".

This Current Maintainer of this work is Heiko Oberdiek.

This work consists of the main source file colonequals.dtx
and the derived files
   colonequals.sty, colonequals.pdf, colonequals.ins, colonequals.drv.

\endpreamble
\let\MetaPrefix\DoubleperCent

\generate{%
  \file{colonequals.ins}{\from{colonequals.dtx}{install}}%
  \file{colonequals.drv}{\from{colonequals.dtx}{driver}}%
  \usedir{tex/latex/oberdiek}%
  \file{colonequals.sty}{\from{colonequals.dtx}{package}}%
  \nopreamble
  \nopostamble
  \usedir{source/latex/oberdiek/catalogue}%
  \file{colonequals.xml}{\from{colonequals.dtx}{catalogue}}%
}

\catcode32=13\relax% active space
\let =\space%
\Msg{************************************************************************}
\Msg{*}
\Msg{* To finish the installation you have to move the following}
\Msg{* file into a directory searched by TeX:}
\Msg{*}
\Msg{*     colonequals.sty}
\Msg{*}
\Msg{* To produce the documentation run the file `colonequals.drv'}
\Msg{* through LaTeX.}
\Msg{*}
\Msg{* Happy TeXing!}
\Msg{*}
\Msg{************************************************************************}

\endbatchfile
%</install>
%<*ignore>
\fi
%</ignore>
%<*driver>
\NeedsTeXFormat{LaTeX2e}
\ProvidesFile{colonequals.drv}%
  [2006/08/01 v1.0 Colon equals symbols (HO)]%
\documentclass{ltxdoc}
\usepackage{holtxdoc}[2011/11/22]
\usepackage{colonequals}
\usepackage{array}
\usepackage{capt-of}
\usepackage{longtable}
\begin{document}
  \DocInput{colonequals.dtx}%
\end{document}
%</driver>
% \fi
%
% \CheckSum{92}
%
% \CharacterTable
%  {Upper-case    \A\B\C\D\E\F\G\H\I\J\K\L\M\N\O\P\Q\R\S\T\U\V\W\X\Y\Z
%   Lower-case    \a\b\c\d\e\f\g\h\i\j\k\l\m\n\o\p\q\r\s\t\u\v\w\x\y\z
%   Digits        \0\1\2\3\4\5\6\7\8\9
%   Exclamation   \!     Double quote  \"     Hash (number) \#
%   Dollar        \$     Percent       \%     Ampersand     \&
%   Acute accent  \'     Left paren    \(     Right paren   \)
%   Asterisk      \*     Plus          \+     Comma         \,
%   Minus         \-     Point         \.     Solidus       \/
%   Colon         \:     Semicolon     \;     Less than     \<
%   Equals        \=     Greater than  \>     Question mark \?
%   Commercial at \@     Left bracket  \[     Backslash     \\
%   Right bracket \]     Circumflex    \^     Underscore    \_
%   Grave accent  \`     Left brace    \{     Vertical bar  \|
%   Right brace   \}     Tilde         \~}
%
% \GetFileInfo{colonequals.drv}
%
% \title{The \xpackage{colonequals} package}
% \date{2006/08/01 v1.0}
% \author{Heiko Oberdiek\\\xemail{heiko.oberdiek at googlemail.com}}
%
% \maketitle
%
% \begin{abstract}
% Package \xpackage{colonequals} defines poor man's symbols
% for math relation symbols such as ``colon equals''.
% The colon is centered around the horizontal math axis.
% \end{abstract}
%
% \tableofcontents
%
% \section{User interface}
%
% \subsection{Introduction}
%
% Math symbols consisting of the colon character can be
% constructed with the colon text character, if the math font
% lacks of the complete symbol. Often, however, the colon text
% character is not centered around the math axis. Especially
% combined with the equals symbol the composed symbol does not
% look symmetrically. Thus this packages defines a colon
% math symbol \cs{ratio} that is centered around the horizontal
% math axis. Also math symbols are provided that consist of the
% colon symbol. The package is not necessary, if the math
% font contains the composed symbols. Examples are \textsf{txfonts}
% (\cite{txfonts}) or \textsf{mathabx} (\cite{mathabx}).
%
% \subsection{Symbols}
%
% All symbols of this package are relation symbols.
% The relation property can be changed by the appropriate
% \TeX\ command \cs{mathbin}, \cs{mathord}, \dots
%
% \begin{center}
% \captionof{table}{Unicode mathematical operators}
% \kern1ex
% \begin{tabular}{l>{\scshape}l>{$}l<{$}l}
%   U+2236 & ratio        & \ratio       & \cs{ratio}       \\
%   U+2237 & proportion   & \coloncolon  & \cs{coloncolon}  \\
%   U+2239 & excess       & \colonminus  & \cs{colonminus}  \\
%   U+2254 & colon equals & \colonequals & \cs{colonequals} \\
%   U+2255 & equals colon & \equalscolon & \cs{equalscolon} \\
% \end{tabular}
% \end{center}
%
% The following grammar generates all symbols that are supported by
% this package:
% \begin{center}
% \captionof{table}{Symbol grammar}
% \kern1ex
% \begin{tabular}{@{}l>{$}r<{$}l@{}}
%   symbols & \coloncolonequals & col \\
%           & \mid & col symbol \\
%           & \mid & symbol col \\
%           & ; & \\[1ex]
%   col     & \coloncolonequals & '\texttt{:}' \\
%           & \mid & '\texttt{::}' \\
%           & ; & \\[1ex]
%   symbol  & \coloncolonequals & '\texttt{=}' \\
%           & \mid & '\texttt{-}' \\
%           & \mid & '$\approx$' \\
%           & \mid & '$\sim$' \\
%           & ; &
% \end{tabular}
% \end{center}
%
% \def\entry#1{\csname #1\endcsname&\cs{#1}\\}
% \def\entryset#1{^^A
%    \entry{colon#1}^^A
%    \entry{coloncolon#1}^^A
%    \entry{#1colon}^^A
%    \entry{#1coloncolon}^^A
% }
% \begin{longtable}{>{$}l<{$}l}
%   \caption{All relation symbols}\\
%   \entry{ratio}
%   \entry{coloncolon}
%   \entryset{equals}
%   \entryset{minus}
%   \entryset{approx}
%   \entryset{sim}
% \end{longtable}
%
% \subsection{Fine tuning}
%
% The distances in composed symbols can be configured:
%
% \begin{declcs}{colonsep}
% \end{declcs}
% Macro \cs{colonsep} is executed between the colon and
% the other symbol.
%
% \begin{declcs}{doublecolonsep}
% \end{declcs}
% Macro \cs{doublecolonsep} controls the distance between
% two colons.
%
% \subsubsection{Example}
% \begin{quote}
%   \verb|\renewcommand*{\colonsep}{\mskip-.5\thinmuskip}|
% \end{quote}
%
%
% \StopEventually{
% }
%
% \section{Implementation}
%
% \subsection{Identification}
%
%    \begin{macrocode}
%<*package>
\NeedsTeXFormat{LaTeX2e}
\ProvidesPackage{colonequals}%
  [2006/08/01 v1.0 Colon equals symbols (HO)]%
%    \end{macrocode}
%
% \subsection{Distance control}
%
%    \begin{macro}{\colonsep}
%    \begin{macrocode}
\newcommand*{\colonsep}{}
%    \end{macrocode}
%    \end{macro}
%
%    \begin{macro}{\doublecolonsep}
%    \begin{macrocode}
\newcommand*{\doublecolonsep}{}
%    \end{macrocode}
%    \end{macro}
%
% \subsection{Centered colons}
%
%    \begin{macrocode}
\def\@center@colon{%
  \mathpalette\@center@math{:}%
}
\def\@center@math#1#2{%
  \vcenter{%
    \m@th
    \hbox{$#1#2$}%
  }%
}
%    \end{macrocode}
%
%    \begin{macro}{\ratio}
%    Because the name \cs{colon} is already in use, the Unicode name
%    \cs{ratio} is used for the centered colon relation symbol.
%    (The \cs{ratio} of package \textsf{calc} is not used outside
%    calc expressions.)
%    \begin{macrocode}
\newcommand*{\ratio}{%
  \ensuremath{%
    \mathrel{%
      \@center@colon
    }%
  }%
}
%    \end{macrocode}
%    \end{macro}
%
%    \begin{macro}{\coloncolon}
%    \begin{macrocode}
\newcommand*{\coloncolon}{%
  \ensuremath{%
    \mathrel{%
      \@center@colon
      \doublecolonsep
      \@center@colon
    }%
  }%
}
%    \end{macrocode}
%    \end{macro}
%
% \subsection{Combined symbols}
%
%    \begin{macrocode}
\def\@make@colon@set#1#2{%
  \begingroup
    \let\@center@colon\relax
    \let\newcommand\relax
    \let\ensuremath\relax
    \let\mathrel\relax
    \let\colonsep\relax
    \let\doublecolonsep\relax
    \def\csx##1{%
      \expandafter\noexpand\csname ##1\endcsname
    }%
    \edef\x{\endgroup
      \newcommand*{\csx{colon#1}}{%
        \ensuremath{%
          \mathrel{%
            \@center@colon
            \colonsep
            {#2}%
          }%
        }%
      }%
      \newcommand*{\csx{coloncolon#1}}{%
        \ensuremath{%
          \mathrel{%
            \@center@colon
            \doublecolonsep
            \@center@colon
            \colonsep
            {#2}%
          }%
        }%
      }%
      \newcommand*{\csx{#1colon}}{%
        \ensuremath{%
          \mathrel{%
            {#2}%
            \colonsep
            \@center@colon
          }%
        }%
      }%
      \newcommand*{\csx{#1coloncolon}}{%
        \ensuremath{%
          \mathrel{%
            {#2}%
            \colonsep
            \@center@colon
            \doublecolonsep
            \@center@colon
          }%
        }%
      }%
    }%
  \x
}
%    \end{macrocode}
%
%    \begin{macrocode}
\@make@colon@set{equals}{=}%
\@make@colon@set{minus}{-}%
\@make@colon@set{approx}{\approx}
\@make@colon@set{sim}{\sim}
%    \end{macrocode}
%
%    \begin{macrocode}
%</package>
%    \end{macrocode}
%
% \section{Installation}
%
% \subsection{Download}
%
% \paragraph{Package.} This package is available on
% CTAN\footnote{\url{ftp://ftp.ctan.org/tex-archive/}}:
% \begin{description}
% \item[\CTAN{macros/latex/contrib/oberdiek/colonequals.dtx}] The source file.
% \item[\CTAN{macros/latex/contrib/oberdiek/colonequals.pdf}] Documentation.
% \end{description}
%
%
% \paragraph{Bundle.} All the packages of the bundle `oberdiek'
% are also available in a TDS compliant ZIP archive. There
% the packages are already unpacked and the documentation files
% are generated. The files and directories obey the TDS standard.
% \begin{description}
% \item[\CTAN{install/macros/latex/contrib/oberdiek.tds.zip}]
% \end{description}
% \emph{TDS} refers to the standard ``A Directory Structure
% for \TeX\ Files'' (\CTAN{tds/tds.pdf}). Directories
% with \xfile{texmf} in their name are usually organized this way.
%
% \subsection{Bundle installation}
%
% \paragraph{Unpacking.} Unpack the \xfile{oberdiek.tds.zip} in the
% TDS tree (also known as \xfile{texmf} tree) of your choice.
% Example (linux):
% \begin{quote}
%   |unzip oberdiek.tds.zip -d ~/texmf|
% \end{quote}
%
% \paragraph{Script installation.}
% Check the directory \xfile{TDS:scripts/oberdiek/} for
% scripts that need further installation steps.
% Package \xpackage{attachfile2} comes with the Perl script
% \xfile{pdfatfi.pl} that should be installed in such a way
% that it can be called as \texttt{pdfatfi}.
% Example (linux):
% \begin{quote}
%   |chmod +x scripts/oberdiek/pdfatfi.pl|\\
%   |cp scripts/oberdiek/pdfatfi.pl /usr/local/bin/|
% \end{quote}
%
% \subsection{Package installation}
%
% \paragraph{Unpacking.} The \xfile{.dtx} file is a self-extracting
% \docstrip\ archive. The files are extracted by running the
% \xfile{.dtx} through \plainTeX:
% \begin{quote}
%   \verb|tex colonequals.dtx|
% \end{quote}
%
% \paragraph{TDS.} Now the different files must be moved into
% the different directories in your installation TDS tree
% (also known as \xfile{texmf} tree):
% \begin{quote}
% \def\t{^^A
% \begin{tabular}{@{}>{\ttfamily}l@{ $\rightarrow$ }>{\ttfamily}l@{}}
%   colonequals.sty & tex/latex/oberdiek/colonequals.sty\\
%   colonequals.pdf & doc/latex/oberdiek/colonequals.pdf\\
%   colonequals.dtx & source/latex/oberdiek/colonequals.dtx\\
% \end{tabular}^^A
% }^^A
% \sbox0{\t}^^A
% \ifdim\wd0>\linewidth
%   \begingroup
%     \advance\linewidth by\leftmargin
%     \advance\linewidth by\rightmargin
%   \edef\x{\endgroup
%     \def\noexpand\lw{\the\linewidth}^^A
%   }\x
%   \def\lwbox{^^A
%     \leavevmode
%     \hbox to \linewidth{^^A
%       \kern-\leftmargin\relax
%       \hss
%       \usebox0
%       \hss
%       \kern-\rightmargin\relax
%     }^^A
%   }^^A
%   \ifdim\wd0>\lw
%     \sbox0{\small\t}^^A
%     \ifdim\wd0>\linewidth
%       \ifdim\wd0>\lw
%         \sbox0{\footnotesize\t}^^A
%         \ifdim\wd0>\linewidth
%           \ifdim\wd0>\lw
%             \sbox0{\scriptsize\t}^^A
%             \ifdim\wd0>\linewidth
%               \ifdim\wd0>\lw
%                 \sbox0{\tiny\t}^^A
%                 \ifdim\wd0>\linewidth
%                   \lwbox
%                 \else
%                   \usebox0
%                 \fi
%               \else
%                 \lwbox
%               \fi
%             \else
%               \usebox0
%             \fi
%           \else
%             \lwbox
%           \fi
%         \else
%           \usebox0
%         \fi
%       \else
%         \lwbox
%       \fi
%     \else
%       \usebox0
%     \fi
%   \else
%     \lwbox
%   \fi
% \else
%   \usebox0
% \fi
% \end{quote}
% If you have a \xfile{docstrip.cfg} that configures and enables \docstrip's
% TDS installing feature, then some files can already be in the right
% place, see the documentation of \docstrip.
%
% \subsection{Refresh file name databases}
%
% If your \TeX~distribution
% (\teTeX, \mikTeX, \dots) relies on file name databases, you must refresh
% these. For example, \teTeX\ users run \verb|texhash| or
% \verb|mktexlsr|.
%
% \subsection{Some details for the interested}
%
% \paragraph{Attached source.}
%
% The PDF documentation on CTAN also includes the
% \xfile{.dtx} source file. It can be extracted by
% AcrobatReader 6 or higher. Another option is \textsf{pdftk},
% e.g. unpack the file into the current directory:
% \begin{quote}
%   \verb|pdftk colonequals.pdf unpack_files output .|
% \end{quote}
%
% \paragraph{Unpacking with \LaTeX.}
% The \xfile{.dtx} chooses its action depending on the format:
% \begin{description}
% \item[\plainTeX:] Run \docstrip\ and extract the files.
% \item[\LaTeX:] Generate the documentation.
% \end{description}
% If you insist on using \LaTeX\ for \docstrip\ (really,
% \docstrip\ does not need \LaTeX), then inform the autodetect routine
% about your intention:
% \begin{quote}
%   \verb|latex \let\install=y\input{colonequals.dtx}|
% \end{quote}
% Do not forget to quote the argument according to the demands
% of your shell.
%
% \paragraph{Generating the documentation.}
% You can use both the \xfile{.dtx} or the \xfile{.drv} to generate
% the documentation. The process can be configured by the
% configuration file \xfile{ltxdoc.cfg}. For instance, put this
% line into this file, if you want to have A4 as paper format:
% \begin{quote}
%   \verb|\PassOptionsToClass{a4paper}{article}|
% \end{quote}
% An example follows how to generate the
% documentation with pdf\LaTeX:
% \begin{quote}
%\begin{verbatim}
%pdflatex colonequals.dtx
%makeindex -s gind.ist colonequals.idx
%pdflatex colonequals.dtx
%makeindex -s gind.ist colonequals.idx
%pdflatex colonequals.dtx
%\end{verbatim}
% \end{quote}
%
% \section{Catalogue}
%
% The following XML file can be used as source for the
% \href{http://mirror.ctan.org/help/Catalogue/catalogue.html}{\TeX\ Catalogue}.
% The elements \texttt{caption} and \texttt{description} are imported
% from the original XML file from the Catalogue.
% The name of the XML file in the Catalogue is \xfile{colonequals.xml}.
%    \begin{macrocode}
%<*catalogue>
<?xml version='1.0' encoding='us-ascii'?>
<!DOCTYPE entry SYSTEM 'catalogue.dtd'>
<entry datestamp='$Date$' modifier='$Author$' id='colonequals'>
  <name>colonequals</name>
  <caption>Colon equals symbols.</caption>
  <authorref id='auth:oberdiek'/>
  <copyright owner='Heiko Oberdiek' year='2006'/>
  <license type='lppl1.3'/>
  <version number='1.0'/>
  <description>
    This package defines poor man&#x2018;s symbols for mathematical
    relation symbols such as &#x201C;colon equals&#x201D;.
    The colon is centered around the horizontal math axis.
    <p/>
    The package is part of the <xref refid='oberdiek'>oberdiek</xref>
    bundle.
  </description>
  <documentation details='Package documentation'
      href='ctan:/macros/latex/contrib/oberdiek/colonequals.pdf'/>
  <ctan file='true' path='/macros/latex/contrib/oberdiek/colonequals.dtx'/>
  <miktex location='oberdiek'/>
  <texlive location='oberdiek'/>
  <install path='/macros/latex/contrib/oberdiek/oberdiek.tds.zip'/>
</entry>
%</catalogue>
%    \end{macrocode}
%
% \begin{thebibliography}{9}
%
% \bibitem{txfonts}
%   Young Ryu: \textit{The TX Fonts};
%   2000/12/15;
%   \CTAN{fonts/txfonts/}.
%
% \bibitem{mathabx}
%   Anthony Phan: \textit{Mathabx font series};
%   2005/05/16;
%   \CTAN{fonts/mathabx/}.
%
% \end{thebibliography}
%
% \begin{History}
%   \begin{Version}{2006/08/01 v1.0}
%   \item
%     First version.
%   \end{Version}
% \end{History}
%
% \PrintIndex
%
% \Finale
\endinput
|
% \end{quote}
% Do not forget to quote the argument according to the demands
% of your shell.
%
% \paragraph{Generating the documentation.}
% You can use both the \xfile{.dtx} or the \xfile{.drv} to generate
% the documentation. The process can be configured by the
% configuration file \xfile{ltxdoc.cfg}. For instance, put this
% line into this file, if you want to have A4 as paper format:
% \begin{quote}
%   \verb|\PassOptionsToClass{a4paper}{article}|
% \end{quote}
% An example follows how to generate the
% documentation with pdf\LaTeX:
% \begin{quote}
%\begin{verbatim}
%pdflatex colonequals.dtx
%makeindex -s gind.ist colonequals.idx
%pdflatex colonequals.dtx
%makeindex -s gind.ist colonequals.idx
%pdflatex colonequals.dtx
%\end{verbatim}
% \end{quote}
%
% \section{Catalogue}
%
% The following XML file can be used as source for the
% \href{http://mirror.ctan.org/help/Catalogue/catalogue.html}{\TeX\ Catalogue}.
% The elements \texttt{caption} and \texttt{description} are imported
% from the original XML file from the Catalogue.
% The name of the XML file in the Catalogue is \xfile{colonequals.xml}.
%    \begin{macrocode}
%<*catalogue>
<?xml version='1.0' encoding='us-ascii'?>
<!DOCTYPE entry SYSTEM 'catalogue.dtd'>
<entry datestamp='$Date$' modifier='$Author$' id='colonequals'>
  <name>colonequals</name>
  <caption>Colon equals symbols.</caption>
  <authorref id='auth:oberdiek'/>
  <copyright owner='Heiko Oberdiek' year='2006'/>
  <license type='lppl1.3'/>
  <version number='1.0'/>
  <description>
    This package defines poor man&#x2018;s symbols for mathematical
    relation symbols such as &#x201C;colon equals&#x201D;.
    The colon is centered around the horizontal math axis.
    <p/>
    The package is part of the <xref refid='oberdiek'>oberdiek</xref>
    bundle.
  </description>
  <documentation details='Package documentation'
      href='ctan:/macros/latex/contrib/oberdiek/colonequals.pdf'/>
  <ctan file='true' path='/macros/latex/contrib/oberdiek/colonequals.dtx'/>
  <miktex location='oberdiek'/>
  <texlive location='oberdiek'/>
  <install path='/macros/latex/contrib/oberdiek/oberdiek.tds.zip'/>
</entry>
%</catalogue>
%    \end{macrocode}
%
% \begin{thebibliography}{9}
%
% \bibitem{txfonts}
%   Young Ryu: \textit{The TX Fonts};
%   2000/12/15;
%   \CTAN{fonts/txfonts/}.
%
% \bibitem{mathabx}
%   Anthony Phan: \textit{Mathabx font series};
%   2005/05/16;
%   \CTAN{fonts/mathabx/}.
%
% \end{thebibliography}
%
% \begin{History}
%   \begin{Version}{2006/08/01 v1.0}
%   \item
%     First version.
%   \end{Version}
% \end{History}
%
% \PrintIndex
%
% \Finale
\endinput

%        (quote the arguments according to the demands of your shell)
%
% Documentation:
%    (a) If colonequals.drv is present:
%           latex colonequals.drv
%    (b) Without colonequals.drv:
%           latex colonequals.dtx; ...
%    The class ltxdoc loads the configuration file ltxdoc.cfg
%    if available. Here you can specify further options, e.g.
%    use A4 as paper format:
%       \PassOptionsToClass{a4paper}{article}
%
%    Programm calls to get the documentation (example):
%       pdflatex colonequals.dtx
%       makeindex -s gind.ist colonequals.idx
%       pdflatex colonequals.dtx
%       makeindex -s gind.ist colonequals.idx
%       pdflatex colonequals.dtx
%
% Installation:
%    TDS:tex/latex/oberdiek/colonequals.sty
%    TDS:doc/latex/oberdiek/colonequals.pdf
%    TDS:source/latex/oberdiek/colonequals.dtx
%
%<*ignore>
\begingroup
  \catcode123=1 %
  \catcode125=2 %
  \def\x{LaTeX2e}%
\expandafter\endgroup
\ifcase 0\ifx\install y1\fi\expandafter
         \ifx\csname processbatchFile\endcsname\relax\else1\fi
         \ifx\fmtname\x\else 1\fi\relax
\else\csname fi\endcsname
%</ignore>
%<*install>
\input docstrip.tex
\Msg{************************************************************************}
\Msg{* Installation}
\Msg{* Package: colonequals 2006/08/01 v1.0 Colon equals symbols (HO)}
\Msg{************************************************************************}

\keepsilent
\askforoverwritefalse

\let\MetaPrefix\relax
\preamble

This is a generated file.

Project: colonequals
Version: 2006/08/01 v1.0

Copyright (C) 2006 by
   Heiko Oberdiek <heiko.oberdiek at googlemail.com>

This work may be distributed and/or modified under the
conditions of the LaTeX Project Public License, either
version 1.3c of this license or (at your option) any later
version. This version of this license is in
   http://www.latex-project.org/lppl/lppl-1-3c.txt
and the latest version of this license is in
   http://www.latex-project.org/lppl.txt
and version 1.3 or later is part of all distributions of
LaTeX version 2005/12/01 or later.

This work has the LPPL maintenance status "maintained".

This Current Maintainer of this work is Heiko Oberdiek.

This work consists of the main source file colonequals.dtx
and the derived files
   colonequals.sty, colonequals.pdf, colonequals.ins, colonequals.drv.

\endpreamble
\let\MetaPrefix\DoubleperCent

\generate{%
  \file{colonequals.ins}{\from{colonequals.dtx}{install}}%
  \file{colonequals.drv}{\from{colonequals.dtx}{driver}}%
  \usedir{tex/latex/oberdiek}%
  \file{colonequals.sty}{\from{colonequals.dtx}{package}}%
  \nopreamble
  \nopostamble
  \usedir{source/latex/oberdiek/catalogue}%
  \file{colonequals.xml}{\from{colonequals.dtx}{catalogue}}%
}

\catcode32=13\relax% active space
\let =\space%
\Msg{************************************************************************}
\Msg{*}
\Msg{* To finish the installation you have to move the following}
\Msg{* file into a directory searched by TeX:}
\Msg{*}
\Msg{*     colonequals.sty}
\Msg{*}
\Msg{* To produce the documentation run the file `colonequals.drv'}
\Msg{* through LaTeX.}
\Msg{*}
\Msg{* Happy TeXing!}
\Msg{*}
\Msg{************************************************************************}

\endbatchfile
%</install>
%<*ignore>
\fi
%</ignore>
%<*driver>
\NeedsTeXFormat{LaTeX2e}
\ProvidesFile{colonequals.drv}%
  [2006/08/01 v1.0 Colon equals symbols (HO)]%
\documentclass{ltxdoc}
\usepackage{holtxdoc}[2011/11/22]
\usepackage{colonequals}
\usepackage{array}
\usepackage{capt-of}
\usepackage{longtable}
\begin{document}
  \DocInput{colonequals.dtx}%
\end{document}
%</driver>
% \fi
%
% \CheckSum{92}
%
% \CharacterTable
%  {Upper-case    \A\B\C\D\E\F\G\H\I\J\K\L\M\N\O\P\Q\R\S\T\U\V\W\X\Y\Z
%   Lower-case    \a\b\c\d\e\f\g\h\i\j\k\l\m\n\o\p\q\r\s\t\u\v\w\x\y\z
%   Digits        \0\1\2\3\4\5\6\7\8\9
%   Exclamation   \!     Double quote  \"     Hash (number) \#
%   Dollar        \$     Percent       \%     Ampersand     \&
%   Acute accent  \'     Left paren    \(     Right paren   \)
%   Asterisk      \*     Plus          \+     Comma         \,
%   Minus         \-     Point         \.     Solidus       \/
%   Colon         \:     Semicolon     \;     Less than     \<
%   Equals        \=     Greater than  \>     Question mark \?
%   Commercial at \@     Left bracket  \[     Backslash     \\
%   Right bracket \]     Circumflex    \^     Underscore    \_
%   Grave accent  \`     Left brace    \{     Vertical bar  \|
%   Right brace   \}     Tilde         \~}
%
% \GetFileInfo{colonequals.drv}
%
% \title{The \xpackage{colonequals} package}
% \date{2006/08/01 v1.0}
% \author{Heiko Oberdiek\\\xemail{heiko.oberdiek at googlemail.com}}
%
% \maketitle
%
% \begin{abstract}
% Package \xpackage{colonequals} defines poor man's symbols
% for math relation symbols such as ``colon equals''.
% The colon is centered around the horizontal math axis.
% \end{abstract}
%
% \tableofcontents
%
% \section{User interface}
%
% \subsection{Introduction}
%
% Math symbols consisting of the colon character can be
% constructed with the colon text character, if the math font
% lacks of the complete symbol. Often, however, the colon text
% character is not centered around the math axis. Especially
% combined with the equals symbol the composed symbol does not
% look symmetrically. Thus this packages defines a colon
% math symbol \cs{ratio} that is centered around the horizontal
% math axis. Also math symbols are provided that consist of the
% colon symbol. The package is not necessary, if the math
% font contains the composed symbols. Examples are \textsf{txfonts}
% (\cite{txfonts}) or \textsf{mathabx} (\cite{mathabx}).
%
% \subsection{Symbols}
%
% All symbols of this package are relation symbols.
% The relation property can be changed by the appropriate
% \TeX\ command \cs{mathbin}, \cs{mathord}, \dots
%
% \begin{center}
% \captionof{table}{Unicode mathematical operators}
% \kern1ex
% \begin{tabular}{l>{\scshape}l>{$}l<{$}l}
%   U+2236 & ratio        & \ratio       & \cs{ratio}       \\
%   U+2237 & proportion   & \coloncolon  & \cs{coloncolon}  \\
%   U+2239 & excess       & \colonminus  & \cs{colonminus}  \\
%   U+2254 & colon equals & \colonequals & \cs{colonequals} \\
%   U+2255 & equals colon & \equalscolon & \cs{equalscolon} \\
% \end{tabular}
% \end{center}
%
% The following grammar generates all symbols that are supported by
% this package:
% \begin{center}
% \captionof{table}{Symbol grammar}
% \kern1ex
% \begin{tabular}{@{}l>{$}r<{$}l@{}}
%   symbols & \coloncolonequals & col \\
%           & \mid & col symbol \\
%           & \mid & symbol col \\
%           & ; & \\[1ex]
%   col     & \coloncolonequals & '\texttt{:}' \\
%           & \mid & '\texttt{::}' \\
%           & ; & \\[1ex]
%   symbol  & \coloncolonequals & '\texttt{=}' \\
%           & \mid & '\texttt{-}' \\
%           & \mid & '$\approx$' \\
%           & \mid & '$\sim$' \\
%           & ; &
% \end{tabular}
% \end{center}
%
% \def\entry#1{\csname #1\endcsname&\cs{#1}\\}
% \def\entryset#1{^^A
%    \entry{colon#1}^^A
%    \entry{coloncolon#1}^^A
%    \entry{#1colon}^^A
%    \entry{#1coloncolon}^^A
% }
% \begin{longtable}{>{$}l<{$}l}
%   \caption{All relation symbols}\\
%   \entry{ratio}
%   \entry{coloncolon}
%   \entryset{equals}
%   \entryset{minus}
%   \entryset{approx}
%   \entryset{sim}
% \end{longtable}
%
% \subsection{Fine tuning}
%
% The distances in composed symbols can be configured:
%
% \begin{declcs}{colonsep}
% \end{declcs}
% Macro \cs{colonsep} is executed between the colon and
% the other symbol.
%
% \begin{declcs}{doublecolonsep}
% \end{declcs}
% Macro \cs{doublecolonsep} controls the distance between
% two colons.
%
% \subsubsection{Example}
% \begin{quote}
%   \verb|\renewcommand*{\colonsep}{\mskip-.5\thinmuskip}|
% \end{quote}
%
%
% \StopEventually{
% }
%
% \section{Implementation}
%
% \subsection{Identification}
%
%    \begin{macrocode}
%<*package>
\NeedsTeXFormat{LaTeX2e}
\ProvidesPackage{colonequals}%
  [2006/08/01 v1.0 Colon equals symbols (HO)]%
%    \end{macrocode}
%
% \subsection{Distance control}
%
%    \begin{macro}{\colonsep}
%    \begin{macrocode}
\newcommand*{\colonsep}{}
%    \end{macrocode}
%    \end{macro}
%
%    \begin{macro}{\doublecolonsep}
%    \begin{macrocode}
\newcommand*{\doublecolonsep}{}
%    \end{macrocode}
%    \end{macro}
%
% \subsection{Centered colons}
%
%    \begin{macrocode}
\def\@center@colon{%
  \mathpalette\@center@math{:}%
}
\def\@center@math#1#2{%
  \vcenter{%
    \m@th
    \hbox{$#1#2$}%
  }%
}
%    \end{macrocode}
%
%    \begin{macro}{\ratio}
%    Because the name \cs{colon} is already in use, the Unicode name
%    \cs{ratio} is used for the centered colon relation symbol.
%    (The \cs{ratio} of package \textsf{calc} is not used outside
%    calc expressions.)
%    \begin{macrocode}
\newcommand*{\ratio}{%
  \ensuremath{%
    \mathrel{%
      \@center@colon
    }%
  }%
}
%    \end{macrocode}
%    \end{macro}
%
%    \begin{macro}{\coloncolon}
%    \begin{macrocode}
\newcommand*{\coloncolon}{%
  \ensuremath{%
    \mathrel{%
      \@center@colon
      \doublecolonsep
      \@center@colon
    }%
  }%
}
%    \end{macrocode}
%    \end{macro}
%
% \subsection{Combined symbols}
%
%    \begin{macrocode}
\def\@make@colon@set#1#2{%
  \begingroup
    \let\@center@colon\relax
    \let\newcommand\relax
    \let\ensuremath\relax
    \let\mathrel\relax
    \let\colonsep\relax
    \let\doublecolonsep\relax
    \def\csx##1{%
      \expandafter\noexpand\csname ##1\endcsname
    }%
    \edef\x{\endgroup
      \newcommand*{\csx{colon#1}}{%
        \ensuremath{%
          \mathrel{%
            \@center@colon
            \colonsep
            {#2}%
          }%
        }%
      }%
      \newcommand*{\csx{coloncolon#1}}{%
        \ensuremath{%
          \mathrel{%
            \@center@colon
            \doublecolonsep
            \@center@colon
            \colonsep
            {#2}%
          }%
        }%
      }%
      \newcommand*{\csx{#1colon}}{%
        \ensuremath{%
          \mathrel{%
            {#2}%
            \colonsep
            \@center@colon
          }%
        }%
      }%
      \newcommand*{\csx{#1coloncolon}}{%
        \ensuremath{%
          \mathrel{%
            {#2}%
            \colonsep
            \@center@colon
            \doublecolonsep
            \@center@colon
          }%
        }%
      }%
    }%
  \x
}
%    \end{macrocode}
%
%    \begin{macrocode}
\@make@colon@set{equals}{=}%
\@make@colon@set{minus}{-}%
\@make@colon@set{approx}{\approx}
\@make@colon@set{sim}{\sim}
%    \end{macrocode}
%
%    \begin{macrocode}
%</package>
%    \end{macrocode}
%
% \section{Installation}
%
% \subsection{Download}
%
% \paragraph{Package.} This package is available on
% CTAN\footnote{\url{ftp://ftp.ctan.org/tex-archive/}}:
% \begin{description}
% \item[\CTAN{macros/latex/contrib/oberdiek/colonequals.dtx}] The source file.
% \item[\CTAN{macros/latex/contrib/oberdiek/colonequals.pdf}] Documentation.
% \end{description}
%
%
% \paragraph{Bundle.} All the packages of the bundle `oberdiek'
% are also available in a TDS compliant ZIP archive. There
% the packages are already unpacked and the documentation files
% are generated. The files and directories obey the TDS standard.
% \begin{description}
% \item[\CTAN{install/macros/latex/contrib/oberdiek.tds.zip}]
% \end{description}
% \emph{TDS} refers to the standard ``A Directory Structure
% for \TeX\ Files'' (\CTAN{tds/tds.pdf}). Directories
% with \xfile{texmf} in their name are usually organized this way.
%
% \subsection{Bundle installation}
%
% \paragraph{Unpacking.} Unpack the \xfile{oberdiek.tds.zip} in the
% TDS tree (also known as \xfile{texmf} tree) of your choice.
% Example (linux):
% \begin{quote}
%   |unzip oberdiek.tds.zip -d ~/texmf|
% \end{quote}
%
% \paragraph{Script installation.}
% Check the directory \xfile{TDS:scripts/oberdiek/} for
% scripts that need further installation steps.
% Package \xpackage{attachfile2} comes with the Perl script
% \xfile{pdfatfi.pl} that should be installed in such a way
% that it can be called as \texttt{pdfatfi}.
% Example (linux):
% \begin{quote}
%   |chmod +x scripts/oberdiek/pdfatfi.pl|\\
%   |cp scripts/oberdiek/pdfatfi.pl /usr/local/bin/|
% \end{quote}
%
% \subsection{Package installation}
%
% \paragraph{Unpacking.} The \xfile{.dtx} file is a self-extracting
% \docstrip\ archive. The files are extracted by running the
% \xfile{.dtx} through \plainTeX:
% \begin{quote}
%   \verb|tex colonequals.dtx|
% \end{quote}
%
% \paragraph{TDS.} Now the different files must be moved into
% the different directories in your installation TDS tree
% (also known as \xfile{texmf} tree):
% \begin{quote}
% \def\t{^^A
% \begin{tabular}{@{}>{\ttfamily}l@{ $\rightarrow$ }>{\ttfamily}l@{}}
%   colonequals.sty & tex/latex/oberdiek/colonequals.sty\\
%   colonequals.pdf & doc/latex/oberdiek/colonequals.pdf\\
%   colonequals.dtx & source/latex/oberdiek/colonequals.dtx\\
% \end{tabular}^^A
% }^^A
% \sbox0{\t}^^A
% \ifdim\wd0>\linewidth
%   \begingroup
%     \advance\linewidth by\leftmargin
%     \advance\linewidth by\rightmargin
%   \edef\x{\endgroup
%     \def\noexpand\lw{\the\linewidth}^^A
%   }\x
%   \def\lwbox{^^A
%     \leavevmode
%     \hbox to \linewidth{^^A
%       \kern-\leftmargin\relax
%       \hss
%       \usebox0
%       \hss
%       \kern-\rightmargin\relax
%     }^^A
%   }^^A
%   \ifdim\wd0>\lw
%     \sbox0{\small\t}^^A
%     \ifdim\wd0>\linewidth
%       \ifdim\wd0>\lw
%         \sbox0{\footnotesize\t}^^A
%         \ifdim\wd0>\linewidth
%           \ifdim\wd0>\lw
%             \sbox0{\scriptsize\t}^^A
%             \ifdim\wd0>\linewidth
%               \ifdim\wd0>\lw
%                 \sbox0{\tiny\t}^^A
%                 \ifdim\wd0>\linewidth
%                   \lwbox
%                 \else
%                   \usebox0
%                 \fi
%               \else
%                 \lwbox
%               \fi
%             \else
%               \usebox0
%             \fi
%           \else
%             \lwbox
%           \fi
%         \else
%           \usebox0
%         \fi
%       \else
%         \lwbox
%       \fi
%     \else
%       \usebox0
%     \fi
%   \else
%     \lwbox
%   \fi
% \else
%   \usebox0
% \fi
% \end{quote}
% If you have a \xfile{docstrip.cfg} that configures and enables \docstrip's
% TDS installing feature, then some files can already be in the right
% place, see the documentation of \docstrip.
%
% \subsection{Refresh file name databases}
%
% If your \TeX~distribution
% (\teTeX, \mikTeX, \dots) relies on file name databases, you must refresh
% these. For example, \teTeX\ users run \verb|texhash| or
% \verb|mktexlsr|.
%
% \subsection{Some details for the interested}
%
% \paragraph{Attached source.}
%
% The PDF documentation on CTAN also includes the
% \xfile{.dtx} source file. It can be extracted by
% AcrobatReader 6 or higher. Another option is \textsf{pdftk},
% e.g. unpack the file into the current directory:
% \begin{quote}
%   \verb|pdftk colonequals.pdf unpack_files output .|
% \end{quote}
%
% \paragraph{Unpacking with \LaTeX.}
% The \xfile{.dtx} chooses its action depending on the format:
% \begin{description}
% \item[\plainTeX:] Run \docstrip\ and extract the files.
% \item[\LaTeX:] Generate the documentation.
% \end{description}
% If you insist on using \LaTeX\ for \docstrip\ (really,
% \docstrip\ does not need \LaTeX), then inform the autodetect routine
% about your intention:
% \begin{quote}
%   \verb|latex \let\install=y% \iffalse meta-comment
%
% File: colonequals.dtx
% Version: 2006/08/01 v1.0
% Info: Colon equals symbols
%
% Copyright (C) 2006 by
%    Heiko Oberdiek <heiko.oberdiek at googlemail.com>
%
% This work may be distributed and/or modified under the
% conditions of the LaTeX Project Public License, either
% version 1.3c of this license or (at your option) any later
% version. This version of this license is in
%    http://www.latex-project.org/lppl/lppl-1-3c.txt
% and the latest version of this license is in
%    http://www.latex-project.org/lppl.txt
% and version 1.3 or later is part of all distributions of
% LaTeX version 2005/12/01 or later.
%
% This work has the LPPL maintenance status "maintained".
%
% This Current Maintainer of this work is Heiko Oberdiek.
%
% This work consists of the main source file colonequals.dtx
% and the derived files
%    colonequals.sty, colonequals.pdf, colonequals.ins, colonequals.drv.
%
% Distribution:
%    CTAN:macros/latex/contrib/oberdiek/colonequals.dtx
%    CTAN:macros/latex/contrib/oberdiek/colonequals.pdf
%
% Unpacking:
%    (a) If colonequals.ins is present:
%           tex colonequals.ins
%    (b) Without colonequals.ins:
%           tex colonequals.dtx
%    (c) If you insist on using LaTeX
%           latex \let\install=y% \iffalse meta-comment
%
% File: colonequals.dtx
% Version: 2006/08/01 v1.0
% Info: Colon equals symbols
%
% Copyright (C) 2006 by
%    Heiko Oberdiek <heiko.oberdiek at googlemail.com>
%
% This work may be distributed and/or modified under the
% conditions of the LaTeX Project Public License, either
% version 1.3c of this license or (at your option) any later
% version. This version of this license is in
%    http://www.latex-project.org/lppl/lppl-1-3c.txt
% and the latest version of this license is in
%    http://www.latex-project.org/lppl.txt
% and version 1.3 or later is part of all distributions of
% LaTeX version 2005/12/01 or later.
%
% This work has the LPPL maintenance status "maintained".
%
% This Current Maintainer of this work is Heiko Oberdiek.
%
% This work consists of the main source file colonequals.dtx
% and the derived files
%    colonequals.sty, colonequals.pdf, colonequals.ins, colonequals.drv.
%
% Distribution:
%    CTAN:macros/latex/contrib/oberdiek/colonequals.dtx
%    CTAN:macros/latex/contrib/oberdiek/colonequals.pdf
%
% Unpacking:
%    (a) If colonequals.ins is present:
%           tex colonequals.ins
%    (b) Without colonequals.ins:
%           tex colonequals.dtx
%    (c) If you insist on using LaTeX
%           latex \let\install=y\input{colonequals.dtx}
%        (quote the arguments according to the demands of your shell)
%
% Documentation:
%    (a) If colonequals.drv is present:
%           latex colonequals.drv
%    (b) Without colonequals.drv:
%           latex colonequals.dtx; ...
%    The class ltxdoc loads the configuration file ltxdoc.cfg
%    if available. Here you can specify further options, e.g.
%    use A4 as paper format:
%       \PassOptionsToClass{a4paper}{article}
%
%    Programm calls to get the documentation (example):
%       pdflatex colonequals.dtx
%       makeindex -s gind.ist colonequals.idx
%       pdflatex colonequals.dtx
%       makeindex -s gind.ist colonequals.idx
%       pdflatex colonequals.dtx
%
% Installation:
%    TDS:tex/latex/oberdiek/colonequals.sty
%    TDS:doc/latex/oberdiek/colonequals.pdf
%    TDS:source/latex/oberdiek/colonequals.dtx
%
%<*ignore>
\begingroup
  \catcode123=1 %
  \catcode125=2 %
  \def\x{LaTeX2e}%
\expandafter\endgroup
\ifcase 0\ifx\install y1\fi\expandafter
         \ifx\csname processbatchFile\endcsname\relax\else1\fi
         \ifx\fmtname\x\else 1\fi\relax
\else\csname fi\endcsname
%</ignore>
%<*install>
\input docstrip.tex
\Msg{************************************************************************}
\Msg{* Installation}
\Msg{* Package: colonequals 2006/08/01 v1.0 Colon equals symbols (HO)}
\Msg{************************************************************************}

\keepsilent
\askforoverwritefalse

\let\MetaPrefix\relax
\preamble

This is a generated file.

Project: colonequals
Version: 2006/08/01 v1.0

Copyright (C) 2006 by
   Heiko Oberdiek <heiko.oberdiek at googlemail.com>

This work may be distributed and/or modified under the
conditions of the LaTeX Project Public License, either
version 1.3c of this license or (at your option) any later
version. This version of this license is in
   http://www.latex-project.org/lppl/lppl-1-3c.txt
and the latest version of this license is in
   http://www.latex-project.org/lppl.txt
and version 1.3 or later is part of all distributions of
LaTeX version 2005/12/01 or later.

This work has the LPPL maintenance status "maintained".

This Current Maintainer of this work is Heiko Oberdiek.

This work consists of the main source file colonequals.dtx
and the derived files
   colonequals.sty, colonequals.pdf, colonequals.ins, colonequals.drv.

\endpreamble
\let\MetaPrefix\DoubleperCent

\generate{%
  \file{colonequals.ins}{\from{colonequals.dtx}{install}}%
  \file{colonequals.drv}{\from{colonequals.dtx}{driver}}%
  \usedir{tex/latex/oberdiek}%
  \file{colonequals.sty}{\from{colonequals.dtx}{package}}%
  \nopreamble
  \nopostamble
  \usedir{source/latex/oberdiek/catalogue}%
  \file{colonequals.xml}{\from{colonequals.dtx}{catalogue}}%
}

\catcode32=13\relax% active space
\let =\space%
\Msg{************************************************************************}
\Msg{*}
\Msg{* To finish the installation you have to move the following}
\Msg{* file into a directory searched by TeX:}
\Msg{*}
\Msg{*     colonequals.sty}
\Msg{*}
\Msg{* To produce the documentation run the file `colonequals.drv'}
\Msg{* through LaTeX.}
\Msg{*}
\Msg{* Happy TeXing!}
\Msg{*}
\Msg{************************************************************************}

\endbatchfile
%</install>
%<*ignore>
\fi
%</ignore>
%<*driver>
\NeedsTeXFormat{LaTeX2e}
\ProvidesFile{colonequals.drv}%
  [2006/08/01 v1.0 Colon equals symbols (HO)]%
\documentclass{ltxdoc}
\usepackage{holtxdoc}[2011/11/22]
\usepackage{colonequals}
\usepackage{array}
\usepackage{capt-of}
\usepackage{longtable}
\begin{document}
  \DocInput{colonequals.dtx}%
\end{document}
%</driver>
% \fi
%
% \CheckSum{92}
%
% \CharacterTable
%  {Upper-case    \A\B\C\D\E\F\G\H\I\J\K\L\M\N\O\P\Q\R\S\T\U\V\W\X\Y\Z
%   Lower-case    \a\b\c\d\e\f\g\h\i\j\k\l\m\n\o\p\q\r\s\t\u\v\w\x\y\z
%   Digits        \0\1\2\3\4\5\6\7\8\9
%   Exclamation   \!     Double quote  \"     Hash (number) \#
%   Dollar        \$     Percent       \%     Ampersand     \&
%   Acute accent  \'     Left paren    \(     Right paren   \)
%   Asterisk      \*     Plus          \+     Comma         \,
%   Minus         \-     Point         \.     Solidus       \/
%   Colon         \:     Semicolon     \;     Less than     \<
%   Equals        \=     Greater than  \>     Question mark \?
%   Commercial at \@     Left bracket  \[     Backslash     \\
%   Right bracket \]     Circumflex    \^     Underscore    \_
%   Grave accent  \`     Left brace    \{     Vertical bar  \|
%   Right brace   \}     Tilde         \~}
%
% \GetFileInfo{colonequals.drv}
%
% \title{The \xpackage{colonequals} package}
% \date{2006/08/01 v1.0}
% \author{Heiko Oberdiek\\\xemail{heiko.oberdiek at googlemail.com}}
%
% \maketitle
%
% \begin{abstract}
% Package \xpackage{colonequals} defines poor man's symbols
% for math relation symbols such as ``colon equals''.
% The colon is centered around the horizontal math axis.
% \end{abstract}
%
% \tableofcontents
%
% \section{User interface}
%
% \subsection{Introduction}
%
% Math symbols consisting of the colon character can be
% constructed with the colon text character, if the math font
% lacks of the complete symbol. Often, however, the colon text
% character is not centered around the math axis. Especially
% combined with the equals symbol the composed symbol does not
% look symmetrically. Thus this packages defines a colon
% math symbol \cs{ratio} that is centered around the horizontal
% math axis. Also math symbols are provided that consist of the
% colon symbol. The package is not necessary, if the math
% font contains the composed symbols. Examples are \textsf{txfonts}
% (\cite{txfonts}) or \textsf{mathabx} (\cite{mathabx}).
%
% \subsection{Symbols}
%
% All symbols of this package are relation symbols.
% The relation property can be changed by the appropriate
% \TeX\ command \cs{mathbin}, \cs{mathord}, \dots
%
% \begin{center}
% \captionof{table}{Unicode mathematical operators}
% \kern1ex
% \begin{tabular}{l>{\scshape}l>{$}l<{$}l}
%   U+2236 & ratio        & \ratio       & \cs{ratio}       \\
%   U+2237 & proportion   & \coloncolon  & \cs{coloncolon}  \\
%   U+2239 & excess       & \colonminus  & \cs{colonminus}  \\
%   U+2254 & colon equals & \colonequals & \cs{colonequals} \\
%   U+2255 & equals colon & \equalscolon & \cs{equalscolon} \\
% \end{tabular}
% \end{center}
%
% The following grammar generates all symbols that are supported by
% this package:
% \begin{center}
% \captionof{table}{Symbol grammar}
% \kern1ex
% \begin{tabular}{@{}l>{$}r<{$}l@{}}
%   symbols & \coloncolonequals & col \\
%           & \mid & col symbol \\
%           & \mid & symbol col \\
%           & ; & \\[1ex]
%   col     & \coloncolonequals & '\texttt{:}' \\
%           & \mid & '\texttt{::}' \\
%           & ; & \\[1ex]
%   symbol  & \coloncolonequals & '\texttt{=}' \\
%           & \mid & '\texttt{-}' \\
%           & \mid & '$\approx$' \\
%           & \mid & '$\sim$' \\
%           & ; &
% \end{tabular}
% \end{center}
%
% \def\entry#1{\csname #1\endcsname&\cs{#1}\\}
% \def\entryset#1{^^A
%    \entry{colon#1}^^A
%    \entry{coloncolon#1}^^A
%    \entry{#1colon}^^A
%    \entry{#1coloncolon}^^A
% }
% \begin{longtable}{>{$}l<{$}l}
%   \caption{All relation symbols}\\
%   \entry{ratio}
%   \entry{coloncolon}
%   \entryset{equals}
%   \entryset{minus}
%   \entryset{approx}
%   \entryset{sim}
% \end{longtable}
%
% \subsection{Fine tuning}
%
% The distances in composed symbols can be configured:
%
% \begin{declcs}{colonsep}
% \end{declcs}
% Macro \cs{colonsep} is executed between the colon and
% the other symbol.
%
% \begin{declcs}{doublecolonsep}
% \end{declcs}
% Macro \cs{doublecolonsep} controls the distance between
% two colons.
%
% \subsubsection{Example}
% \begin{quote}
%   \verb|\renewcommand*{\colonsep}{\mskip-.5\thinmuskip}|
% \end{quote}
%
%
% \StopEventually{
% }
%
% \section{Implementation}
%
% \subsection{Identification}
%
%    \begin{macrocode}
%<*package>
\NeedsTeXFormat{LaTeX2e}
\ProvidesPackage{colonequals}%
  [2006/08/01 v1.0 Colon equals symbols (HO)]%
%    \end{macrocode}
%
% \subsection{Distance control}
%
%    \begin{macro}{\colonsep}
%    \begin{macrocode}
\newcommand*{\colonsep}{}
%    \end{macrocode}
%    \end{macro}
%
%    \begin{macro}{\doublecolonsep}
%    \begin{macrocode}
\newcommand*{\doublecolonsep}{}
%    \end{macrocode}
%    \end{macro}
%
% \subsection{Centered colons}
%
%    \begin{macrocode}
\def\@center@colon{%
  \mathpalette\@center@math{:}%
}
\def\@center@math#1#2{%
  \vcenter{%
    \m@th
    \hbox{$#1#2$}%
  }%
}
%    \end{macrocode}
%
%    \begin{macro}{\ratio}
%    Because the name \cs{colon} is already in use, the Unicode name
%    \cs{ratio} is used for the centered colon relation symbol.
%    (The \cs{ratio} of package \textsf{calc} is not used outside
%    calc expressions.)
%    \begin{macrocode}
\newcommand*{\ratio}{%
  \ensuremath{%
    \mathrel{%
      \@center@colon
    }%
  }%
}
%    \end{macrocode}
%    \end{macro}
%
%    \begin{macro}{\coloncolon}
%    \begin{macrocode}
\newcommand*{\coloncolon}{%
  \ensuremath{%
    \mathrel{%
      \@center@colon
      \doublecolonsep
      \@center@colon
    }%
  }%
}
%    \end{macrocode}
%    \end{macro}
%
% \subsection{Combined symbols}
%
%    \begin{macrocode}
\def\@make@colon@set#1#2{%
  \begingroup
    \let\@center@colon\relax
    \let\newcommand\relax
    \let\ensuremath\relax
    \let\mathrel\relax
    \let\colonsep\relax
    \let\doublecolonsep\relax
    \def\csx##1{%
      \expandafter\noexpand\csname ##1\endcsname
    }%
    \edef\x{\endgroup
      \newcommand*{\csx{colon#1}}{%
        \ensuremath{%
          \mathrel{%
            \@center@colon
            \colonsep
            {#2}%
          }%
        }%
      }%
      \newcommand*{\csx{coloncolon#1}}{%
        \ensuremath{%
          \mathrel{%
            \@center@colon
            \doublecolonsep
            \@center@colon
            \colonsep
            {#2}%
          }%
        }%
      }%
      \newcommand*{\csx{#1colon}}{%
        \ensuremath{%
          \mathrel{%
            {#2}%
            \colonsep
            \@center@colon
          }%
        }%
      }%
      \newcommand*{\csx{#1coloncolon}}{%
        \ensuremath{%
          \mathrel{%
            {#2}%
            \colonsep
            \@center@colon
            \doublecolonsep
            \@center@colon
          }%
        }%
      }%
    }%
  \x
}
%    \end{macrocode}
%
%    \begin{macrocode}
\@make@colon@set{equals}{=}%
\@make@colon@set{minus}{-}%
\@make@colon@set{approx}{\approx}
\@make@colon@set{sim}{\sim}
%    \end{macrocode}
%
%    \begin{macrocode}
%</package>
%    \end{macrocode}
%
% \section{Installation}
%
% \subsection{Download}
%
% \paragraph{Package.} This package is available on
% CTAN\footnote{\url{ftp://ftp.ctan.org/tex-archive/}}:
% \begin{description}
% \item[\CTAN{macros/latex/contrib/oberdiek/colonequals.dtx}] The source file.
% \item[\CTAN{macros/latex/contrib/oberdiek/colonequals.pdf}] Documentation.
% \end{description}
%
%
% \paragraph{Bundle.} All the packages of the bundle `oberdiek'
% are also available in a TDS compliant ZIP archive. There
% the packages are already unpacked and the documentation files
% are generated. The files and directories obey the TDS standard.
% \begin{description}
% \item[\CTAN{install/macros/latex/contrib/oberdiek.tds.zip}]
% \end{description}
% \emph{TDS} refers to the standard ``A Directory Structure
% for \TeX\ Files'' (\CTAN{tds/tds.pdf}). Directories
% with \xfile{texmf} in their name are usually organized this way.
%
% \subsection{Bundle installation}
%
% \paragraph{Unpacking.} Unpack the \xfile{oberdiek.tds.zip} in the
% TDS tree (also known as \xfile{texmf} tree) of your choice.
% Example (linux):
% \begin{quote}
%   |unzip oberdiek.tds.zip -d ~/texmf|
% \end{quote}
%
% \paragraph{Script installation.}
% Check the directory \xfile{TDS:scripts/oberdiek/} for
% scripts that need further installation steps.
% Package \xpackage{attachfile2} comes with the Perl script
% \xfile{pdfatfi.pl} that should be installed in such a way
% that it can be called as \texttt{pdfatfi}.
% Example (linux):
% \begin{quote}
%   |chmod +x scripts/oberdiek/pdfatfi.pl|\\
%   |cp scripts/oberdiek/pdfatfi.pl /usr/local/bin/|
% \end{quote}
%
% \subsection{Package installation}
%
% \paragraph{Unpacking.} The \xfile{.dtx} file is a self-extracting
% \docstrip\ archive. The files are extracted by running the
% \xfile{.dtx} through \plainTeX:
% \begin{quote}
%   \verb|tex colonequals.dtx|
% \end{quote}
%
% \paragraph{TDS.} Now the different files must be moved into
% the different directories in your installation TDS tree
% (also known as \xfile{texmf} tree):
% \begin{quote}
% \def\t{^^A
% \begin{tabular}{@{}>{\ttfamily}l@{ $\rightarrow$ }>{\ttfamily}l@{}}
%   colonequals.sty & tex/latex/oberdiek/colonequals.sty\\
%   colonequals.pdf & doc/latex/oberdiek/colonequals.pdf\\
%   colonequals.dtx & source/latex/oberdiek/colonequals.dtx\\
% \end{tabular}^^A
% }^^A
% \sbox0{\t}^^A
% \ifdim\wd0>\linewidth
%   \begingroup
%     \advance\linewidth by\leftmargin
%     \advance\linewidth by\rightmargin
%   \edef\x{\endgroup
%     \def\noexpand\lw{\the\linewidth}^^A
%   }\x
%   \def\lwbox{^^A
%     \leavevmode
%     \hbox to \linewidth{^^A
%       \kern-\leftmargin\relax
%       \hss
%       \usebox0
%       \hss
%       \kern-\rightmargin\relax
%     }^^A
%   }^^A
%   \ifdim\wd0>\lw
%     \sbox0{\small\t}^^A
%     \ifdim\wd0>\linewidth
%       \ifdim\wd0>\lw
%         \sbox0{\footnotesize\t}^^A
%         \ifdim\wd0>\linewidth
%           \ifdim\wd0>\lw
%             \sbox0{\scriptsize\t}^^A
%             \ifdim\wd0>\linewidth
%               \ifdim\wd0>\lw
%                 \sbox0{\tiny\t}^^A
%                 \ifdim\wd0>\linewidth
%                   \lwbox
%                 \else
%                   \usebox0
%                 \fi
%               \else
%                 \lwbox
%               \fi
%             \else
%               \usebox0
%             \fi
%           \else
%             \lwbox
%           \fi
%         \else
%           \usebox0
%         \fi
%       \else
%         \lwbox
%       \fi
%     \else
%       \usebox0
%     \fi
%   \else
%     \lwbox
%   \fi
% \else
%   \usebox0
% \fi
% \end{quote}
% If you have a \xfile{docstrip.cfg} that configures and enables \docstrip's
% TDS installing feature, then some files can already be in the right
% place, see the documentation of \docstrip.
%
% \subsection{Refresh file name databases}
%
% If your \TeX~distribution
% (\teTeX, \mikTeX, \dots) relies on file name databases, you must refresh
% these. For example, \teTeX\ users run \verb|texhash| or
% \verb|mktexlsr|.
%
% \subsection{Some details for the interested}
%
% \paragraph{Attached source.}
%
% The PDF documentation on CTAN also includes the
% \xfile{.dtx} source file. It can be extracted by
% AcrobatReader 6 or higher. Another option is \textsf{pdftk},
% e.g. unpack the file into the current directory:
% \begin{quote}
%   \verb|pdftk colonequals.pdf unpack_files output .|
% \end{quote}
%
% \paragraph{Unpacking with \LaTeX.}
% The \xfile{.dtx} chooses its action depending on the format:
% \begin{description}
% \item[\plainTeX:] Run \docstrip\ and extract the files.
% \item[\LaTeX:] Generate the documentation.
% \end{description}
% If you insist on using \LaTeX\ for \docstrip\ (really,
% \docstrip\ does not need \LaTeX), then inform the autodetect routine
% about your intention:
% \begin{quote}
%   \verb|latex \let\install=y\input{colonequals.dtx}|
% \end{quote}
% Do not forget to quote the argument according to the demands
% of your shell.
%
% \paragraph{Generating the documentation.}
% You can use both the \xfile{.dtx} or the \xfile{.drv} to generate
% the documentation. The process can be configured by the
% configuration file \xfile{ltxdoc.cfg}. For instance, put this
% line into this file, if you want to have A4 as paper format:
% \begin{quote}
%   \verb|\PassOptionsToClass{a4paper}{article}|
% \end{quote}
% An example follows how to generate the
% documentation with pdf\LaTeX:
% \begin{quote}
%\begin{verbatim}
%pdflatex colonequals.dtx
%makeindex -s gind.ist colonequals.idx
%pdflatex colonequals.dtx
%makeindex -s gind.ist colonequals.idx
%pdflatex colonequals.dtx
%\end{verbatim}
% \end{quote}
%
% \section{Catalogue}
%
% The following XML file can be used as source for the
% \href{http://mirror.ctan.org/help/Catalogue/catalogue.html}{\TeX\ Catalogue}.
% The elements \texttt{caption} and \texttt{description} are imported
% from the original XML file from the Catalogue.
% The name of the XML file in the Catalogue is \xfile{colonequals.xml}.
%    \begin{macrocode}
%<*catalogue>
<?xml version='1.0' encoding='us-ascii'?>
<!DOCTYPE entry SYSTEM 'catalogue.dtd'>
<entry datestamp='$Date$' modifier='$Author$' id='colonequals'>
  <name>colonequals</name>
  <caption>Colon equals symbols.</caption>
  <authorref id='auth:oberdiek'/>
  <copyright owner='Heiko Oberdiek' year='2006'/>
  <license type='lppl1.3'/>
  <version number='1.0'/>
  <description>
    This package defines poor man&#x2018;s symbols for mathematical
    relation symbols such as &#x201C;colon equals&#x201D;.
    The colon is centered around the horizontal math axis.
    <p/>
    The package is part of the <xref refid='oberdiek'>oberdiek</xref>
    bundle.
  </description>
  <documentation details='Package documentation'
      href='ctan:/macros/latex/contrib/oberdiek/colonequals.pdf'/>
  <ctan file='true' path='/macros/latex/contrib/oberdiek/colonequals.dtx'/>
  <miktex location='oberdiek'/>
  <texlive location='oberdiek'/>
  <install path='/macros/latex/contrib/oberdiek/oberdiek.tds.zip'/>
</entry>
%</catalogue>
%    \end{macrocode}
%
% \begin{thebibliography}{9}
%
% \bibitem{txfonts}
%   Young Ryu: \textit{The TX Fonts};
%   2000/12/15;
%   \CTAN{fonts/txfonts/}.
%
% \bibitem{mathabx}
%   Anthony Phan: \textit{Mathabx font series};
%   2005/05/16;
%   \CTAN{fonts/mathabx/}.
%
% \end{thebibliography}
%
% \begin{History}
%   \begin{Version}{2006/08/01 v1.0}
%   \item
%     First version.
%   \end{Version}
% \end{History}
%
% \PrintIndex
%
% \Finale
\endinput

%        (quote the arguments according to the demands of your shell)
%
% Documentation:
%    (a) If colonequals.drv is present:
%           latex colonequals.drv
%    (b) Without colonequals.drv:
%           latex colonequals.dtx; ...
%    The class ltxdoc loads the configuration file ltxdoc.cfg
%    if available. Here you can specify further options, e.g.
%    use A4 as paper format:
%       \PassOptionsToClass{a4paper}{article}
%
%    Programm calls to get the documentation (example):
%       pdflatex colonequals.dtx
%       makeindex -s gind.ist colonequals.idx
%       pdflatex colonequals.dtx
%       makeindex -s gind.ist colonequals.idx
%       pdflatex colonequals.dtx
%
% Installation:
%    TDS:tex/latex/oberdiek/colonequals.sty
%    TDS:doc/latex/oberdiek/colonequals.pdf
%    TDS:source/latex/oberdiek/colonequals.dtx
%
%<*ignore>
\begingroup
  \catcode123=1 %
  \catcode125=2 %
  \def\x{LaTeX2e}%
\expandafter\endgroup
\ifcase 0\ifx\install y1\fi\expandafter
         \ifx\csname processbatchFile\endcsname\relax\else1\fi
         \ifx\fmtname\x\else 1\fi\relax
\else\csname fi\endcsname
%</ignore>
%<*install>
\input docstrip.tex
\Msg{************************************************************************}
\Msg{* Installation}
\Msg{* Package: colonequals 2006/08/01 v1.0 Colon equals symbols (HO)}
\Msg{************************************************************************}

\keepsilent
\askforoverwritefalse

\let\MetaPrefix\relax
\preamble

This is a generated file.

Project: colonequals
Version: 2006/08/01 v1.0

Copyright (C) 2006 by
   Heiko Oberdiek <heiko.oberdiek at googlemail.com>

This work may be distributed and/or modified under the
conditions of the LaTeX Project Public License, either
version 1.3c of this license or (at your option) any later
version. This version of this license is in
   http://www.latex-project.org/lppl/lppl-1-3c.txt
and the latest version of this license is in
   http://www.latex-project.org/lppl.txt
and version 1.3 or later is part of all distributions of
LaTeX version 2005/12/01 or later.

This work has the LPPL maintenance status "maintained".

This Current Maintainer of this work is Heiko Oberdiek.

This work consists of the main source file colonequals.dtx
and the derived files
   colonequals.sty, colonequals.pdf, colonequals.ins, colonequals.drv.

\endpreamble
\let\MetaPrefix\DoubleperCent

\generate{%
  \file{colonequals.ins}{\from{colonequals.dtx}{install}}%
  \file{colonequals.drv}{\from{colonequals.dtx}{driver}}%
  \usedir{tex/latex/oberdiek}%
  \file{colonequals.sty}{\from{colonequals.dtx}{package}}%
  \nopreamble
  \nopostamble
  \usedir{source/latex/oberdiek/catalogue}%
  \file{colonequals.xml}{\from{colonequals.dtx}{catalogue}}%
}

\catcode32=13\relax% active space
\let =\space%
\Msg{************************************************************************}
\Msg{*}
\Msg{* To finish the installation you have to move the following}
\Msg{* file into a directory searched by TeX:}
\Msg{*}
\Msg{*     colonequals.sty}
\Msg{*}
\Msg{* To produce the documentation run the file `colonequals.drv'}
\Msg{* through LaTeX.}
\Msg{*}
\Msg{* Happy TeXing!}
\Msg{*}
\Msg{************************************************************************}

\endbatchfile
%</install>
%<*ignore>
\fi
%</ignore>
%<*driver>
\NeedsTeXFormat{LaTeX2e}
\ProvidesFile{colonequals.drv}%
  [2006/08/01 v1.0 Colon equals symbols (HO)]%
\documentclass{ltxdoc}
\usepackage{holtxdoc}[2011/11/22]
\usepackage{colonequals}
\usepackage{array}
\usepackage{capt-of}
\usepackage{longtable}
\begin{document}
  \DocInput{colonequals.dtx}%
\end{document}
%</driver>
% \fi
%
% \CheckSum{92}
%
% \CharacterTable
%  {Upper-case    \A\B\C\D\E\F\G\H\I\J\K\L\M\N\O\P\Q\R\S\T\U\V\W\X\Y\Z
%   Lower-case    \a\b\c\d\e\f\g\h\i\j\k\l\m\n\o\p\q\r\s\t\u\v\w\x\y\z
%   Digits        \0\1\2\3\4\5\6\7\8\9
%   Exclamation   \!     Double quote  \"     Hash (number) \#
%   Dollar        \$     Percent       \%     Ampersand     \&
%   Acute accent  \'     Left paren    \(     Right paren   \)
%   Asterisk      \*     Plus          \+     Comma         \,
%   Minus         \-     Point         \.     Solidus       \/
%   Colon         \:     Semicolon     \;     Less than     \<
%   Equals        \=     Greater than  \>     Question mark \?
%   Commercial at \@     Left bracket  \[     Backslash     \\
%   Right bracket \]     Circumflex    \^     Underscore    \_
%   Grave accent  \`     Left brace    \{     Vertical bar  \|
%   Right brace   \}     Tilde         \~}
%
% \GetFileInfo{colonequals.drv}
%
% \title{The \xpackage{colonequals} package}
% \date{2006/08/01 v1.0}
% \author{Heiko Oberdiek\\\xemail{heiko.oberdiek at googlemail.com}}
%
% \maketitle
%
% \begin{abstract}
% Package \xpackage{colonequals} defines poor man's symbols
% for math relation symbols such as ``colon equals''.
% The colon is centered around the horizontal math axis.
% \end{abstract}
%
% \tableofcontents
%
% \section{User interface}
%
% \subsection{Introduction}
%
% Math symbols consisting of the colon character can be
% constructed with the colon text character, if the math font
% lacks of the complete symbol. Often, however, the colon text
% character is not centered around the math axis. Especially
% combined with the equals symbol the composed symbol does not
% look symmetrically. Thus this packages defines a colon
% math symbol \cs{ratio} that is centered around the horizontal
% math axis. Also math symbols are provided that consist of the
% colon symbol. The package is not necessary, if the math
% font contains the composed symbols. Examples are \textsf{txfonts}
% (\cite{txfonts}) or \textsf{mathabx} (\cite{mathabx}).
%
% \subsection{Symbols}
%
% All symbols of this package are relation symbols.
% The relation property can be changed by the appropriate
% \TeX\ command \cs{mathbin}, \cs{mathord}, \dots
%
% \begin{center}
% \captionof{table}{Unicode mathematical operators}
% \kern1ex
% \begin{tabular}{l>{\scshape}l>{$}l<{$}l}
%   U+2236 & ratio        & \ratio       & \cs{ratio}       \\
%   U+2237 & proportion   & \coloncolon  & \cs{coloncolon}  \\
%   U+2239 & excess       & \colonminus  & \cs{colonminus}  \\
%   U+2254 & colon equals & \colonequals & \cs{colonequals} \\
%   U+2255 & equals colon & \equalscolon & \cs{equalscolon} \\
% \end{tabular}
% \end{center}
%
% The following grammar generates all symbols that are supported by
% this package:
% \begin{center}
% \captionof{table}{Symbol grammar}
% \kern1ex
% \begin{tabular}{@{}l>{$}r<{$}l@{}}
%   symbols & \coloncolonequals & col \\
%           & \mid & col symbol \\
%           & \mid & symbol col \\
%           & ; & \\[1ex]
%   col     & \coloncolonequals & '\texttt{:}' \\
%           & \mid & '\texttt{::}' \\
%           & ; & \\[1ex]
%   symbol  & \coloncolonequals & '\texttt{=}' \\
%           & \mid & '\texttt{-}' \\
%           & \mid & '$\approx$' \\
%           & \mid & '$\sim$' \\
%           & ; &
% \end{tabular}
% \end{center}
%
% \def\entry#1{\csname #1\endcsname&\cs{#1}\\}
% \def\entryset#1{^^A
%    \entry{colon#1}^^A
%    \entry{coloncolon#1}^^A
%    \entry{#1colon}^^A
%    \entry{#1coloncolon}^^A
% }
% \begin{longtable}{>{$}l<{$}l}
%   \caption{All relation symbols}\\
%   \entry{ratio}
%   \entry{coloncolon}
%   \entryset{equals}
%   \entryset{minus}
%   \entryset{approx}
%   \entryset{sim}
% \end{longtable}
%
% \subsection{Fine tuning}
%
% The distances in composed symbols can be configured:
%
% \begin{declcs}{colonsep}
% \end{declcs}
% Macro \cs{colonsep} is executed between the colon and
% the other symbol.
%
% \begin{declcs}{doublecolonsep}
% \end{declcs}
% Macro \cs{doublecolonsep} controls the distance between
% two colons.
%
% \subsubsection{Example}
% \begin{quote}
%   \verb|\renewcommand*{\colonsep}{\mskip-.5\thinmuskip}|
% \end{quote}
%
%
% \StopEventually{
% }
%
% \section{Implementation}
%
% \subsection{Identification}
%
%    \begin{macrocode}
%<*package>
\NeedsTeXFormat{LaTeX2e}
\ProvidesPackage{colonequals}%
  [2006/08/01 v1.0 Colon equals symbols (HO)]%
%    \end{macrocode}
%
% \subsection{Distance control}
%
%    \begin{macro}{\colonsep}
%    \begin{macrocode}
\newcommand*{\colonsep}{}
%    \end{macrocode}
%    \end{macro}
%
%    \begin{macro}{\doublecolonsep}
%    \begin{macrocode}
\newcommand*{\doublecolonsep}{}
%    \end{macrocode}
%    \end{macro}
%
% \subsection{Centered colons}
%
%    \begin{macrocode}
\def\@center@colon{%
  \mathpalette\@center@math{:}%
}
\def\@center@math#1#2{%
  \vcenter{%
    \m@th
    \hbox{$#1#2$}%
  }%
}
%    \end{macrocode}
%
%    \begin{macro}{\ratio}
%    Because the name \cs{colon} is already in use, the Unicode name
%    \cs{ratio} is used for the centered colon relation symbol.
%    (The \cs{ratio} of package \textsf{calc} is not used outside
%    calc expressions.)
%    \begin{macrocode}
\newcommand*{\ratio}{%
  \ensuremath{%
    \mathrel{%
      \@center@colon
    }%
  }%
}
%    \end{macrocode}
%    \end{macro}
%
%    \begin{macro}{\coloncolon}
%    \begin{macrocode}
\newcommand*{\coloncolon}{%
  \ensuremath{%
    \mathrel{%
      \@center@colon
      \doublecolonsep
      \@center@colon
    }%
  }%
}
%    \end{macrocode}
%    \end{macro}
%
% \subsection{Combined symbols}
%
%    \begin{macrocode}
\def\@make@colon@set#1#2{%
  \begingroup
    \let\@center@colon\relax
    \let\newcommand\relax
    \let\ensuremath\relax
    \let\mathrel\relax
    \let\colonsep\relax
    \let\doublecolonsep\relax
    \def\csx##1{%
      \expandafter\noexpand\csname ##1\endcsname
    }%
    \edef\x{\endgroup
      \newcommand*{\csx{colon#1}}{%
        \ensuremath{%
          \mathrel{%
            \@center@colon
            \colonsep
            {#2}%
          }%
        }%
      }%
      \newcommand*{\csx{coloncolon#1}}{%
        \ensuremath{%
          \mathrel{%
            \@center@colon
            \doublecolonsep
            \@center@colon
            \colonsep
            {#2}%
          }%
        }%
      }%
      \newcommand*{\csx{#1colon}}{%
        \ensuremath{%
          \mathrel{%
            {#2}%
            \colonsep
            \@center@colon
          }%
        }%
      }%
      \newcommand*{\csx{#1coloncolon}}{%
        \ensuremath{%
          \mathrel{%
            {#2}%
            \colonsep
            \@center@colon
            \doublecolonsep
            \@center@colon
          }%
        }%
      }%
    }%
  \x
}
%    \end{macrocode}
%
%    \begin{macrocode}
\@make@colon@set{equals}{=}%
\@make@colon@set{minus}{-}%
\@make@colon@set{approx}{\approx}
\@make@colon@set{sim}{\sim}
%    \end{macrocode}
%
%    \begin{macrocode}
%</package>
%    \end{macrocode}
%
% \section{Installation}
%
% \subsection{Download}
%
% \paragraph{Package.} This package is available on
% CTAN\footnote{\url{ftp://ftp.ctan.org/tex-archive/}}:
% \begin{description}
% \item[\CTAN{macros/latex/contrib/oberdiek/colonequals.dtx}] The source file.
% \item[\CTAN{macros/latex/contrib/oberdiek/colonequals.pdf}] Documentation.
% \end{description}
%
%
% \paragraph{Bundle.} All the packages of the bundle `oberdiek'
% are also available in a TDS compliant ZIP archive. There
% the packages are already unpacked and the documentation files
% are generated. The files and directories obey the TDS standard.
% \begin{description}
% \item[\CTAN{install/macros/latex/contrib/oberdiek.tds.zip}]
% \end{description}
% \emph{TDS} refers to the standard ``A Directory Structure
% for \TeX\ Files'' (\CTAN{tds/tds.pdf}). Directories
% with \xfile{texmf} in their name are usually organized this way.
%
% \subsection{Bundle installation}
%
% \paragraph{Unpacking.} Unpack the \xfile{oberdiek.tds.zip} in the
% TDS tree (also known as \xfile{texmf} tree) of your choice.
% Example (linux):
% \begin{quote}
%   |unzip oberdiek.tds.zip -d ~/texmf|
% \end{quote}
%
% \paragraph{Script installation.}
% Check the directory \xfile{TDS:scripts/oberdiek/} for
% scripts that need further installation steps.
% Package \xpackage{attachfile2} comes with the Perl script
% \xfile{pdfatfi.pl} that should be installed in such a way
% that it can be called as \texttt{pdfatfi}.
% Example (linux):
% \begin{quote}
%   |chmod +x scripts/oberdiek/pdfatfi.pl|\\
%   |cp scripts/oberdiek/pdfatfi.pl /usr/local/bin/|
% \end{quote}
%
% \subsection{Package installation}
%
% \paragraph{Unpacking.} The \xfile{.dtx} file is a self-extracting
% \docstrip\ archive. The files are extracted by running the
% \xfile{.dtx} through \plainTeX:
% \begin{quote}
%   \verb|tex colonequals.dtx|
% \end{quote}
%
% \paragraph{TDS.} Now the different files must be moved into
% the different directories in your installation TDS tree
% (also known as \xfile{texmf} tree):
% \begin{quote}
% \def\t{^^A
% \begin{tabular}{@{}>{\ttfamily}l@{ $\rightarrow$ }>{\ttfamily}l@{}}
%   colonequals.sty & tex/latex/oberdiek/colonequals.sty\\
%   colonequals.pdf & doc/latex/oberdiek/colonequals.pdf\\
%   colonequals.dtx & source/latex/oberdiek/colonequals.dtx\\
% \end{tabular}^^A
% }^^A
% \sbox0{\t}^^A
% \ifdim\wd0>\linewidth
%   \begingroup
%     \advance\linewidth by\leftmargin
%     \advance\linewidth by\rightmargin
%   \edef\x{\endgroup
%     \def\noexpand\lw{\the\linewidth}^^A
%   }\x
%   \def\lwbox{^^A
%     \leavevmode
%     \hbox to \linewidth{^^A
%       \kern-\leftmargin\relax
%       \hss
%       \usebox0
%       \hss
%       \kern-\rightmargin\relax
%     }^^A
%   }^^A
%   \ifdim\wd0>\lw
%     \sbox0{\small\t}^^A
%     \ifdim\wd0>\linewidth
%       \ifdim\wd0>\lw
%         \sbox0{\footnotesize\t}^^A
%         \ifdim\wd0>\linewidth
%           \ifdim\wd0>\lw
%             \sbox0{\scriptsize\t}^^A
%             \ifdim\wd0>\linewidth
%               \ifdim\wd0>\lw
%                 \sbox0{\tiny\t}^^A
%                 \ifdim\wd0>\linewidth
%                   \lwbox
%                 \else
%                   \usebox0
%                 \fi
%               \else
%                 \lwbox
%               \fi
%             \else
%               \usebox0
%             \fi
%           \else
%             \lwbox
%           \fi
%         \else
%           \usebox0
%         \fi
%       \else
%         \lwbox
%       \fi
%     \else
%       \usebox0
%     \fi
%   \else
%     \lwbox
%   \fi
% \else
%   \usebox0
% \fi
% \end{quote}
% If you have a \xfile{docstrip.cfg} that configures and enables \docstrip's
% TDS installing feature, then some files can already be in the right
% place, see the documentation of \docstrip.
%
% \subsection{Refresh file name databases}
%
% If your \TeX~distribution
% (\teTeX, \mikTeX, \dots) relies on file name databases, you must refresh
% these. For example, \teTeX\ users run \verb|texhash| or
% \verb|mktexlsr|.
%
% \subsection{Some details for the interested}
%
% \paragraph{Attached source.}
%
% The PDF documentation on CTAN also includes the
% \xfile{.dtx} source file. It can be extracted by
% AcrobatReader 6 or higher. Another option is \textsf{pdftk},
% e.g. unpack the file into the current directory:
% \begin{quote}
%   \verb|pdftk colonequals.pdf unpack_files output .|
% \end{quote}
%
% \paragraph{Unpacking with \LaTeX.}
% The \xfile{.dtx} chooses its action depending on the format:
% \begin{description}
% \item[\plainTeX:] Run \docstrip\ and extract the files.
% \item[\LaTeX:] Generate the documentation.
% \end{description}
% If you insist on using \LaTeX\ for \docstrip\ (really,
% \docstrip\ does not need \LaTeX), then inform the autodetect routine
% about your intention:
% \begin{quote}
%   \verb|latex \let\install=y% \iffalse meta-comment
%
% File: colonequals.dtx
% Version: 2006/08/01 v1.0
% Info: Colon equals symbols
%
% Copyright (C) 2006 by
%    Heiko Oberdiek <heiko.oberdiek at googlemail.com>
%
% This work may be distributed and/or modified under the
% conditions of the LaTeX Project Public License, either
% version 1.3c of this license or (at your option) any later
% version. This version of this license is in
%    http://www.latex-project.org/lppl/lppl-1-3c.txt
% and the latest version of this license is in
%    http://www.latex-project.org/lppl.txt
% and version 1.3 or later is part of all distributions of
% LaTeX version 2005/12/01 or later.
%
% This work has the LPPL maintenance status "maintained".
%
% This Current Maintainer of this work is Heiko Oberdiek.
%
% This work consists of the main source file colonequals.dtx
% and the derived files
%    colonequals.sty, colonequals.pdf, colonequals.ins, colonequals.drv.
%
% Distribution:
%    CTAN:macros/latex/contrib/oberdiek/colonequals.dtx
%    CTAN:macros/latex/contrib/oberdiek/colonequals.pdf
%
% Unpacking:
%    (a) If colonequals.ins is present:
%           tex colonequals.ins
%    (b) Without colonequals.ins:
%           tex colonequals.dtx
%    (c) If you insist on using LaTeX
%           latex \let\install=y\input{colonequals.dtx}
%        (quote the arguments according to the demands of your shell)
%
% Documentation:
%    (a) If colonequals.drv is present:
%           latex colonequals.drv
%    (b) Without colonequals.drv:
%           latex colonequals.dtx; ...
%    The class ltxdoc loads the configuration file ltxdoc.cfg
%    if available. Here you can specify further options, e.g.
%    use A4 as paper format:
%       \PassOptionsToClass{a4paper}{article}
%
%    Programm calls to get the documentation (example):
%       pdflatex colonequals.dtx
%       makeindex -s gind.ist colonequals.idx
%       pdflatex colonequals.dtx
%       makeindex -s gind.ist colonequals.idx
%       pdflatex colonequals.dtx
%
% Installation:
%    TDS:tex/latex/oberdiek/colonequals.sty
%    TDS:doc/latex/oberdiek/colonequals.pdf
%    TDS:source/latex/oberdiek/colonequals.dtx
%
%<*ignore>
\begingroup
  \catcode123=1 %
  \catcode125=2 %
  \def\x{LaTeX2e}%
\expandafter\endgroup
\ifcase 0\ifx\install y1\fi\expandafter
         \ifx\csname processbatchFile\endcsname\relax\else1\fi
         \ifx\fmtname\x\else 1\fi\relax
\else\csname fi\endcsname
%</ignore>
%<*install>
\input docstrip.tex
\Msg{************************************************************************}
\Msg{* Installation}
\Msg{* Package: colonequals 2006/08/01 v1.0 Colon equals symbols (HO)}
\Msg{************************************************************************}

\keepsilent
\askforoverwritefalse

\let\MetaPrefix\relax
\preamble

This is a generated file.

Project: colonequals
Version: 2006/08/01 v1.0

Copyright (C) 2006 by
   Heiko Oberdiek <heiko.oberdiek at googlemail.com>

This work may be distributed and/or modified under the
conditions of the LaTeX Project Public License, either
version 1.3c of this license or (at your option) any later
version. This version of this license is in
   http://www.latex-project.org/lppl/lppl-1-3c.txt
and the latest version of this license is in
   http://www.latex-project.org/lppl.txt
and version 1.3 or later is part of all distributions of
LaTeX version 2005/12/01 or later.

This work has the LPPL maintenance status "maintained".

This Current Maintainer of this work is Heiko Oberdiek.

This work consists of the main source file colonequals.dtx
and the derived files
   colonequals.sty, colonequals.pdf, colonequals.ins, colonequals.drv.

\endpreamble
\let\MetaPrefix\DoubleperCent

\generate{%
  \file{colonequals.ins}{\from{colonequals.dtx}{install}}%
  \file{colonequals.drv}{\from{colonequals.dtx}{driver}}%
  \usedir{tex/latex/oberdiek}%
  \file{colonequals.sty}{\from{colonequals.dtx}{package}}%
  \nopreamble
  \nopostamble
  \usedir{source/latex/oberdiek/catalogue}%
  \file{colonequals.xml}{\from{colonequals.dtx}{catalogue}}%
}

\catcode32=13\relax% active space
\let =\space%
\Msg{************************************************************************}
\Msg{*}
\Msg{* To finish the installation you have to move the following}
\Msg{* file into a directory searched by TeX:}
\Msg{*}
\Msg{*     colonequals.sty}
\Msg{*}
\Msg{* To produce the documentation run the file `colonequals.drv'}
\Msg{* through LaTeX.}
\Msg{*}
\Msg{* Happy TeXing!}
\Msg{*}
\Msg{************************************************************************}

\endbatchfile
%</install>
%<*ignore>
\fi
%</ignore>
%<*driver>
\NeedsTeXFormat{LaTeX2e}
\ProvidesFile{colonequals.drv}%
  [2006/08/01 v1.0 Colon equals symbols (HO)]%
\documentclass{ltxdoc}
\usepackage{holtxdoc}[2011/11/22]
\usepackage{colonequals}
\usepackage{array}
\usepackage{capt-of}
\usepackage{longtable}
\begin{document}
  \DocInput{colonequals.dtx}%
\end{document}
%</driver>
% \fi
%
% \CheckSum{92}
%
% \CharacterTable
%  {Upper-case    \A\B\C\D\E\F\G\H\I\J\K\L\M\N\O\P\Q\R\S\T\U\V\W\X\Y\Z
%   Lower-case    \a\b\c\d\e\f\g\h\i\j\k\l\m\n\o\p\q\r\s\t\u\v\w\x\y\z
%   Digits        \0\1\2\3\4\5\6\7\8\9
%   Exclamation   \!     Double quote  \"     Hash (number) \#
%   Dollar        \$     Percent       \%     Ampersand     \&
%   Acute accent  \'     Left paren    \(     Right paren   \)
%   Asterisk      \*     Plus          \+     Comma         \,
%   Minus         \-     Point         \.     Solidus       \/
%   Colon         \:     Semicolon     \;     Less than     \<
%   Equals        \=     Greater than  \>     Question mark \?
%   Commercial at \@     Left bracket  \[     Backslash     \\
%   Right bracket \]     Circumflex    \^     Underscore    \_
%   Grave accent  \`     Left brace    \{     Vertical bar  \|
%   Right brace   \}     Tilde         \~}
%
% \GetFileInfo{colonequals.drv}
%
% \title{The \xpackage{colonequals} package}
% \date{2006/08/01 v1.0}
% \author{Heiko Oberdiek\\\xemail{heiko.oberdiek at googlemail.com}}
%
% \maketitle
%
% \begin{abstract}
% Package \xpackage{colonequals} defines poor man's symbols
% for math relation symbols such as ``colon equals''.
% The colon is centered around the horizontal math axis.
% \end{abstract}
%
% \tableofcontents
%
% \section{User interface}
%
% \subsection{Introduction}
%
% Math symbols consisting of the colon character can be
% constructed with the colon text character, if the math font
% lacks of the complete symbol. Often, however, the colon text
% character is not centered around the math axis. Especially
% combined with the equals symbol the composed symbol does not
% look symmetrically. Thus this packages defines a colon
% math symbol \cs{ratio} that is centered around the horizontal
% math axis. Also math symbols are provided that consist of the
% colon symbol. The package is not necessary, if the math
% font contains the composed symbols. Examples are \textsf{txfonts}
% (\cite{txfonts}) or \textsf{mathabx} (\cite{mathabx}).
%
% \subsection{Symbols}
%
% All symbols of this package are relation symbols.
% The relation property can be changed by the appropriate
% \TeX\ command \cs{mathbin}, \cs{mathord}, \dots
%
% \begin{center}
% \captionof{table}{Unicode mathematical operators}
% \kern1ex
% \begin{tabular}{l>{\scshape}l>{$}l<{$}l}
%   U+2236 & ratio        & \ratio       & \cs{ratio}       \\
%   U+2237 & proportion   & \coloncolon  & \cs{coloncolon}  \\
%   U+2239 & excess       & \colonminus  & \cs{colonminus}  \\
%   U+2254 & colon equals & \colonequals & \cs{colonequals} \\
%   U+2255 & equals colon & \equalscolon & \cs{equalscolon} \\
% \end{tabular}
% \end{center}
%
% The following grammar generates all symbols that are supported by
% this package:
% \begin{center}
% \captionof{table}{Symbol grammar}
% \kern1ex
% \begin{tabular}{@{}l>{$}r<{$}l@{}}
%   symbols & \coloncolonequals & col \\
%           & \mid & col symbol \\
%           & \mid & symbol col \\
%           & ; & \\[1ex]
%   col     & \coloncolonequals & '\texttt{:}' \\
%           & \mid & '\texttt{::}' \\
%           & ; & \\[1ex]
%   symbol  & \coloncolonequals & '\texttt{=}' \\
%           & \mid & '\texttt{-}' \\
%           & \mid & '$\approx$' \\
%           & \mid & '$\sim$' \\
%           & ; &
% \end{tabular}
% \end{center}
%
% \def\entry#1{\csname #1\endcsname&\cs{#1}\\}
% \def\entryset#1{^^A
%    \entry{colon#1}^^A
%    \entry{coloncolon#1}^^A
%    \entry{#1colon}^^A
%    \entry{#1coloncolon}^^A
% }
% \begin{longtable}{>{$}l<{$}l}
%   \caption{All relation symbols}\\
%   \entry{ratio}
%   \entry{coloncolon}
%   \entryset{equals}
%   \entryset{minus}
%   \entryset{approx}
%   \entryset{sim}
% \end{longtable}
%
% \subsection{Fine tuning}
%
% The distances in composed symbols can be configured:
%
% \begin{declcs}{colonsep}
% \end{declcs}
% Macro \cs{colonsep} is executed between the colon and
% the other symbol.
%
% \begin{declcs}{doublecolonsep}
% \end{declcs}
% Macro \cs{doublecolonsep} controls the distance between
% two colons.
%
% \subsubsection{Example}
% \begin{quote}
%   \verb|\renewcommand*{\colonsep}{\mskip-.5\thinmuskip}|
% \end{quote}
%
%
% \StopEventually{
% }
%
% \section{Implementation}
%
% \subsection{Identification}
%
%    \begin{macrocode}
%<*package>
\NeedsTeXFormat{LaTeX2e}
\ProvidesPackage{colonequals}%
  [2006/08/01 v1.0 Colon equals symbols (HO)]%
%    \end{macrocode}
%
% \subsection{Distance control}
%
%    \begin{macro}{\colonsep}
%    \begin{macrocode}
\newcommand*{\colonsep}{}
%    \end{macrocode}
%    \end{macro}
%
%    \begin{macro}{\doublecolonsep}
%    \begin{macrocode}
\newcommand*{\doublecolonsep}{}
%    \end{macrocode}
%    \end{macro}
%
% \subsection{Centered colons}
%
%    \begin{macrocode}
\def\@center@colon{%
  \mathpalette\@center@math{:}%
}
\def\@center@math#1#2{%
  \vcenter{%
    \m@th
    \hbox{$#1#2$}%
  }%
}
%    \end{macrocode}
%
%    \begin{macro}{\ratio}
%    Because the name \cs{colon} is already in use, the Unicode name
%    \cs{ratio} is used for the centered colon relation symbol.
%    (The \cs{ratio} of package \textsf{calc} is not used outside
%    calc expressions.)
%    \begin{macrocode}
\newcommand*{\ratio}{%
  \ensuremath{%
    \mathrel{%
      \@center@colon
    }%
  }%
}
%    \end{macrocode}
%    \end{macro}
%
%    \begin{macro}{\coloncolon}
%    \begin{macrocode}
\newcommand*{\coloncolon}{%
  \ensuremath{%
    \mathrel{%
      \@center@colon
      \doublecolonsep
      \@center@colon
    }%
  }%
}
%    \end{macrocode}
%    \end{macro}
%
% \subsection{Combined symbols}
%
%    \begin{macrocode}
\def\@make@colon@set#1#2{%
  \begingroup
    \let\@center@colon\relax
    \let\newcommand\relax
    \let\ensuremath\relax
    \let\mathrel\relax
    \let\colonsep\relax
    \let\doublecolonsep\relax
    \def\csx##1{%
      \expandafter\noexpand\csname ##1\endcsname
    }%
    \edef\x{\endgroup
      \newcommand*{\csx{colon#1}}{%
        \ensuremath{%
          \mathrel{%
            \@center@colon
            \colonsep
            {#2}%
          }%
        }%
      }%
      \newcommand*{\csx{coloncolon#1}}{%
        \ensuremath{%
          \mathrel{%
            \@center@colon
            \doublecolonsep
            \@center@colon
            \colonsep
            {#2}%
          }%
        }%
      }%
      \newcommand*{\csx{#1colon}}{%
        \ensuremath{%
          \mathrel{%
            {#2}%
            \colonsep
            \@center@colon
          }%
        }%
      }%
      \newcommand*{\csx{#1coloncolon}}{%
        \ensuremath{%
          \mathrel{%
            {#2}%
            \colonsep
            \@center@colon
            \doublecolonsep
            \@center@colon
          }%
        }%
      }%
    }%
  \x
}
%    \end{macrocode}
%
%    \begin{macrocode}
\@make@colon@set{equals}{=}%
\@make@colon@set{minus}{-}%
\@make@colon@set{approx}{\approx}
\@make@colon@set{sim}{\sim}
%    \end{macrocode}
%
%    \begin{macrocode}
%</package>
%    \end{macrocode}
%
% \section{Installation}
%
% \subsection{Download}
%
% \paragraph{Package.} This package is available on
% CTAN\footnote{\url{ftp://ftp.ctan.org/tex-archive/}}:
% \begin{description}
% \item[\CTAN{macros/latex/contrib/oberdiek/colonequals.dtx}] The source file.
% \item[\CTAN{macros/latex/contrib/oberdiek/colonequals.pdf}] Documentation.
% \end{description}
%
%
% \paragraph{Bundle.} All the packages of the bundle `oberdiek'
% are also available in a TDS compliant ZIP archive. There
% the packages are already unpacked and the documentation files
% are generated. The files and directories obey the TDS standard.
% \begin{description}
% \item[\CTAN{install/macros/latex/contrib/oberdiek.tds.zip}]
% \end{description}
% \emph{TDS} refers to the standard ``A Directory Structure
% for \TeX\ Files'' (\CTAN{tds/tds.pdf}). Directories
% with \xfile{texmf} in their name are usually organized this way.
%
% \subsection{Bundle installation}
%
% \paragraph{Unpacking.} Unpack the \xfile{oberdiek.tds.zip} in the
% TDS tree (also known as \xfile{texmf} tree) of your choice.
% Example (linux):
% \begin{quote}
%   |unzip oberdiek.tds.zip -d ~/texmf|
% \end{quote}
%
% \paragraph{Script installation.}
% Check the directory \xfile{TDS:scripts/oberdiek/} for
% scripts that need further installation steps.
% Package \xpackage{attachfile2} comes with the Perl script
% \xfile{pdfatfi.pl} that should be installed in such a way
% that it can be called as \texttt{pdfatfi}.
% Example (linux):
% \begin{quote}
%   |chmod +x scripts/oberdiek/pdfatfi.pl|\\
%   |cp scripts/oberdiek/pdfatfi.pl /usr/local/bin/|
% \end{quote}
%
% \subsection{Package installation}
%
% \paragraph{Unpacking.} The \xfile{.dtx} file is a self-extracting
% \docstrip\ archive. The files are extracted by running the
% \xfile{.dtx} through \plainTeX:
% \begin{quote}
%   \verb|tex colonequals.dtx|
% \end{quote}
%
% \paragraph{TDS.} Now the different files must be moved into
% the different directories in your installation TDS tree
% (also known as \xfile{texmf} tree):
% \begin{quote}
% \def\t{^^A
% \begin{tabular}{@{}>{\ttfamily}l@{ $\rightarrow$ }>{\ttfamily}l@{}}
%   colonequals.sty & tex/latex/oberdiek/colonequals.sty\\
%   colonequals.pdf & doc/latex/oberdiek/colonequals.pdf\\
%   colonequals.dtx & source/latex/oberdiek/colonequals.dtx\\
% \end{tabular}^^A
% }^^A
% \sbox0{\t}^^A
% \ifdim\wd0>\linewidth
%   \begingroup
%     \advance\linewidth by\leftmargin
%     \advance\linewidth by\rightmargin
%   \edef\x{\endgroup
%     \def\noexpand\lw{\the\linewidth}^^A
%   }\x
%   \def\lwbox{^^A
%     \leavevmode
%     \hbox to \linewidth{^^A
%       \kern-\leftmargin\relax
%       \hss
%       \usebox0
%       \hss
%       \kern-\rightmargin\relax
%     }^^A
%   }^^A
%   \ifdim\wd0>\lw
%     \sbox0{\small\t}^^A
%     \ifdim\wd0>\linewidth
%       \ifdim\wd0>\lw
%         \sbox0{\footnotesize\t}^^A
%         \ifdim\wd0>\linewidth
%           \ifdim\wd0>\lw
%             \sbox0{\scriptsize\t}^^A
%             \ifdim\wd0>\linewidth
%               \ifdim\wd0>\lw
%                 \sbox0{\tiny\t}^^A
%                 \ifdim\wd0>\linewidth
%                   \lwbox
%                 \else
%                   \usebox0
%                 \fi
%               \else
%                 \lwbox
%               \fi
%             \else
%               \usebox0
%             \fi
%           \else
%             \lwbox
%           \fi
%         \else
%           \usebox0
%         \fi
%       \else
%         \lwbox
%       \fi
%     \else
%       \usebox0
%     \fi
%   \else
%     \lwbox
%   \fi
% \else
%   \usebox0
% \fi
% \end{quote}
% If you have a \xfile{docstrip.cfg} that configures and enables \docstrip's
% TDS installing feature, then some files can already be in the right
% place, see the documentation of \docstrip.
%
% \subsection{Refresh file name databases}
%
% If your \TeX~distribution
% (\teTeX, \mikTeX, \dots) relies on file name databases, you must refresh
% these. For example, \teTeX\ users run \verb|texhash| or
% \verb|mktexlsr|.
%
% \subsection{Some details for the interested}
%
% \paragraph{Attached source.}
%
% The PDF documentation on CTAN also includes the
% \xfile{.dtx} source file. It can be extracted by
% AcrobatReader 6 or higher. Another option is \textsf{pdftk},
% e.g. unpack the file into the current directory:
% \begin{quote}
%   \verb|pdftk colonequals.pdf unpack_files output .|
% \end{quote}
%
% \paragraph{Unpacking with \LaTeX.}
% The \xfile{.dtx} chooses its action depending on the format:
% \begin{description}
% \item[\plainTeX:] Run \docstrip\ and extract the files.
% \item[\LaTeX:] Generate the documentation.
% \end{description}
% If you insist on using \LaTeX\ for \docstrip\ (really,
% \docstrip\ does not need \LaTeX), then inform the autodetect routine
% about your intention:
% \begin{quote}
%   \verb|latex \let\install=y\input{colonequals.dtx}|
% \end{quote}
% Do not forget to quote the argument according to the demands
% of your shell.
%
% \paragraph{Generating the documentation.}
% You can use both the \xfile{.dtx} or the \xfile{.drv} to generate
% the documentation. The process can be configured by the
% configuration file \xfile{ltxdoc.cfg}. For instance, put this
% line into this file, if you want to have A4 as paper format:
% \begin{quote}
%   \verb|\PassOptionsToClass{a4paper}{article}|
% \end{quote}
% An example follows how to generate the
% documentation with pdf\LaTeX:
% \begin{quote}
%\begin{verbatim}
%pdflatex colonequals.dtx
%makeindex -s gind.ist colonequals.idx
%pdflatex colonequals.dtx
%makeindex -s gind.ist colonequals.idx
%pdflatex colonequals.dtx
%\end{verbatim}
% \end{quote}
%
% \section{Catalogue}
%
% The following XML file can be used as source for the
% \href{http://mirror.ctan.org/help/Catalogue/catalogue.html}{\TeX\ Catalogue}.
% The elements \texttt{caption} and \texttt{description} are imported
% from the original XML file from the Catalogue.
% The name of the XML file in the Catalogue is \xfile{colonequals.xml}.
%    \begin{macrocode}
%<*catalogue>
<?xml version='1.0' encoding='us-ascii'?>
<!DOCTYPE entry SYSTEM 'catalogue.dtd'>
<entry datestamp='$Date$' modifier='$Author$' id='colonequals'>
  <name>colonequals</name>
  <caption>Colon equals symbols.</caption>
  <authorref id='auth:oberdiek'/>
  <copyright owner='Heiko Oberdiek' year='2006'/>
  <license type='lppl1.3'/>
  <version number='1.0'/>
  <description>
    This package defines poor man&#x2018;s symbols for mathematical
    relation symbols such as &#x201C;colon equals&#x201D;.
    The colon is centered around the horizontal math axis.
    <p/>
    The package is part of the <xref refid='oberdiek'>oberdiek</xref>
    bundle.
  </description>
  <documentation details='Package documentation'
      href='ctan:/macros/latex/contrib/oberdiek/colonequals.pdf'/>
  <ctan file='true' path='/macros/latex/contrib/oberdiek/colonequals.dtx'/>
  <miktex location='oberdiek'/>
  <texlive location='oberdiek'/>
  <install path='/macros/latex/contrib/oberdiek/oberdiek.tds.zip'/>
</entry>
%</catalogue>
%    \end{macrocode}
%
% \begin{thebibliography}{9}
%
% \bibitem{txfonts}
%   Young Ryu: \textit{The TX Fonts};
%   2000/12/15;
%   \CTAN{fonts/txfonts/}.
%
% \bibitem{mathabx}
%   Anthony Phan: \textit{Mathabx font series};
%   2005/05/16;
%   \CTAN{fonts/mathabx/}.
%
% \end{thebibliography}
%
% \begin{History}
%   \begin{Version}{2006/08/01 v1.0}
%   \item
%     First version.
%   \end{Version}
% \end{History}
%
% \PrintIndex
%
% \Finale
\endinput
|
% \end{quote}
% Do not forget to quote the argument according to the demands
% of your shell.
%
% \paragraph{Generating the documentation.}
% You can use both the \xfile{.dtx} or the \xfile{.drv} to generate
% the documentation. The process can be configured by the
% configuration file \xfile{ltxdoc.cfg}. For instance, put this
% line into this file, if you want to have A4 as paper format:
% \begin{quote}
%   \verb|\PassOptionsToClass{a4paper}{article}|
% \end{quote}
% An example follows how to generate the
% documentation with pdf\LaTeX:
% \begin{quote}
%\begin{verbatim}
%pdflatex colonequals.dtx
%makeindex -s gind.ist colonequals.idx
%pdflatex colonequals.dtx
%makeindex -s gind.ist colonequals.idx
%pdflatex colonequals.dtx
%\end{verbatim}
% \end{quote}
%
% \section{Catalogue}
%
% The following XML file can be used as source for the
% \href{http://mirror.ctan.org/help/Catalogue/catalogue.html}{\TeX\ Catalogue}.
% The elements \texttt{caption} and \texttt{description} are imported
% from the original XML file from the Catalogue.
% The name of the XML file in the Catalogue is \xfile{colonequals.xml}.
%    \begin{macrocode}
%<*catalogue>
<?xml version='1.0' encoding='us-ascii'?>
<!DOCTYPE entry SYSTEM 'catalogue.dtd'>
<entry datestamp='$Date$' modifier='$Author$' id='colonequals'>
  <name>colonequals</name>
  <caption>Colon equals symbols.</caption>
  <authorref id='auth:oberdiek'/>
  <copyright owner='Heiko Oberdiek' year='2006'/>
  <license type='lppl1.3'/>
  <version number='1.0'/>
  <description>
    This package defines poor man&#x2018;s symbols for mathematical
    relation symbols such as &#x201C;colon equals&#x201D;.
    The colon is centered around the horizontal math axis.
    <p/>
    The package is part of the <xref refid='oberdiek'>oberdiek</xref>
    bundle.
  </description>
  <documentation details='Package documentation'
      href='ctan:/macros/latex/contrib/oberdiek/colonequals.pdf'/>
  <ctan file='true' path='/macros/latex/contrib/oberdiek/colonequals.dtx'/>
  <miktex location='oberdiek'/>
  <texlive location='oberdiek'/>
  <install path='/macros/latex/contrib/oberdiek/oberdiek.tds.zip'/>
</entry>
%</catalogue>
%    \end{macrocode}
%
% \begin{thebibliography}{9}
%
% \bibitem{txfonts}
%   Young Ryu: \textit{The TX Fonts};
%   2000/12/15;
%   \CTAN{fonts/txfonts/}.
%
% \bibitem{mathabx}
%   Anthony Phan: \textit{Mathabx font series};
%   2005/05/16;
%   \CTAN{fonts/mathabx/}.
%
% \end{thebibliography}
%
% \begin{History}
%   \begin{Version}{2006/08/01 v1.0}
%   \item
%     First version.
%   \end{Version}
% \end{History}
%
% \PrintIndex
%
% \Finale
\endinput
|
% \end{quote}
% Do not forget to quote the argument according to the demands
% of your shell.
%
% \paragraph{Generating the documentation.}
% You can use both the \xfile{.dtx} or the \xfile{.drv} to generate
% the documentation. The process can be configured by the
% configuration file \xfile{ltxdoc.cfg}. For instance, put this
% line into this file, if you want to have A4 as paper format:
% \begin{quote}
%   \verb|\PassOptionsToClass{a4paper}{article}|
% \end{quote}
% An example follows how to generate the
% documentation with pdf\LaTeX:
% \begin{quote}
%\begin{verbatim}
%pdflatex colonequals.dtx
%makeindex -s gind.ist colonequals.idx
%pdflatex colonequals.dtx
%makeindex -s gind.ist colonequals.idx
%pdflatex colonequals.dtx
%\end{verbatim}
% \end{quote}
%
% \section{Catalogue}
%
% The following XML file can be used as source for the
% \href{http://mirror.ctan.org/help/Catalogue/catalogue.html}{\TeX\ Catalogue}.
% The elements \texttt{caption} and \texttt{description} are imported
% from the original XML file from the Catalogue.
% The name of the XML file in the Catalogue is \xfile{colonequals.xml}.
%    \begin{macrocode}
%<*catalogue>
<?xml version='1.0' encoding='us-ascii'?>
<!DOCTYPE entry SYSTEM 'catalogue.dtd'>
<entry datestamp='$Date$' modifier='$Author$' id='colonequals'>
  <name>colonequals</name>
  <caption>Colon equals symbols.</caption>
  <authorref id='auth:oberdiek'/>
  <copyright owner='Heiko Oberdiek' year='2006'/>
  <license type='lppl1.3'/>
  <version number='1.0'/>
  <description>
    This package defines poor man&#x2018;s symbols for mathematical
    relation symbols such as &#x201C;colon equals&#x201D;.
    The colon is centered around the horizontal math axis.
    <p/>
    The package is part of the <xref refid='oberdiek'>oberdiek</xref>
    bundle.
  </description>
  <documentation details='Package documentation'
      href='ctan:/macros/latex/contrib/oberdiek/colonequals.pdf'/>
  <ctan file='true' path='/macros/latex/contrib/oberdiek/colonequals.dtx'/>
  <miktex location='oberdiek'/>
  <texlive location='oberdiek'/>
  <install path='/macros/latex/contrib/oberdiek/oberdiek.tds.zip'/>
</entry>
%</catalogue>
%    \end{macrocode}
%
% \begin{thebibliography}{9}
%
% \bibitem{txfonts}
%   Young Ryu: \textit{The TX Fonts};
%   2000/12/15;
%   \CTAN{fonts/txfonts/}.
%
% \bibitem{mathabx}
%   Anthony Phan: \textit{Mathabx font series};
%   2005/05/16;
%   \CTAN{fonts/mathabx/}.
%
% \end{thebibliography}
%
% \begin{History}
%   \begin{Version}{2006/08/01 v1.0}
%   \item
%     First version.
%   \end{Version}
% \end{History}
%
% \PrintIndex
%
% \Finale
\endinput

%        (quote the arguments according to the demands of your shell)
%
% Documentation:
%    (a) If colonequals.drv is present:
%           latex colonequals.drv
%    (b) Without colonequals.drv:
%           latex colonequals.dtx; ...
%    The class ltxdoc loads the configuration file ltxdoc.cfg
%    if available. Here you can specify further options, e.g.
%    use A4 as paper format:
%       \PassOptionsToClass{a4paper}{article}
%
%    Programm calls to get the documentation (example):
%       pdflatex colonequals.dtx
%       makeindex -s gind.ist colonequals.idx
%       pdflatex colonequals.dtx
%       makeindex -s gind.ist colonequals.idx
%       pdflatex colonequals.dtx
%
% Installation:
%    TDS:tex/latex/oberdiek/colonequals.sty
%    TDS:doc/latex/oberdiek/colonequals.pdf
%    TDS:source/latex/oberdiek/colonequals.dtx
%
%<*ignore>
\begingroup
  \catcode123=1 %
  \catcode125=2 %
  \def\x{LaTeX2e}%
\expandafter\endgroup
\ifcase 0\ifx\install y1\fi\expandafter
         \ifx\csname processbatchFile\endcsname\relax\else1\fi
         \ifx\fmtname\x\else 1\fi\relax
\else\csname fi\endcsname
%</ignore>
%<*install>
\input docstrip.tex
\Msg{************************************************************************}
\Msg{* Installation}
\Msg{* Package: colonequals 2006/08/01 v1.0 Colon equals symbols (HO)}
\Msg{************************************************************************}

\keepsilent
\askforoverwritefalse

\let\MetaPrefix\relax
\preamble

This is a generated file.

Project: colonequals
Version: 2006/08/01 v1.0

Copyright (C) 2006 by
   Heiko Oberdiek <heiko.oberdiek at googlemail.com>

This work may be distributed and/or modified under the
conditions of the LaTeX Project Public License, either
version 1.3c of this license or (at your option) any later
version. This version of this license is in
   http://www.latex-project.org/lppl/lppl-1-3c.txt
and the latest version of this license is in
   http://www.latex-project.org/lppl.txt
and version 1.3 or later is part of all distributions of
LaTeX version 2005/12/01 or later.

This work has the LPPL maintenance status "maintained".

This Current Maintainer of this work is Heiko Oberdiek.

This work consists of the main source file colonequals.dtx
and the derived files
   colonequals.sty, colonequals.pdf, colonequals.ins, colonequals.drv.

\endpreamble
\let\MetaPrefix\DoubleperCent

\generate{%
  \file{colonequals.ins}{\from{colonequals.dtx}{install}}%
  \file{colonequals.drv}{\from{colonequals.dtx}{driver}}%
  \usedir{tex/latex/oberdiek}%
  \file{colonequals.sty}{\from{colonequals.dtx}{package}}%
  \nopreamble
  \nopostamble
  \usedir{source/latex/oberdiek/catalogue}%
  \file{colonequals.xml}{\from{colonequals.dtx}{catalogue}}%
}

\catcode32=13\relax% active space
\let =\space%
\Msg{************************************************************************}
\Msg{*}
\Msg{* To finish the installation you have to move the following}
\Msg{* file into a directory searched by TeX:}
\Msg{*}
\Msg{*     colonequals.sty}
\Msg{*}
\Msg{* To produce the documentation run the file `colonequals.drv'}
\Msg{* through LaTeX.}
\Msg{*}
\Msg{* Happy TeXing!}
\Msg{*}
\Msg{************************************************************************}

\endbatchfile
%</install>
%<*ignore>
\fi
%</ignore>
%<*driver>
\NeedsTeXFormat{LaTeX2e}
\ProvidesFile{colonequals.drv}%
  [2006/08/01 v1.0 Colon equals symbols (HO)]%
\documentclass{ltxdoc}
\usepackage{holtxdoc}[2011/11/22]
\usepackage{colonequals}
\usepackage{array}
\usepackage{capt-of}
\usepackage{longtable}
\begin{document}
  \DocInput{colonequals.dtx}%
\end{document}
%</driver>
% \fi
%
% \CheckSum{92}
%
% \CharacterTable
%  {Upper-case    \A\B\C\D\E\F\G\H\I\J\K\L\M\N\O\P\Q\R\S\T\U\V\W\X\Y\Z
%   Lower-case    \a\b\c\d\e\f\g\h\i\j\k\l\m\n\o\p\q\r\s\t\u\v\w\x\y\z
%   Digits        \0\1\2\3\4\5\6\7\8\9
%   Exclamation   \!     Double quote  \"     Hash (number) \#
%   Dollar        \$     Percent       \%     Ampersand     \&
%   Acute accent  \'     Left paren    \(     Right paren   \)
%   Asterisk      \*     Plus          \+     Comma         \,
%   Minus         \-     Point         \.     Solidus       \/
%   Colon         \:     Semicolon     \;     Less than     \<
%   Equals        \=     Greater than  \>     Question mark \?
%   Commercial at \@     Left bracket  \[     Backslash     \\
%   Right bracket \]     Circumflex    \^     Underscore    \_
%   Grave accent  \`     Left brace    \{     Vertical bar  \|
%   Right brace   \}     Tilde         \~}
%
% \GetFileInfo{colonequals.drv}
%
% \title{The \xpackage{colonequals} package}
% \date{2006/08/01 v1.0}
% \author{Heiko Oberdiek\\\xemail{heiko.oberdiek at googlemail.com}}
%
% \maketitle
%
% \begin{abstract}
% Package \xpackage{colonequals} defines poor man's symbols
% for math relation symbols such as ``colon equals''.
% The colon is centered around the horizontal math axis.
% \end{abstract}
%
% \tableofcontents
%
% \section{User interface}
%
% \subsection{Introduction}
%
% Math symbols consisting of the colon character can be
% constructed with the colon text character, if the math font
% lacks of the complete symbol. Often, however, the colon text
% character is not centered around the math axis. Especially
% combined with the equals symbol the composed symbol does not
% look symmetrically. Thus this packages defines a colon
% math symbol \cs{ratio} that is centered around the horizontal
% math axis. Also math symbols are provided that consist of the
% colon symbol. The package is not necessary, if the math
% font contains the composed symbols. Examples are \textsf{txfonts}
% (\cite{txfonts}) or \textsf{mathabx} (\cite{mathabx}).
%
% \subsection{Symbols}
%
% All symbols of this package are relation symbols.
% The relation property can be changed by the appropriate
% \TeX\ command \cs{mathbin}, \cs{mathord}, \dots
%
% \begin{center}
% \captionof{table}{Unicode mathematical operators}
% \kern1ex
% \begin{tabular}{l>{\scshape}l>{$}l<{$}l}
%   U+2236 & ratio        & \ratio       & \cs{ratio}       \\
%   U+2237 & proportion   & \coloncolon  & \cs{coloncolon}  \\
%   U+2239 & excess       & \colonminus  & \cs{colonminus}  \\
%   U+2254 & colon equals & \colonequals & \cs{colonequals} \\
%   U+2255 & equals colon & \equalscolon & \cs{equalscolon} \\
% \end{tabular}
% \end{center}
%
% The following grammar generates all symbols that are supported by
% this package:
% \begin{center}
% \captionof{table}{Symbol grammar}
% \kern1ex
% \begin{tabular}{@{}l>{$}r<{$}l@{}}
%   symbols & \coloncolonequals & col \\
%           & \mid & col symbol \\
%           & \mid & symbol col \\
%           & ; & \\[1ex]
%   col     & \coloncolonequals & '\texttt{:}' \\
%           & \mid & '\texttt{::}' \\
%           & ; & \\[1ex]
%   symbol  & \coloncolonequals & '\texttt{=}' \\
%           & \mid & '\texttt{-}' \\
%           & \mid & '$\approx$' \\
%           & \mid & '$\sim$' \\
%           & ; &
% \end{tabular}
% \end{center}
%
% \def\entry#1{\csname #1\endcsname&\cs{#1}\\}
% \def\entryset#1{^^A
%    \entry{colon#1}^^A
%    \entry{coloncolon#1}^^A
%    \entry{#1colon}^^A
%    \entry{#1coloncolon}^^A
% }
% \begin{longtable}{>{$}l<{$}l}
%   \caption{All relation symbols}\\
%   \entry{ratio}
%   \entry{coloncolon}
%   \entryset{equals}
%   \entryset{minus}
%   \entryset{approx}
%   \entryset{sim}
% \end{longtable}
%
% \subsection{Fine tuning}
%
% The distances in composed symbols can be configured:
%
% \begin{declcs}{colonsep}
% \end{declcs}
% Macro \cs{colonsep} is executed between the colon and
% the other symbol.
%
% \begin{declcs}{doublecolonsep}
% \end{declcs}
% Macro \cs{doublecolonsep} controls the distance between
% two colons.
%
% \subsubsection{Example}
% \begin{quote}
%   \verb|\renewcommand*{\colonsep}{\mskip-.5\thinmuskip}|
% \end{quote}
%
%
% \StopEventually{
% }
%
% \section{Implementation}
%
% \subsection{Identification}
%
%    \begin{macrocode}
%<*package>
\NeedsTeXFormat{LaTeX2e}
\ProvidesPackage{colonequals}%
  [2006/08/01 v1.0 Colon equals symbols (HO)]%
%    \end{macrocode}
%
% \subsection{Distance control}
%
%    \begin{macro}{\colonsep}
%    \begin{macrocode}
\newcommand*{\colonsep}{}
%    \end{macrocode}
%    \end{macro}
%
%    \begin{macro}{\doublecolonsep}
%    \begin{macrocode}
\newcommand*{\doublecolonsep}{}
%    \end{macrocode}
%    \end{macro}
%
% \subsection{Centered colons}
%
%    \begin{macrocode}
\def\@center@colon{%
  \mathpalette\@center@math{:}%
}
\def\@center@math#1#2{%
  \vcenter{%
    \m@th
    \hbox{$#1#2$}%
  }%
}
%    \end{macrocode}
%
%    \begin{macro}{\ratio}
%    Because the name \cs{colon} is already in use, the Unicode name
%    \cs{ratio} is used for the centered colon relation symbol.
%    (The \cs{ratio} of package \textsf{calc} is not used outside
%    calc expressions.)
%    \begin{macrocode}
\newcommand*{\ratio}{%
  \ensuremath{%
    \mathrel{%
      \@center@colon
    }%
  }%
}
%    \end{macrocode}
%    \end{macro}
%
%    \begin{macro}{\coloncolon}
%    \begin{macrocode}
\newcommand*{\coloncolon}{%
  \ensuremath{%
    \mathrel{%
      \@center@colon
      \doublecolonsep
      \@center@colon
    }%
  }%
}
%    \end{macrocode}
%    \end{macro}
%
% \subsection{Combined symbols}
%
%    \begin{macrocode}
\def\@make@colon@set#1#2{%
  \begingroup
    \let\@center@colon\relax
    \let\newcommand\relax
    \let\ensuremath\relax
    \let\mathrel\relax
    \let\colonsep\relax
    \let\doublecolonsep\relax
    \def\csx##1{%
      \expandafter\noexpand\csname ##1\endcsname
    }%
    \edef\x{\endgroup
      \newcommand*{\csx{colon#1}}{%
        \ensuremath{%
          \mathrel{%
            \@center@colon
            \colonsep
            {#2}%
          }%
        }%
      }%
      \newcommand*{\csx{coloncolon#1}}{%
        \ensuremath{%
          \mathrel{%
            \@center@colon
            \doublecolonsep
            \@center@colon
            \colonsep
            {#2}%
          }%
        }%
      }%
      \newcommand*{\csx{#1colon}}{%
        \ensuremath{%
          \mathrel{%
            {#2}%
            \colonsep
            \@center@colon
          }%
        }%
      }%
      \newcommand*{\csx{#1coloncolon}}{%
        \ensuremath{%
          \mathrel{%
            {#2}%
            \colonsep
            \@center@colon
            \doublecolonsep
            \@center@colon
          }%
        }%
      }%
    }%
  \x
}
%    \end{macrocode}
%
%    \begin{macrocode}
\@make@colon@set{equals}{=}%
\@make@colon@set{minus}{-}%
\@make@colon@set{approx}{\approx}
\@make@colon@set{sim}{\sim}
%    \end{macrocode}
%
%    \begin{macrocode}
%</package>
%    \end{macrocode}
%
% \section{Installation}
%
% \subsection{Download}
%
% \paragraph{Package.} This package is available on
% CTAN\footnote{\url{ftp://ftp.ctan.org/tex-archive/}}:
% \begin{description}
% \item[\CTAN{macros/latex/contrib/oberdiek/colonequals.dtx}] The source file.
% \item[\CTAN{macros/latex/contrib/oberdiek/colonequals.pdf}] Documentation.
% \end{description}
%
%
% \paragraph{Bundle.} All the packages of the bundle `oberdiek'
% are also available in a TDS compliant ZIP archive. There
% the packages are already unpacked and the documentation files
% are generated. The files and directories obey the TDS standard.
% \begin{description}
% \item[\CTAN{install/macros/latex/contrib/oberdiek.tds.zip}]
% \end{description}
% \emph{TDS} refers to the standard ``A Directory Structure
% for \TeX\ Files'' (\CTAN{tds/tds.pdf}). Directories
% with \xfile{texmf} in their name are usually organized this way.
%
% \subsection{Bundle installation}
%
% \paragraph{Unpacking.} Unpack the \xfile{oberdiek.tds.zip} in the
% TDS tree (also known as \xfile{texmf} tree) of your choice.
% Example (linux):
% \begin{quote}
%   |unzip oberdiek.tds.zip -d ~/texmf|
% \end{quote}
%
% \paragraph{Script installation.}
% Check the directory \xfile{TDS:scripts/oberdiek/} for
% scripts that need further installation steps.
% Package \xpackage{attachfile2} comes with the Perl script
% \xfile{pdfatfi.pl} that should be installed in such a way
% that it can be called as \texttt{pdfatfi}.
% Example (linux):
% \begin{quote}
%   |chmod +x scripts/oberdiek/pdfatfi.pl|\\
%   |cp scripts/oberdiek/pdfatfi.pl /usr/local/bin/|
% \end{quote}
%
% \subsection{Package installation}
%
% \paragraph{Unpacking.} The \xfile{.dtx} file is a self-extracting
% \docstrip\ archive. The files are extracted by running the
% \xfile{.dtx} through \plainTeX:
% \begin{quote}
%   \verb|tex colonequals.dtx|
% \end{quote}
%
% \paragraph{TDS.} Now the different files must be moved into
% the different directories in your installation TDS tree
% (also known as \xfile{texmf} tree):
% \begin{quote}
% \def\t{^^A
% \begin{tabular}{@{}>{\ttfamily}l@{ $\rightarrow$ }>{\ttfamily}l@{}}
%   colonequals.sty & tex/latex/oberdiek/colonequals.sty\\
%   colonequals.pdf & doc/latex/oberdiek/colonequals.pdf\\
%   colonequals.dtx & source/latex/oberdiek/colonequals.dtx\\
% \end{tabular}^^A
% }^^A
% \sbox0{\t}^^A
% \ifdim\wd0>\linewidth
%   \begingroup
%     \advance\linewidth by\leftmargin
%     \advance\linewidth by\rightmargin
%   \edef\x{\endgroup
%     \def\noexpand\lw{\the\linewidth}^^A
%   }\x
%   \def\lwbox{^^A
%     \leavevmode
%     \hbox to \linewidth{^^A
%       \kern-\leftmargin\relax
%       \hss
%       \usebox0
%       \hss
%       \kern-\rightmargin\relax
%     }^^A
%   }^^A
%   \ifdim\wd0>\lw
%     \sbox0{\small\t}^^A
%     \ifdim\wd0>\linewidth
%       \ifdim\wd0>\lw
%         \sbox0{\footnotesize\t}^^A
%         \ifdim\wd0>\linewidth
%           \ifdim\wd0>\lw
%             \sbox0{\scriptsize\t}^^A
%             \ifdim\wd0>\linewidth
%               \ifdim\wd0>\lw
%                 \sbox0{\tiny\t}^^A
%                 \ifdim\wd0>\linewidth
%                   \lwbox
%                 \else
%                   \usebox0
%                 \fi
%               \else
%                 \lwbox
%               \fi
%             \else
%               \usebox0
%             \fi
%           \else
%             \lwbox
%           \fi
%         \else
%           \usebox0
%         \fi
%       \else
%         \lwbox
%       \fi
%     \else
%       \usebox0
%     \fi
%   \else
%     \lwbox
%   \fi
% \else
%   \usebox0
% \fi
% \end{quote}
% If you have a \xfile{docstrip.cfg} that configures and enables \docstrip's
% TDS installing feature, then some files can already be in the right
% place, see the documentation of \docstrip.
%
% \subsection{Refresh file name databases}
%
% If your \TeX~distribution
% (\teTeX, \mikTeX, \dots) relies on file name databases, you must refresh
% these. For example, \teTeX\ users run \verb|texhash| or
% \verb|mktexlsr|.
%
% \subsection{Some details for the interested}
%
% \paragraph{Attached source.}
%
% The PDF documentation on CTAN also includes the
% \xfile{.dtx} source file. It can be extracted by
% AcrobatReader 6 or higher. Another option is \textsf{pdftk},
% e.g. unpack the file into the current directory:
% \begin{quote}
%   \verb|pdftk colonequals.pdf unpack_files output .|
% \end{quote}
%
% \paragraph{Unpacking with \LaTeX.}
% The \xfile{.dtx} chooses its action depending on the format:
% \begin{description}
% \item[\plainTeX:] Run \docstrip\ and extract the files.
% \item[\LaTeX:] Generate the documentation.
% \end{description}
% If you insist on using \LaTeX\ for \docstrip\ (really,
% \docstrip\ does not need \LaTeX), then inform the autodetect routine
% about your intention:
% \begin{quote}
%   \verb|latex \let\install=y% \iffalse meta-comment
%
% File: colonequals.dtx
% Version: 2006/08/01 v1.0
% Info: Colon equals symbols
%
% Copyright (C) 2006 by
%    Heiko Oberdiek <heiko.oberdiek at googlemail.com>
%
% This work may be distributed and/or modified under the
% conditions of the LaTeX Project Public License, either
% version 1.3c of this license or (at your option) any later
% version. This version of this license is in
%    http://www.latex-project.org/lppl/lppl-1-3c.txt
% and the latest version of this license is in
%    http://www.latex-project.org/lppl.txt
% and version 1.3 or later is part of all distributions of
% LaTeX version 2005/12/01 or later.
%
% This work has the LPPL maintenance status "maintained".
%
% This Current Maintainer of this work is Heiko Oberdiek.
%
% This work consists of the main source file colonequals.dtx
% and the derived files
%    colonequals.sty, colonequals.pdf, colonequals.ins, colonequals.drv.
%
% Distribution:
%    CTAN:macros/latex/contrib/oberdiek/colonequals.dtx
%    CTAN:macros/latex/contrib/oberdiek/colonequals.pdf
%
% Unpacking:
%    (a) If colonequals.ins is present:
%           tex colonequals.ins
%    (b) Without colonequals.ins:
%           tex colonequals.dtx
%    (c) If you insist on using LaTeX
%           latex \let\install=y% \iffalse meta-comment
%
% File: colonequals.dtx
% Version: 2006/08/01 v1.0
% Info: Colon equals symbols
%
% Copyright (C) 2006 by
%    Heiko Oberdiek <heiko.oberdiek at googlemail.com>
%
% This work may be distributed and/or modified under the
% conditions of the LaTeX Project Public License, either
% version 1.3c of this license or (at your option) any later
% version. This version of this license is in
%    http://www.latex-project.org/lppl/lppl-1-3c.txt
% and the latest version of this license is in
%    http://www.latex-project.org/lppl.txt
% and version 1.3 or later is part of all distributions of
% LaTeX version 2005/12/01 or later.
%
% This work has the LPPL maintenance status "maintained".
%
% This Current Maintainer of this work is Heiko Oberdiek.
%
% This work consists of the main source file colonequals.dtx
% and the derived files
%    colonequals.sty, colonequals.pdf, colonequals.ins, colonequals.drv.
%
% Distribution:
%    CTAN:macros/latex/contrib/oberdiek/colonequals.dtx
%    CTAN:macros/latex/contrib/oberdiek/colonequals.pdf
%
% Unpacking:
%    (a) If colonequals.ins is present:
%           tex colonequals.ins
%    (b) Without colonequals.ins:
%           tex colonequals.dtx
%    (c) If you insist on using LaTeX
%           latex \let\install=y% \iffalse meta-comment
%
% File: colonequals.dtx
% Version: 2006/08/01 v1.0
% Info: Colon equals symbols
%
% Copyright (C) 2006 by
%    Heiko Oberdiek <heiko.oberdiek at googlemail.com>
%
% This work may be distributed and/or modified under the
% conditions of the LaTeX Project Public License, either
% version 1.3c of this license or (at your option) any later
% version. This version of this license is in
%    http://www.latex-project.org/lppl/lppl-1-3c.txt
% and the latest version of this license is in
%    http://www.latex-project.org/lppl.txt
% and version 1.3 or later is part of all distributions of
% LaTeX version 2005/12/01 or later.
%
% This work has the LPPL maintenance status "maintained".
%
% This Current Maintainer of this work is Heiko Oberdiek.
%
% This work consists of the main source file colonequals.dtx
% and the derived files
%    colonequals.sty, colonequals.pdf, colonequals.ins, colonequals.drv.
%
% Distribution:
%    CTAN:macros/latex/contrib/oberdiek/colonequals.dtx
%    CTAN:macros/latex/contrib/oberdiek/colonequals.pdf
%
% Unpacking:
%    (a) If colonequals.ins is present:
%           tex colonequals.ins
%    (b) Without colonequals.ins:
%           tex colonequals.dtx
%    (c) If you insist on using LaTeX
%           latex \let\install=y\input{colonequals.dtx}
%        (quote the arguments according to the demands of your shell)
%
% Documentation:
%    (a) If colonequals.drv is present:
%           latex colonequals.drv
%    (b) Without colonequals.drv:
%           latex colonequals.dtx; ...
%    The class ltxdoc loads the configuration file ltxdoc.cfg
%    if available. Here you can specify further options, e.g.
%    use A4 as paper format:
%       \PassOptionsToClass{a4paper}{article}
%
%    Programm calls to get the documentation (example):
%       pdflatex colonequals.dtx
%       makeindex -s gind.ist colonequals.idx
%       pdflatex colonequals.dtx
%       makeindex -s gind.ist colonequals.idx
%       pdflatex colonequals.dtx
%
% Installation:
%    TDS:tex/latex/oberdiek/colonequals.sty
%    TDS:doc/latex/oberdiek/colonequals.pdf
%    TDS:source/latex/oberdiek/colonequals.dtx
%
%<*ignore>
\begingroup
  \catcode123=1 %
  \catcode125=2 %
  \def\x{LaTeX2e}%
\expandafter\endgroup
\ifcase 0\ifx\install y1\fi\expandafter
         \ifx\csname processbatchFile\endcsname\relax\else1\fi
         \ifx\fmtname\x\else 1\fi\relax
\else\csname fi\endcsname
%</ignore>
%<*install>
\input docstrip.tex
\Msg{************************************************************************}
\Msg{* Installation}
\Msg{* Package: colonequals 2006/08/01 v1.0 Colon equals symbols (HO)}
\Msg{************************************************************************}

\keepsilent
\askforoverwritefalse

\let\MetaPrefix\relax
\preamble

This is a generated file.

Project: colonequals
Version: 2006/08/01 v1.0

Copyright (C) 2006 by
   Heiko Oberdiek <heiko.oberdiek at googlemail.com>

This work may be distributed and/or modified under the
conditions of the LaTeX Project Public License, either
version 1.3c of this license or (at your option) any later
version. This version of this license is in
   http://www.latex-project.org/lppl/lppl-1-3c.txt
and the latest version of this license is in
   http://www.latex-project.org/lppl.txt
and version 1.3 or later is part of all distributions of
LaTeX version 2005/12/01 or later.

This work has the LPPL maintenance status "maintained".

This Current Maintainer of this work is Heiko Oberdiek.

This work consists of the main source file colonequals.dtx
and the derived files
   colonequals.sty, colonequals.pdf, colonequals.ins, colonequals.drv.

\endpreamble
\let\MetaPrefix\DoubleperCent

\generate{%
  \file{colonequals.ins}{\from{colonequals.dtx}{install}}%
  \file{colonequals.drv}{\from{colonequals.dtx}{driver}}%
  \usedir{tex/latex/oberdiek}%
  \file{colonequals.sty}{\from{colonequals.dtx}{package}}%
  \nopreamble
  \nopostamble
  \usedir{source/latex/oberdiek/catalogue}%
  \file{colonequals.xml}{\from{colonequals.dtx}{catalogue}}%
}

\catcode32=13\relax% active space
\let =\space%
\Msg{************************************************************************}
\Msg{*}
\Msg{* To finish the installation you have to move the following}
\Msg{* file into a directory searched by TeX:}
\Msg{*}
\Msg{*     colonequals.sty}
\Msg{*}
\Msg{* To produce the documentation run the file `colonequals.drv'}
\Msg{* through LaTeX.}
\Msg{*}
\Msg{* Happy TeXing!}
\Msg{*}
\Msg{************************************************************************}

\endbatchfile
%</install>
%<*ignore>
\fi
%</ignore>
%<*driver>
\NeedsTeXFormat{LaTeX2e}
\ProvidesFile{colonequals.drv}%
  [2006/08/01 v1.0 Colon equals symbols (HO)]%
\documentclass{ltxdoc}
\usepackage{holtxdoc}[2011/11/22]
\usepackage{colonequals}
\usepackage{array}
\usepackage{capt-of}
\usepackage{longtable}
\begin{document}
  \DocInput{colonequals.dtx}%
\end{document}
%</driver>
% \fi
%
% \CheckSum{92}
%
% \CharacterTable
%  {Upper-case    \A\B\C\D\E\F\G\H\I\J\K\L\M\N\O\P\Q\R\S\T\U\V\W\X\Y\Z
%   Lower-case    \a\b\c\d\e\f\g\h\i\j\k\l\m\n\o\p\q\r\s\t\u\v\w\x\y\z
%   Digits        \0\1\2\3\4\5\6\7\8\9
%   Exclamation   \!     Double quote  \"     Hash (number) \#
%   Dollar        \$     Percent       \%     Ampersand     \&
%   Acute accent  \'     Left paren    \(     Right paren   \)
%   Asterisk      \*     Plus          \+     Comma         \,
%   Minus         \-     Point         \.     Solidus       \/
%   Colon         \:     Semicolon     \;     Less than     \<
%   Equals        \=     Greater than  \>     Question mark \?
%   Commercial at \@     Left bracket  \[     Backslash     \\
%   Right bracket \]     Circumflex    \^     Underscore    \_
%   Grave accent  \`     Left brace    \{     Vertical bar  \|
%   Right brace   \}     Tilde         \~}
%
% \GetFileInfo{colonequals.drv}
%
% \title{The \xpackage{colonequals} package}
% \date{2006/08/01 v1.0}
% \author{Heiko Oberdiek\\\xemail{heiko.oberdiek at googlemail.com}}
%
% \maketitle
%
% \begin{abstract}
% Package \xpackage{colonequals} defines poor man's symbols
% for math relation symbols such as ``colon equals''.
% The colon is centered around the horizontal math axis.
% \end{abstract}
%
% \tableofcontents
%
% \section{User interface}
%
% \subsection{Introduction}
%
% Math symbols consisting of the colon character can be
% constructed with the colon text character, if the math font
% lacks of the complete symbol. Often, however, the colon text
% character is not centered around the math axis. Especially
% combined with the equals symbol the composed symbol does not
% look symmetrically. Thus this packages defines a colon
% math symbol \cs{ratio} that is centered around the horizontal
% math axis. Also math symbols are provided that consist of the
% colon symbol. The package is not necessary, if the math
% font contains the composed symbols. Examples are \textsf{txfonts}
% (\cite{txfonts}) or \textsf{mathabx} (\cite{mathabx}).
%
% \subsection{Symbols}
%
% All symbols of this package are relation symbols.
% The relation property can be changed by the appropriate
% \TeX\ command \cs{mathbin}, \cs{mathord}, \dots
%
% \begin{center}
% \captionof{table}{Unicode mathematical operators}
% \kern1ex
% \begin{tabular}{l>{\scshape}l>{$}l<{$}l}
%   U+2236 & ratio        & \ratio       & \cs{ratio}       \\
%   U+2237 & proportion   & \coloncolon  & \cs{coloncolon}  \\
%   U+2239 & excess       & \colonminus  & \cs{colonminus}  \\
%   U+2254 & colon equals & \colonequals & \cs{colonequals} \\
%   U+2255 & equals colon & \equalscolon & \cs{equalscolon} \\
% \end{tabular}
% \end{center}
%
% The following grammar generates all symbols that are supported by
% this package:
% \begin{center}
% \captionof{table}{Symbol grammar}
% \kern1ex
% \begin{tabular}{@{}l>{$}r<{$}l@{}}
%   symbols & \coloncolonequals & col \\
%           & \mid & col symbol \\
%           & \mid & symbol col \\
%           & ; & \\[1ex]
%   col     & \coloncolonequals & '\texttt{:}' \\
%           & \mid & '\texttt{::}' \\
%           & ; & \\[1ex]
%   symbol  & \coloncolonequals & '\texttt{=}' \\
%           & \mid & '\texttt{-}' \\
%           & \mid & '$\approx$' \\
%           & \mid & '$\sim$' \\
%           & ; &
% \end{tabular}
% \end{center}
%
% \def\entry#1{\csname #1\endcsname&\cs{#1}\\}
% \def\entryset#1{^^A
%    \entry{colon#1}^^A
%    \entry{coloncolon#1}^^A
%    \entry{#1colon}^^A
%    \entry{#1coloncolon}^^A
% }
% \begin{longtable}{>{$}l<{$}l}
%   \caption{All relation symbols}\\
%   \entry{ratio}
%   \entry{coloncolon}
%   \entryset{equals}
%   \entryset{minus}
%   \entryset{approx}
%   \entryset{sim}
% \end{longtable}
%
% \subsection{Fine tuning}
%
% The distances in composed symbols can be configured:
%
% \begin{declcs}{colonsep}
% \end{declcs}
% Macro \cs{colonsep} is executed between the colon and
% the other symbol.
%
% \begin{declcs}{doublecolonsep}
% \end{declcs}
% Macro \cs{doublecolonsep} controls the distance between
% two colons.
%
% \subsubsection{Example}
% \begin{quote}
%   \verb|\renewcommand*{\colonsep}{\mskip-.5\thinmuskip}|
% \end{quote}
%
%
% \StopEventually{
% }
%
% \section{Implementation}
%
% \subsection{Identification}
%
%    \begin{macrocode}
%<*package>
\NeedsTeXFormat{LaTeX2e}
\ProvidesPackage{colonequals}%
  [2006/08/01 v1.0 Colon equals symbols (HO)]%
%    \end{macrocode}
%
% \subsection{Distance control}
%
%    \begin{macro}{\colonsep}
%    \begin{macrocode}
\newcommand*{\colonsep}{}
%    \end{macrocode}
%    \end{macro}
%
%    \begin{macro}{\doublecolonsep}
%    \begin{macrocode}
\newcommand*{\doublecolonsep}{}
%    \end{macrocode}
%    \end{macro}
%
% \subsection{Centered colons}
%
%    \begin{macrocode}
\def\@center@colon{%
  \mathpalette\@center@math{:}%
}
\def\@center@math#1#2{%
  \vcenter{%
    \m@th
    \hbox{$#1#2$}%
  }%
}
%    \end{macrocode}
%
%    \begin{macro}{\ratio}
%    Because the name \cs{colon} is already in use, the Unicode name
%    \cs{ratio} is used for the centered colon relation symbol.
%    (The \cs{ratio} of package \textsf{calc} is not used outside
%    calc expressions.)
%    \begin{macrocode}
\newcommand*{\ratio}{%
  \ensuremath{%
    \mathrel{%
      \@center@colon
    }%
  }%
}
%    \end{macrocode}
%    \end{macro}
%
%    \begin{macro}{\coloncolon}
%    \begin{macrocode}
\newcommand*{\coloncolon}{%
  \ensuremath{%
    \mathrel{%
      \@center@colon
      \doublecolonsep
      \@center@colon
    }%
  }%
}
%    \end{macrocode}
%    \end{macro}
%
% \subsection{Combined symbols}
%
%    \begin{macrocode}
\def\@make@colon@set#1#2{%
  \begingroup
    \let\@center@colon\relax
    \let\newcommand\relax
    \let\ensuremath\relax
    \let\mathrel\relax
    \let\colonsep\relax
    \let\doublecolonsep\relax
    \def\csx##1{%
      \expandafter\noexpand\csname ##1\endcsname
    }%
    \edef\x{\endgroup
      \newcommand*{\csx{colon#1}}{%
        \ensuremath{%
          \mathrel{%
            \@center@colon
            \colonsep
            {#2}%
          }%
        }%
      }%
      \newcommand*{\csx{coloncolon#1}}{%
        \ensuremath{%
          \mathrel{%
            \@center@colon
            \doublecolonsep
            \@center@colon
            \colonsep
            {#2}%
          }%
        }%
      }%
      \newcommand*{\csx{#1colon}}{%
        \ensuremath{%
          \mathrel{%
            {#2}%
            \colonsep
            \@center@colon
          }%
        }%
      }%
      \newcommand*{\csx{#1coloncolon}}{%
        \ensuremath{%
          \mathrel{%
            {#2}%
            \colonsep
            \@center@colon
            \doublecolonsep
            \@center@colon
          }%
        }%
      }%
    }%
  \x
}
%    \end{macrocode}
%
%    \begin{macrocode}
\@make@colon@set{equals}{=}%
\@make@colon@set{minus}{-}%
\@make@colon@set{approx}{\approx}
\@make@colon@set{sim}{\sim}
%    \end{macrocode}
%
%    \begin{macrocode}
%</package>
%    \end{macrocode}
%
% \section{Installation}
%
% \subsection{Download}
%
% \paragraph{Package.} This package is available on
% CTAN\footnote{\url{ftp://ftp.ctan.org/tex-archive/}}:
% \begin{description}
% \item[\CTAN{macros/latex/contrib/oberdiek/colonequals.dtx}] The source file.
% \item[\CTAN{macros/latex/contrib/oberdiek/colonequals.pdf}] Documentation.
% \end{description}
%
%
% \paragraph{Bundle.} All the packages of the bundle `oberdiek'
% are also available in a TDS compliant ZIP archive. There
% the packages are already unpacked and the documentation files
% are generated. The files and directories obey the TDS standard.
% \begin{description}
% \item[\CTAN{install/macros/latex/contrib/oberdiek.tds.zip}]
% \end{description}
% \emph{TDS} refers to the standard ``A Directory Structure
% for \TeX\ Files'' (\CTAN{tds/tds.pdf}). Directories
% with \xfile{texmf} in their name are usually organized this way.
%
% \subsection{Bundle installation}
%
% \paragraph{Unpacking.} Unpack the \xfile{oberdiek.tds.zip} in the
% TDS tree (also known as \xfile{texmf} tree) of your choice.
% Example (linux):
% \begin{quote}
%   |unzip oberdiek.tds.zip -d ~/texmf|
% \end{quote}
%
% \paragraph{Script installation.}
% Check the directory \xfile{TDS:scripts/oberdiek/} for
% scripts that need further installation steps.
% Package \xpackage{attachfile2} comes with the Perl script
% \xfile{pdfatfi.pl} that should be installed in such a way
% that it can be called as \texttt{pdfatfi}.
% Example (linux):
% \begin{quote}
%   |chmod +x scripts/oberdiek/pdfatfi.pl|\\
%   |cp scripts/oberdiek/pdfatfi.pl /usr/local/bin/|
% \end{quote}
%
% \subsection{Package installation}
%
% \paragraph{Unpacking.} The \xfile{.dtx} file is a self-extracting
% \docstrip\ archive. The files are extracted by running the
% \xfile{.dtx} through \plainTeX:
% \begin{quote}
%   \verb|tex colonequals.dtx|
% \end{quote}
%
% \paragraph{TDS.} Now the different files must be moved into
% the different directories in your installation TDS tree
% (also known as \xfile{texmf} tree):
% \begin{quote}
% \def\t{^^A
% \begin{tabular}{@{}>{\ttfamily}l@{ $\rightarrow$ }>{\ttfamily}l@{}}
%   colonequals.sty & tex/latex/oberdiek/colonequals.sty\\
%   colonequals.pdf & doc/latex/oberdiek/colonequals.pdf\\
%   colonequals.dtx & source/latex/oberdiek/colonequals.dtx\\
% \end{tabular}^^A
% }^^A
% \sbox0{\t}^^A
% \ifdim\wd0>\linewidth
%   \begingroup
%     \advance\linewidth by\leftmargin
%     \advance\linewidth by\rightmargin
%   \edef\x{\endgroup
%     \def\noexpand\lw{\the\linewidth}^^A
%   }\x
%   \def\lwbox{^^A
%     \leavevmode
%     \hbox to \linewidth{^^A
%       \kern-\leftmargin\relax
%       \hss
%       \usebox0
%       \hss
%       \kern-\rightmargin\relax
%     }^^A
%   }^^A
%   \ifdim\wd0>\lw
%     \sbox0{\small\t}^^A
%     \ifdim\wd0>\linewidth
%       \ifdim\wd0>\lw
%         \sbox0{\footnotesize\t}^^A
%         \ifdim\wd0>\linewidth
%           \ifdim\wd0>\lw
%             \sbox0{\scriptsize\t}^^A
%             \ifdim\wd0>\linewidth
%               \ifdim\wd0>\lw
%                 \sbox0{\tiny\t}^^A
%                 \ifdim\wd0>\linewidth
%                   \lwbox
%                 \else
%                   \usebox0
%                 \fi
%               \else
%                 \lwbox
%               \fi
%             \else
%               \usebox0
%             \fi
%           \else
%             \lwbox
%           \fi
%         \else
%           \usebox0
%         \fi
%       \else
%         \lwbox
%       \fi
%     \else
%       \usebox0
%     \fi
%   \else
%     \lwbox
%   \fi
% \else
%   \usebox0
% \fi
% \end{quote}
% If you have a \xfile{docstrip.cfg} that configures and enables \docstrip's
% TDS installing feature, then some files can already be in the right
% place, see the documentation of \docstrip.
%
% \subsection{Refresh file name databases}
%
% If your \TeX~distribution
% (\teTeX, \mikTeX, \dots) relies on file name databases, you must refresh
% these. For example, \teTeX\ users run \verb|texhash| or
% \verb|mktexlsr|.
%
% \subsection{Some details for the interested}
%
% \paragraph{Attached source.}
%
% The PDF documentation on CTAN also includes the
% \xfile{.dtx} source file. It can be extracted by
% AcrobatReader 6 or higher. Another option is \textsf{pdftk},
% e.g. unpack the file into the current directory:
% \begin{quote}
%   \verb|pdftk colonequals.pdf unpack_files output .|
% \end{quote}
%
% \paragraph{Unpacking with \LaTeX.}
% The \xfile{.dtx} chooses its action depending on the format:
% \begin{description}
% \item[\plainTeX:] Run \docstrip\ and extract the files.
% \item[\LaTeX:] Generate the documentation.
% \end{description}
% If you insist on using \LaTeX\ for \docstrip\ (really,
% \docstrip\ does not need \LaTeX), then inform the autodetect routine
% about your intention:
% \begin{quote}
%   \verb|latex \let\install=y\input{colonequals.dtx}|
% \end{quote}
% Do not forget to quote the argument according to the demands
% of your shell.
%
% \paragraph{Generating the documentation.}
% You can use both the \xfile{.dtx} or the \xfile{.drv} to generate
% the documentation. The process can be configured by the
% configuration file \xfile{ltxdoc.cfg}. For instance, put this
% line into this file, if you want to have A4 as paper format:
% \begin{quote}
%   \verb|\PassOptionsToClass{a4paper}{article}|
% \end{quote}
% An example follows how to generate the
% documentation with pdf\LaTeX:
% \begin{quote}
%\begin{verbatim}
%pdflatex colonequals.dtx
%makeindex -s gind.ist colonequals.idx
%pdflatex colonequals.dtx
%makeindex -s gind.ist colonequals.idx
%pdflatex colonequals.dtx
%\end{verbatim}
% \end{quote}
%
% \section{Catalogue}
%
% The following XML file can be used as source for the
% \href{http://mirror.ctan.org/help/Catalogue/catalogue.html}{\TeX\ Catalogue}.
% The elements \texttt{caption} and \texttt{description} are imported
% from the original XML file from the Catalogue.
% The name of the XML file in the Catalogue is \xfile{colonequals.xml}.
%    \begin{macrocode}
%<*catalogue>
<?xml version='1.0' encoding='us-ascii'?>
<!DOCTYPE entry SYSTEM 'catalogue.dtd'>
<entry datestamp='$Date$' modifier='$Author$' id='colonequals'>
  <name>colonequals</name>
  <caption>Colon equals symbols.</caption>
  <authorref id='auth:oberdiek'/>
  <copyright owner='Heiko Oberdiek' year='2006'/>
  <license type='lppl1.3'/>
  <version number='1.0'/>
  <description>
    This package defines poor man&#x2018;s symbols for mathematical
    relation symbols such as &#x201C;colon equals&#x201D;.
    The colon is centered around the horizontal math axis.
    <p/>
    The package is part of the <xref refid='oberdiek'>oberdiek</xref>
    bundle.
  </description>
  <documentation details='Package documentation'
      href='ctan:/macros/latex/contrib/oberdiek/colonequals.pdf'/>
  <ctan file='true' path='/macros/latex/contrib/oberdiek/colonequals.dtx'/>
  <miktex location='oberdiek'/>
  <texlive location='oberdiek'/>
  <install path='/macros/latex/contrib/oberdiek/oberdiek.tds.zip'/>
</entry>
%</catalogue>
%    \end{macrocode}
%
% \begin{thebibliography}{9}
%
% \bibitem{txfonts}
%   Young Ryu: \textit{The TX Fonts};
%   2000/12/15;
%   \CTAN{fonts/txfonts/}.
%
% \bibitem{mathabx}
%   Anthony Phan: \textit{Mathabx font series};
%   2005/05/16;
%   \CTAN{fonts/mathabx/}.
%
% \end{thebibliography}
%
% \begin{History}
%   \begin{Version}{2006/08/01 v1.0}
%   \item
%     First version.
%   \end{Version}
% \end{History}
%
% \PrintIndex
%
% \Finale
\endinput

%        (quote the arguments according to the demands of your shell)
%
% Documentation:
%    (a) If colonequals.drv is present:
%           latex colonequals.drv
%    (b) Without colonequals.drv:
%           latex colonequals.dtx; ...
%    The class ltxdoc loads the configuration file ltxdoc.cfg
%    if available. Here you can specify further options, e.g.
%    use A4 as paper format:
%       \PassOptionsToClass{a4paper}{article}
%
%    Programm calls to get the documentation (example):
%       pdflatex colonequals.dtx
%       makeindex -s gind.ist colonequals.idx
%       pdflatex colonequals.dtx
%       makeindex -s gind.ist colonequals.idx
%       pdflatex colonequals.dtx
%
% Installation:
%    TDS:tex/latex/oberdiek/colonequals.sty
%    TDS:doc/latex/oberdiek/colonequals.pdf
%    TDS:source/latex/oberdiek/colonequals.dtx
%
%<*ignore>
\begingroup
  \catcode123=1 %
  \catcode125=2 %
  \def\x{LaTeX2e}%
\expandafter\endgroup
\ifcase 0\ifx\install y1\fi\expandafter
         \ifx\csname processbatchFile\endcsname\relax\else1\fi
         \ifx\fmtname\x\else 1\fi\relax
\else\csname fi\endcsname
%</ignore>
%<*install>
\input docstrip.tex
\Msg{************************************************************************}
\Msg{* Installation}
\Msg{* Package: colonequals 2006/08/01 v1.0 Colon equals symbols (HO)}
\Msg{************************************************************************}

\keepsilent
\askforoverwritefalse

\let\MetaPrefix\relax
\preamble

This is a generated file.

Project: colonequals
Version: 2006/08/01 v1.0

Copyright (C) 2006 by
   Heiko Oberdiek <heiko.oberdiek at googlemail.com>

This work may be distributed and/or modified under the
conditions of the LaTeX Project Public License, either
version 1.3c of this license or (at your option) any later
version. This version of this license is in
   http://www.latex-project.org/lppl/lppl-1-3c.txt
and the latest version of this license is in
   http://www.latex-project.org/lppl.txt
and version 1.3 or later is part of all distributions of
LaTeX version 2005/12/01 or later.

This work has the LPPL maintenance status "maintained".

This Current Maintainer of this work is Heiko Oberdiek.

This work consists of the main source file colonequals.dtx
and the derived files
   colonequals.sty, colonequals.pdf, colonequals.ins, colonequals.drv.

\endpreamble
\let\MetaPrefix\DoubleperCent

\generate{%
  \file{colonequals.ins}{\from{colonequals.dtx}{install}}%
  \file{colonequals.drv}{\from{colonequals.dtx}{driver}}%
  \usedir{tex/latex/oberdiek}%
  \file{colonequals.sty}{\from{colonequals.dtx}{package}}%
  \nopreamble
  \nopostamble
  \usedir{source/latex/oberdiek/catalogue}%
  \file{colonequals.xml}{\from{colonequals.dtx}{catalogue}}%
}

\catcode32=13\relax% active space
\let =\space%
\Msg{************************************************************************}
\Msg{*}
\Msg{* To finish the installation you have to move the following}
\Msg{* file into a directory searched by TeX:}
\Msg{*}
\Msg{*     colonequals.sty}
\Msg{*}
\Msg{* To produce the documentation run the file `colonequals.drv'}
\Msg{* through LaTeX.}
\Msg{*}
\Msg{* Happy TeXing!}
\Msg{*}
\Msg{************************************************************************}

\endbatchfile
%</install>
%<*ignore>
\fi
%</ignore>
%<*driver>
\NeedsTeXFormat{LaTeX2e}
\ProvidesFile{colonequals.drv}%
  [2006/08/01 v1.0 Colon equals symbols (HO)]%
\documentclass{ltxdoc}
\usepackage{holtxdoc}[2011/11/22]
\usepackage{colonequals}
\usepackage{array}
\usepackage{capt-of}
\usepackage{longtable}
\begin{document}
  \DocInput{colonequals.dtx}%
\end{document}
%</driver>
% \fi
%
% \CheckSum{92}
%
% \CharacterTable
%  {Upper-case    \A\B\C\D\E\F\G\H\I\J\K\L\M\N\O\P\Q\R\S\T\U\V\W\X\Y\Z
%   Lower-case    \a\b\c\d\e\f\g\h\i\j\k\l\m\n\o\p\q\r\s\t\u\v\w\x\y\z
%   Digits        \0\1\2\3\4\5\6\7\8\9
%   Exclamation   \!     Double quote  \"     Hash (number) \#
%   Dollar        \$     Percent       \%     Ampersand     \&
%   Acute accent  \'     Left paren    \(     Right paren   \)
%   Asterisk      \*     Plus          \+     Comma         \,
%   Minus         \-     Point         \.     Solidus       \/
%   Colon         \:     Semicolon     \;     Less than     \<
%   Equals        \=     Greater than  \>     Question mark \?
%   Commercial at \@     Left bracket  \[     Backslash     \\
%   Right bracket \]     Circumflex    \^     Underscore    \_
%   Grave accent  \`     Left brace    \{     Vertical bar  \|
%   Right brace   \}     Tilde         \~}
%
% \GetFileInfo{colonequals.drv}
%
% \title{The \xpackage{colonequals} package}
% \date{2006/08/01 v1.0}
% \author{Heiko Oberdiek\\\xemail{heiko.oberdiek at googlemail.com}}
%
% \maketitle
%
% \begin{abstract}
% Package \xpackage{colonequals} defines poor man's symbols
% for math relation symbols such as ``colon equals''.
% The colon is centered around the horizontal math axis.
% \end{abstract}
%
% \tableofcontents
%
% \section{User interface}
%
% \subsection{Introduction}
%
% Math symbols consisting of the colon character can be
% constructed with the colon text character, if the math font
% lacks of the complete symbol. Often, however, the colon text
% character is not centered around the math axis. Especially
% combined with the equals symbol the composed symbol does not
% look symmetrically. Thus this packages defines a colon
% math symbol \cs{ratio} that is centered around the horizontal
% math axis. Also math symbols are provided that consist of the
% colon symbol. The package is not necessary, if the math
% font contains the composed symbols. Examples are \textsf{txfonts}
% (\cite{txfonts}) or \textsf{mathabx} (\cite{mathabx}).
%
% \subsection{Symbols}
%
% All symbols of this package are relation symbols.
% The relation property can be changed by the appropriate
% \TeX\ command \cs{mathbin}, \cs{mathord}, \dots
%
% \begin{center}
% \captionof{table}{Unicode mathematical operators}
% \kern1ex
% \begin{tabular}{l>{\scshape}l>{$}l<{$}l}
%   U+2236 & ratio        & \ratio       & \cs{ratio}       \\
%   U+2237 & proportion   & \coloncolon  & \cs{coloncolon}  \\
%   U+2239 & excess       & \colonminus  & \cs{colonminus}  \\
%   U+2254 & colon equals & \colonequals & \cs{colonequals} \\
%   U+2255 & equals colon & \equalscolon & \cs{equalscolon} \\
% \end{tabular}
% \end{center}
%
% The following grammar generates all symbols that are supported by
% this package:
% \begin{center}
% \captionof{table}{Symbol grammar}
% \kern1ex
% \begin{tabular}{@{}l>{$}r<{$}l@{}}
%   symbols & \coloncolonequals & col \\
%           & \mid & col symbol \\
%           & \mid & symbol col \\
%           & ; & \\[1ex]
%   col     & \coloncolonequals & '\texttt{:}' \\
%           & \mid & '\texttt{::}' \\
%           & ; & \\[1ex]
%   symbol  & \coloncolonequals & '\texttt{=}' \\
%           & \mid & '\texttt{-}' \\
%           & \mid & '$\approx$' \\
%           & \mid & '$\sim$' \\
%           & ; &
% \end{tabular}
% \end{center}
%
% \def\entry#1{\csname #1\endcsname&\cs{#1}\\}
% \def\entryset#1{^^A
%    \entry{colon#1}^^A
%    \entry{coloncolon#1}^^A
%    \entry{#1colon}^^A
%    \entry{#1coloncolon}^^A
% }
% \begin{longtable}{>{$}l<{$}l}
%   \caption{All relation symbols}\\
%   \entry{ratio}
%   \entry{coloncolon}
%   \entryset{equals}
%   \entryset{minus}
%   \entryset{approx}
%   \entryset{sim}
% \end{longtable}
%
% \subsection{Fine tuning}
%
% The distances in composed symbols can be configured:
%
% \begin{declcs}{colonsep}
% \end{declcs}
% Macro \cs{colonsep} is executed between the colon and
% the other symbol.
%
% \begin{declcs}{doublecolonsep}
% \end{declcs}
% Macro \cs{doublecolonsep} controls the distance between
% two colons.
%
% \subsubsection{Example}
% \begin{quote}
%   \verb|\renewcommand*{\colonsep}{\mskip-.5\thinmuskip}|
% \end{quote}
%
%
% \StopEventually{
% }
%
% \section{Implementation}
%
% \subsection{Identification}
%
%    \begin{macrocode}
%<*package>
\NeedsTeXFormat{LaTeX2e}
\ProvidesPackage{colonequals}%
  [2006/08/01 v1.0 Colon equals symbols (HO)]%
%    \end{macrocode}
%
% \subsection{Distance control}
%
%    \begin{macro}{\colonsep}
%    \begin{macrocode}
\newcommand*{\colonsep}{}
%    \end{macrocode}
%    \end{macro}
%
%    \begin{macro}{\doublecolonsep}
%    \begin{macrocode}
\newcommand*{\doublecolonsep}{}
%    \end{macrocode}
%    \end{macro}
%
% \subsection{Centered colons}
%
%    \begin{macrocode}
\def\@center@colon{%
  \mathpalette\@center@math{:}%
}
\def\@center@math#1#2{%
  \vcenter{%
    \m@th
    \hbox{$#1#2$}%
  }%
}
%    \end{macrocode}
%
%    \begin{macro}{\ratio}
%    Because the name \cs{colon} is already in use, the Unicode name
%    \cs{ratio} is used for the centered colon relation symbol.
%    (The \cs{ratio} of package \textsf{calc} is not used outside
%    calc expressions.)
%    \begin{macrocode}
\newcommand*{\ratio}{%
  \ensuremath{%
    \mathrel{%
      \@center@colon
    }%
  }%
}
%    \end{macrocode}
%    \end{macro}
%
%    \begin{macro}{\coloncolon}
%    \begin{macrocode}
\newcommand*{\coloncolon}{%
  \ensuremath{%
    \mathrel{%
      \@center@colon
      \doublecolonsep
      \@center@colon
    }%
  }%
}
%    \end{macrocode}
%    \end{macro}
%
% \subsection{Combined symbols}
%
%    \begin{macrocode}
\def\@make@colon@set#1#2{%
  \begingroup
    \let\@center@colon\relax
    \let\newcommand\relax
    \let\ensuremath\relax
    \let\mathrel\relax
    \let\colonsep\relax
    \let\doublecolonsep\relax
    \def\csx##1{%
      \expandafter\noexpand\csname ##1\endcsname
    }%
    \edef\x{\endgroup
      \newcommand*{\csx{colon#1}}{%
        \ensuremath{%
          \mathrel{%
            \@center@colon
            \colonsep
            {#2}%
          }%
        }%
      }%
      \newcommand*{\csx{coloncolon#1}}{%
        \ensuremath{%
          \mathrel{%
            \@center@colon
            \doublecolonsep
            \@center@colon
            \colonsep
            {#2}%
          }%
        }%
      }%
      \newcommand*{\csx{#1colon}}{%
        \ensuremath{%
          \mathrel{%
            {#2}%
            \colonsep
            \@center@colon
          }%
        }%
      }%
      \newcommand*{\csx{#1coloncolon}}{%
        \ensuremath{%
          \mathrel{%
            {#2}%
            \colonsep
            \@center@colon
            \doublecolonsep
            \@center@colon
          }%
        }%
      }%
    }%
  \x
}
%    \end{macrocode}
%
%    \begin{macrocode}
\@make@colon@set{equals}{=}%
\@make@colon@set{minus}{-}%
\@make@colon@set{approx}{\approx}
\@make@colon@set{sim}{\sim}
%    \end{macrocode}
%
%    \begin{macrocode}
%</package>
%    \end{macrocode}
%
% \section{Installation}
%
% \subsection{Download}
%
% \paragraph{Package.} This package is available on
% CTAN\footnote{\url{ftp://ftp.ctan.org/tex-archive/}}:
% \begin{description}
% \item[\CTAN{macros/latex/contrib/oberdiek/colonequals.dtx}] The source file.
% \item[\CTAN{macros/latex/contrib/oberdiek/colonequals.pdf}] Documentation.
% \end{description}
%
%
% \paragraph{Bundle.} All the packages of the bundle `oberdiek'
% are also available in a TDS compliant ZIP archive. There
% the packages are already unpacked and the documentation files
% are generated. The files and directories obey the TDS standard.
% \begin{description}
% \item[\CTAN{install/macros/latex/contrib/oberdiek.tds.zip}]
% \end{description}
% \emph{TDS} refers to the standard ``A Directory Structure
% for \TeX\ Files'' (\CTAN{tds/tds.pdf}). Directories
% with \xfile{texmf} in their name are usually organized this way.
%
% \subsection{Bundle installation}
%
% \paragraph{Unpacking.} Unpack the \xfile{oberdiek.tds.zip} in the
% TDS tree (also known as \xfile{texmf} tree) of your choice.
% Example (linux):
% \begin{quote}
%   |unzip oberdiek.tds.zip -d ~/texmf|
% \end{quote}
%
% \paragraph{Script installation.}
% Check the directory \xfile{TDS:scripts/oberdiek/} for
% scripts that need further installation steps.
% Package \xpackage{attachfile2} comes with the Perl script
% \xfile{pdfatfi.pl} that should be installed in such a way
% that it can be called as \texttt{pdfatfi}.
% Example (linux):
% \begin{quote}
%   |chmod +x scripts/oberdiek/pdfatfi.pl|\\
%   |cp scripts/oberdiek/pdfatfi.pl /usr/local/bin/|
% \end{quote}
%
% \subsection{Package installation}
%
% \paragraph{Unpacking.} The \xfile{.dtx} file is a self-extracting
% \docstrip\ archive. The files are extracted by running the
% \xfile{.dtx} through \plainTeX:
% \begin{quote}
%   \verb|tex colonequals.dtx|
% \end{quote}
%
% \paragraph{TDS.} Now the different files must be moved into
% the different directories in your installation TDS tree
% (also known as \xfile{texmf} tree):
% \begin{quote}
% \def\t{^^A
% \begin{tabular}{@{}>{\ttfamily}l@{ $\rightarrow$ }>{\ttfamily}l@{}}
%   colonequals.sty & tex/latex/oberdiek/colonequals.sty\\
%   colonequals.pdf & doc/latex/oberdiek/colonequals.pdf\\
%   colonequals.dtx & source/latex/oberdiek/colonequals.dtx\\
% \end{tabular}^^A
% }^^A
% \sbox0{\t}^^A
% \ifdim\wd0>\linewidth
%   \begingroup
%     \advance\linewidth by\leftmargin
%     \advance\linewidth by\rightmargin
%   \edef\x{\endgroup
%     \def\noexpand\lw{\the\linewidth}^^A
%   }\x
%   \def\lwbox{^^A
%     \leavevmode
%     \hbox to \linewidth{^^A
%       \kern-\leftmargin\relax
%       \hss
%       \usebox0
%       \hss
%       \kern-\rightmargin\relax
%     }^^A
%   }^^A
%   \ifdim\wd0>\lw
%     \sbox0{\small\t}^^A
%     \ifdim\wd0>\linewidth
%       \ifdim\wd0>\lw
%         \sbox0{\footnotesize\t}^^A
%         \ifdim\wd0>\linewidth
%           \ifdim\wd0>\lw
%             \sbox0{\scriptsize\t}^^A
%             \ifdim\wd0>\linewidth
%               \ifdim\wd0>\lw
%                 \sbox0{\tiny\t}^^A
%                 \ifdim\wd0>\linewidth
%                   \lwbox
%                 \else
%                   \usebox0
%                 \fi
%               \else
%                 \lwbox
%               \fi
%             \else
%               \usebox0
%             \fi
%           \else
%             \lwbox
%           \fi
%         \else
%           \usebox0
%         \fi
%       \else
%         \lwbox
%       \fi
%     \else
%       \usebox0
%     \fi
%   \else
%     \lwbox
%   \fi
% \else
%   \usebox0
% \fi
% \end{quote}
% If you have a \xfile{docstrip.cfg} that configures and enables \docstrip's
% TDS installing feature, then some files can already be in the right
% place, see the documentation of \docstrip.
%
% \subsection{Refresh file name databases}
%
% If your \TeX~distribution
% (\teTeX, \mikTeX, \dots) relies on file name databases, you must refresh
% these. For example, \teTeX\ users run \verb|texhash| or
% \verb|mktexlsr|.
%
% \subsection{Some details for the interested}
%
% \paragraph{Attached source.}
%
% The PDF documentation on CTAN also includes the
% \xfile{.dtx} source file. It can be extracted by
% AcrobatReader 6 or higher. Another option is \textsf{pdftk},
% e.g. unpack the file into the current directory:
% \begin{quote}
%   \verb|pdftk colonequals.pdf unpack_files output .|
% \end{quote}
%
% \paragraph{Unpacking with \LaTeX.}
% The \xfile{.dtx} chooses its action depending on the format:
% \begin{description}
% \item[\plainTeX:] Run \docstrip\ and extract the files.
% \item[\LaTeX:] Generate the documentation.
% \end{description}
% If you insist on using \LaTeX\ for \docstrip\ (really,
% \docstrip\ does not need \LaTeX), then inform the autodetect routine
% about your intention:
% \begin{quote}
%   \verb|latex \let\install=y% \iffalse meta-comment
%
% File: colonequals.dtx
% Version: 2006/08/01 v1.0
% Info: Colon equals symbols
%
% Copyright (C) 2006 by
%    Heiko Oberdiek <heiko.oberdiek at googlemail.com>
%
% This work may be distributed and/or modified under the
% conditions of the LaTeX Project Public License, either
% version 1.3c of this license or (at your option) any later
% version. This version of this license is in
%    http://www.latex-project.org/lppl/lppl-1-3c.txt
% and the latest version of this license is in
%    http://www.latex-project.org/lppl.txt
% and version 1.3 or later is part of all distributions of
% LaTeX version 2005/12/01 or later.
%
% This work has the LPPL maintenance status "maintained".
%
% This Current Maintainer of this work is Heiko Oberdiek.
%
% This work consists of the main source file colonequals.dtx
% and the derived files
%    colonequals.sty, colonequals.pdf, colonequals.ins, colonequals.drv.
%
% Distribution:
%    CTAN:macros/latex/contrib/oberdiek/colonequals.dtx
%    CTAN:macros/latex/contrib/oberdiek/colonequals.pdf
%
% Unpacking:
%    (a) If colonequals.ins is present:
%           tex colonequals.ins
%    (b) Without colonequals.ins:
%           tex colonequals.dtx
%    (c) If you insist on using LaTeX
%           latex \let\install=y\input{colonequals.dtx}
%        (quote the arguments according to the demands of your shell)
%
% Documentation:
%    (a) If colonequals.drv is present:
%           latex colonequals.drv
%    (b) Without colonequals.drv:
%           latex colonequals.dtx; ...
%    The class ltxdoc loads the configuration file ltxdoc.cfg
%    if available. Here you can specify further options, e.g.
%    use A4 as paper format:
%       \PassOptionsToClass{a4paper}{article}
%
%    Programm calls to get the documentation (example):
%       pdflatex colonequals.dtx
%       makeindex -s gind.ist colonequals.idx
%       pdflatex colonequals.dtx
%       makeindex -s gind.ist colonequals.idx
%       pdflatex colonequals.dtx
%
% Installation:
%    TDS:tex/latex/oberdiek/colonequals.sty
%    TDS:doc/latex/oberdiek/colonequals.pdf
%    TDS:source/latex/oberdiek/colonequals.dtx
%
%<*ignore>
\begingroup
  \catcode123=1 %
  \catcode125=2 %
  \def\x{LaTeX2e}%
\expandafter\endgroup
\ifcase 0\ifx\install y1\fi\expandafter
         \ifx\csname processbatchFile\endcsname\relax\else1\fi
         \ifx\fmtname\x\else 1\fi\relax
\else\csname fi\endcsname
%</ignore>
%<*install>
\input docstrip.tex
\Msg{************************************************************************}
\Msg{* Installation}
\Msg{* Package: colonequals 2006/08/01 v1.0 Colon equals symbols (HO)}
\Msg{************************************************************************}

\keepsilent
\askforoverwritefalse

\let\MetaPrefix\relax
\preamble

This is a generated file.

Project: colonequals
Version: 2006/08/01 v1.0

Copyright (C) 2006 by
   Heiko Oberdiek <heiko.oberdiek at googlemail.com>

This work may be distributed and/or modified under the
conditions of the LaTeX Project Public License, either
version 1.3c of this license or (at your option) any later
version. This version of this license is in
   http://www.latex-project.org/lppl/lppl-1-3c.txt
and the latest version of this license is in
   http://www.latex-project.org/lppl.txt
and version 1.3 or later is part of all distributions of
LaTeX version 2005/12/01 or later.

This work has the LPPL maintenance status "maintained".

This Current Maintainer of this work is Heiko Oberdiek.

This work consists of the main source file colonequals.dtx
and the derived files
   colonequals.sty, colonequals.pdf, colonequals.ins, colonequals.drv.

\endpreamble
\let\MetaPrefix\DoubleperCent

\generate{%
  \file{colonequals.ins}{\from{colonequals.dtx}{install}}%
  \file{colonequals.drv}{\from{colonequals.dtx}{driver}}%
  \usedir{tex/latex/oberdiek}%
  \file{colonequals.sty}{\from{colonequals.dtx}{package}}%
  \nopreamble
  \nopostamble
  \usedir{source/latex/oberdiek/catalogue}%
  \file{colonequals.xml}{\from{colonequals.dtx}{catalogue}}%
}

\catcode32=13\relax% active space
\let =\space%
\Msg{************************************************************************}
\Msg{*}
\Msg{* To finish the installation you have to move the following}
\Msg{* file into a directory searched by TeX:}
\Msg{*}
\Msg{*     colonequals.sty}
\Msg{*}
\Msg{* To produce the documentation run the file `colonequals.drv'}
\Msg{* through LaTeX.}
\Msg{*}
\Msg{* Happy TeXing!}
\Msg{*}
\Msg{************************************************************************}

\endbatchfile
%</install>
%<*ignore>
\fi
%</ignore>
%<*driver>
\NeedsTeXFormat{LaTeX2e}
\ProvidesFile{colonequals.drv}%
  [2006/08/01 v1.0 Colon equals symbols (HO)]%
\documentclass{ltxdoc}
\usepackage{holtxdoc}[2011/11/22]
\usepackage{colonequals}
\usepackage{array}
\usepackage{capt-of}
\usepackage{longtable}
\begin{document}
  \DocInput{colonequals.dtx}%
\end{document}
%</driver>
% \fi
%
% \CheckSum{92}
%
% \CharacterTable
%  {Upper-case    \A\B\C\D\E\F\G\H\I\J\K\L\M\N\O\P\Q\R\S\T\U\V\W\X\Y\Z
%   Lower-case    \a\b\c\d\e\f\g\h\i\j\k\l\m\n\o\p\q\r\s\t\u\v\w\x\y\z
%   Digits        \0\1\2\3\4\5\6\7\8\9
%   Exclamation   \!     Double quote  \"     Hash (number) \#
%   Dollar        \$     Percent       \%     Ampersand     \&
%   Acute accent  \'     Left paren    \(     Right paren   \)
%   Asterisk      \*     Plus          \+     Comma         \,
%   Minus         \-     Point         \.     Solidus       \/
%   Colon         \:     Semicolon     \;     Less than     \<
%   Equals        \=     Greater than  \>     Question mark \?
%   Commercial at \@     Left bracket  \[     Backslash     \\
%   Right bracket \]     Circumflex    \^     Underscore    \_
%   Grave accent  \`     Left brace    \{     Vertical bar  \|
%   Right brace   \}     Tilde         \~}
%
% \GetFileInfo{colonequals.drv}
%
% \title{The \xpackage{colonequals} package}
% \date{2006/08/01 v1.0}
% \author{Heiko Oberdiek\\\xemail{heiko.oberdiek at googlemail.com}}
%
% \maketitle
%
% \begin{abstract}
% Package \xpackage{colonequals} defines poor man's symbols
% for math relation symbols such as ``colon equals''.
% The colon is centered around the horizontal math axis.
% \end{abstract}
%
% \tableofcontents
%
% \section{User interface}
%
% \subsection{Introduction}
%
% Math symbols consisting of the colon character can be
% constructed with the colon text character, if the math font
% lacks of the complete symbol. Often, however, the colon text
% character is not centered around the math axis. Especially
% combined with the equals symbol the composed symbol does not
% look symmetrically. Thus this packages defines a colon
% math symbol \cs{ratio} that is centered around the horizontal
% math axis. Also math symbols are provided that consist of the
% colon symbol. The package is not necessary, if the math
% font contains the composed symbols. Examples are \textsf{txfonts}
% (\cite{txfonts}) or \textsf{mathabx} (\cite{mathabx}).
%
% \subsection{Symbols}
%
% All symbols of this package are relation symbols.
% The relation property can be changed by the appropriate
% \TeX\ command \cs{mathbin}, \cs{mathord}, \dots
%
% \begin{center}
% \captionof{table}{Unicode mathematical operators}
% \kern1ex
% \begin{tabular}{l>{\scshape}l>{$}l<{$}l}
%   U+2236 & ratio        & \ratio       & \cs{ratio}       \\
%   U+2237 & proportion   & \coloncolon  & \cs{coloncolon}  \\
%   U+2239 & excess       & \colonminus  & \cs{colonminus}  \\
%   U+2254 & colon equals & \colonequals & \cs{colonequals} \\
%   U+2255 & equals colon & \equalscolon & \cs{equalscolon} \\
% \end{tabular}
% \end{center}
%
% The following grammar generates all symbols that are supported by
% this package:
% \begin{center}
% \captionof{table}{Symbol grammar}
% \kern1ex
% \begin{tabular}{@{}l>{$}r<{$}l@{}}
%   symbols & \coloncolonequals & col \\
%           & \mid & col symbol \\
%           & \mid & symbol col \\
%           & ; & \\[1ex]
%   col     & \coloncolonequals & '\texttt{:}' \\
%           & \mid & '\texttt{::}' \\
%           & ; & \\[1ex]
%   symbol  & \coloncolonequals & '\texttt{=}' \\
%           & \mid & '\texttt{-}' \\
%           & \mid & '$\approx$' \\
%           & \mid & '$\sim$' \\
%           & ; &
% \end{tabular}
% \end{center}
%
% \def\entry#1{\csname #1\endcsname&\cs{#1}\\}
% \def\entryset#1{^^A
%    \entry{colon#1}^^A
%    \entry{coloncolon#1}^^A
%    \entry{#1colon}^^A
%    \entry{#1coloncolon}^^A
% }
% \begin{longtable}{>{$}l<{$}l}
%   \caption{All relation symbols}\\
%   \entry{ratio}
%   \entry{coloncolon}
%   \entryset{equals}
%   \entryset{minus}
%   \entryset{approx}
%   \entryset{sim}
% \end{longtable}
%
% \subsection{Fine tuning}
%
% The distances in composed symbols can be configured:
%
% \begin{declcs}{colonsep}
% \end{declcs}
% Macro \cs{colonsep} is executed between the colon and
% the other symbol.
%
% \begin{declcs}{doublecolonsep}
% \end{declcs}
% Macro \cs{doublecolonsep} controls the distance between
% two colons.
%
% \subsubsection{Example}
% \begin{quote}
%   \verb|\renewcommand*{\colonsep}{\mskip-.5\thinmuskip}|
% \end{quote}
%
%
% \StopEventually{
% }
%
% \section{Implementation}
%
% \subsection{Identification}
%
%    \begin{macrocode}
%<*package>
\NeedsTeXFormat{LaTeX2e}
\ProvidesPackage{colonequals}%
  [2006/08/01 v1.0 Colon equals symbols (HO)]%
%    \end{macrocode}
%
% \subsection{Distance control}
%
%    \begin{macro}{\colonsep}
%    \begin{macrocode}
\newcommand*{\colonsep}{}
%    \end{macrocode}
%    \end{macro}
%
%    \begin{macro}{\doublecolonsep}
%    \begin{macrocode}
\newcommand*{\doublecolonsep}{}
%    \end{macrocode}
%    \end{macro}
%
% \subsection{Centered colons}
%
%    \begin{macrocode}
\def\@center@colon{%
  \mathpalette\@center@math{:}%
}
\def\@center@math#1#2{%
  \vcenter{%
    \m@th
    \hbox{$#1#2$}%
  }%
}
%    \end{macrocode}
%
%    \begin{macro}{\ratio}
%    Because the name \cs{colon} is already in use, the Unicode name
%    \cs{ratio} is used for the centered colon relation symbol.
%    (The \cs{ratio} of package \textsf{calc} is not used outside
%    calc expressions.)
%    \begin{macrocode}
\newcommand*{\ratio}{%
  \ensuremath{%
    \mathrel{%
      \@center@colon
    }%
  }%
}
%    \end{macrocode}
%    \end{macro}
%
%    \begin{macro}{\coloncolon}
%    \begin{macrocode}
\newcommand*{\coloncolon}{%
  \ensuremath{%
    \mathrel{%
      \@center@colon
      \doublecolonsep
      \@center@colon
    }%
  }%
}
%    \end{macrocode}
%    \end{macro}
%
% \subsection{Combined symbols}
%
%    \begin{macrocode}
\def\@make@colon@set#1#2{%
  \begingroup
    \let\@center@colon\relax
    \let\newcommand\relax
    \let\ensuremath\relax
    \let\mathrel\relax
    \let\colonsep\relax
    \let\doublecolonsep\relax
    \def\csx##1{%
      \expandafter\noexpand\csname ##1\endcsname
    }%
    \edef\x{\endgroup
      \newcommand*{\csx{colon#1}}{%
        \ensuremath{%
          \mathrel{%
            \@center@colon
            \colonsep
            {#2}%
          }%
        }%
      }%
      \newcommand*{\csx{coloncolon#1}}{%
        \ensuremath{%
          \mathrel{%
            \@center@colon
            \doublecolonsep
            \@center@colon
            \colonsep
            {#2}%
          }%
        }%
      }%
      \newcommand*{\csx{#1colon}}{%
        \ensuremath{%
          \mathrel{%
            {#2}%
            \colonsep
            \@center@colon
          }%
        }%
      }%
      \newcommand*{\csx{#1coloncolon}}{%
        \ensuremath{%
          \mathrel{%
            {#2}%
            \colonsep
            \@center@colon
            \doublecolonsep
            \@center@colon
          }%
        }%
      }%
    }%
  \x
}
%    \end{macrocode}
%
%    \begin{macrocode}
\@make@colon@set{equals}{=}%
\@make@colon@set{minus}{-}%
\@make@colon@set{approx}{\approx}
\@make@colon@set{sim}{\sim}
%    \end{macrocode}
%
%    \begin{macrocode}
%</package>
%    \end{macrocode}
%
% \section{Installation}
%
% \subsection{Download}
%
% \paragraph{Package.} This package is available on
% CTAN\footnote{\url{ftp://ftp.ctan.org/tex-archive/}}:
% \begin{description}
% \item[\CTAN{macros/latex/contrib/oberdiek/colonequals.dtx}] The source file.
% \item[\CTAN{macros/latex/contrib/oberdiek/colonequals.pdf}] Documentation.
% \end{description}
%
%
% \paragraph{Bundle.} All the packages of the bundle `oberdiek'
% are also available in a TDS compliant ZIP archive. There
% the packages are already unpacked and the documentation files
% are generated. The files and directories obey the TDS standard.
% \begin{description}
% \item[\CTAN{install/macros/latex/contrib/oberdiek.tds.zip}]
% \end{description}
% \emph{TDS} refers to the standard ``A Directory Structure
% for \TeX\ Files'' (\CTAN{tds/tds.pdf}). Directories
% with \xfile{texmf} in their name are usually organized this way.
%
% \subsection{Bundle installation}
%
% \paragraph{Unpacking.} Unpack the \xfile{oberdiek.tds.zip} in the
% TDS tree (also known as \xfile{texmf} tree) of your choice.
% Example (linux):
% \begin{quote}
%   |unzip oberdiek.tds.zip -d ~/texmf|
% \end{quote}
%
% \paragraph{Script installation.}
% Check the directory \xfile{TDS:scripts/oberdiek/} for
% scripts that need further installation steps.
% Package \xpackage{attachfile2} comes with the Perl script
% \xfile{pdfatfi.pl} that should be installed in such a way
% that it can be called as \texttt{pdfatfi}.
% Example (linux):
% \begin{quote}
%   |chmod +x scripts/oberdiek/pdfatfi.pl|\\
%   |cp scripts/oberdiek/pdfatfi.pl /usr/local/bin/|
% \end{quote}
%
% \subsection{Package installation}
%
% \paragraph{Unpacking.} The \xfile{.dtx} file is a self-extracting
% \docstrip\ archive. The files are extracted by running the
% \xfile{.dtx} through \plainTeX:
% \begin{quote}
%   \verb|tex colonequals.dtx|
% \end{quote}
%
% \paragraph{TDS.} Now the different files must be moved into
% the different directories in your installation TDS tree
% (also known as \xfile{texmf} tree):
% \begin{quote}
% \def\t{^^A
% \begin{tabular}{@{}>{\ttfamily}l@{ $\rightarrow$ }>{\ttfamily}l@{}}
%   colonequals.sty & tex/latex/oberdiek/colonequals.sty\\
%   colonequals.pdf & doc/latex/oberdiek/colonequals.pdf\\
%   colonequals.dtx & source/latex/oberdiek/colonequals.dtx\\
% \end{tabular}^^A
% }^^A
% \sbox0{\t}^^A
% \ifdim\wd0>\linewidth
%   \begingroup
%     \advance\linewidth by\leftmargin
%     \advance\linewidth by\rightmargin
%   \edef\x{\endgroup
%     \def\noexpand\lw{\the\linewidth}^^A
%   }\x
%   \def\lwbox{^^A
%     \leavevmode
%     \hbox to \linewidth{^^A
%       \kern-\leftmargin\relax
%       \hss
%       \usebox0
%       \hss
%       \kern-\rightmargin\relax
%     }^^A
%   }^^A
%   \ifdim\wd0>\lw
%     \sbox0{\small\t}^^A
%     \ifdim\wd0>\linewidth
%       \ifdim\wd0>\lw
%         \sbox0{\footnotesize\t}^^A
%         \ifdim\wd0>\linewidth
%           \ifdim\wd0>\lw
%             \sbox0{\scriptsize\t}^^A
%             \ifdim\wd0>\linewidth
%               \ifdim\wd0>\lw
%                 \sbox0{\tiny\t}^^A
%                 \ifdim\wd0>\linewidth
%                   \lwbox
%                 \else
%                   \usebox0
%                 \fi
%               \else
%                 \lwbox
%               \fi
%             \else
%               \usebox0
%             \fi
%           \else
%             \lwbox
%           \fi
%         \else
%           \usebox0
%         \fi
%       \else
%         \lwbox
%       \fi
%     \else
%       \usebox0
%     \fi
%   \else
%     \lwbox
%   \fi
% \else
%   \usebox0
% \fi
% \end{quote}
% If you have a \xfile{docstrip.cfg} that configures and enables \docstrip's
% TDS installing feature, then some files can already be in the right
% place, see the documentation of \docstrip.
%
% \subsection{Refresh file name databases}
%
% If your \TeX~distribution
% (\teTeX, \mikTeX, \dots) relies on file name databases, you must refresh
% these. For example, \teTeX\ users run \verb|texhash| or
% \verb|mktexlsr|.
%
% \subsection{Some details for the interested}
%
% \paragraph{Attached source.}
%
% The PDF documentation on CTAN also includes the
% \xfile{.dtx} source file. It can be extracted by
% AcrobatReader 6 or higher. Another option is \textsf{pdftk},
% e.g. unpack the file into the current directory:
% \begin{quote}
%   \verb|pdftk colonequals.pdf unpack_files output .|
% \end{quote}
%
% \paragraph{Unpacking with \LaTeX.}
% The \xfile{.dtx} chooses its action depending on the format:
% \begin{description}
% \item[\plainTeX:] Run \docstrip\ and extract the files.
% \item[\LaTeX:] Generate the documentation.
% \end{description}
% If you insist on using \LaTeX\ for \docstrip\ (really,
% \docstrip\ does not need \LaTeX), then inform the autodetect routine
% about your intention:
% \begin{quote}
%   \verb|latex \let\install=y\input{colonequals.dtx}|
% \end{quote}
% Do not forget to quote the argument according to the demands
% of your shell.
%
% \paragraph{Generating the documentation.}
% You can use both the \xfile{.dtx} or the \xfile{.drv} to generate
% the documentation. The process can be configured by the
% configuration file \xfile{ltxdoc.cfg}. For instance, put this
% line into this file, if you want to have A4 as paper format:
% \begin{quote}
%   \verb|\PassOptionsToClass{a4paper}{article}|
% \end{quote}
% An example follows how to generate the
% documentation with pdf\LaTeX:
% \begin{quote}
%\begin{verbatim}
%pdflatex colonequals.dtx
%makeindex -s gind.ist colonequals.idx
%pdflatex colonequals.dtx
%makeindex -s gind.ist colonequals.idx
%pdflatex colonequals.dtx
%\end{verbatim}
% \end{quote}
%
% \section{Catalogue}
%
% The following XML file can be used as source for the
% \href{http://mirror.ctan.org/help/Catalogue/catalogue.html}{\TeX\ Catalogue}.
% The elements \texttt{caption} and \texttt{description} are imported
% from the original XML file from the Catalogue.
% The name of the XML file in the Catalogue is \xfile{colonequals.xml}.
%    \begin{macrocode}
%<*catalogue>
<?xml version='1.0' encoding='us-ascii'?>
<!DOCTYPE entry SYSTEM 'catalogue.dtd'>
<entry datestamp='$Date$' modifier='$Author$' id='colonequals'>
  <name>colonequals</name>
  <caption>Colon equals symbols.</caption>
  <authorref id='auth:oberdiek'/>
  <copyright owner='Heiko Oberdiek' year='2006'/>
  <license type='lppl1.3'/>
  <version number='1.0'/>
  <description>
    This package defines poor man&#x2018;s symbols for mathematical
    relation symbols such as &#x201C;colon equals&#x201D;.
    The colon is centered around the horizontal math axis.
    <p/>
    The package is part of the <xref refid='oberdiek'>oberdiek</xref>
    bundle.
  </description>
  <documentation details='Package documentation'
      href='ctan:/macros/latex/contrib/oberdiek/colonequals.pdf'/>
  <ctan file='true' path='/macros/latex/contrib/oberdiek/colonequals.dtx'/>
  <miktex location='oberdiek'/>
  <texlive location='oberdiek'/>
  <install path='/macros/latex/contrib/oberdiek/oberdiek.tds.zip'/>
</entry>
%</catalogue>
%    \end{macrocode}
%
% \begin{thebibliography}{9}
%
% \bibitem{txfonts}
%   Young Ryu: \textit{The TX Fonts};
%   2000/12/15;
%   \CTAN{fonts/txfonts/}.
%
% \bibitem{mathabx}
%   Anthony Phan: \textit{Mathabx font series};
%   2005/05/16;
%   \CTAN{fonts/mathabx/}.
%
% \end{thebibliography}
%
% \begin{History}
%   \begin{Version}{2006/08/01 v1.0}
%   \item
%     First version.
%   \end{Version}
% \end{History}
%
% \PrintIndex
%
% \Finale
\endinput
|
% \end{quote}
% Do not forget to quote the argument according to the demands
% of your shell.
%
% \paragraph{Generating the documentation.}
% You can use both the \xfile{.dtx} or the \xfile{.drv} to generate
% the documentation. The process can be configured by the
% configuration file \xfile{ltxdoc.cfg}. For instance, put this
% line into this file, if you want to have A4 as paper format:
% \begin{quote}
%   \verb|\PassOptionsToClass{a4paper}{article}|
% \end{quote}
% An example follows how to generate the
% documentation with pdf\LaTeX:
% \begin{quote}
%\begin{verbatim}
%pdflatex colonequals.dtx
%makeindex -s gind.ist colonequals.idx
%pdflatex colonequals.dtx
%makeindex -s gind.ist colonequals.idx
%pdflatex colonequals.dtx
%\end{verbatim}
% \end{quote}
%
% \section{Catalogue}
%
% The following XML file can be used as source for the
% \href{http://mirror.ctan.org/help/Catalogue/catalogue.html}{\TeX\ Catalogue}.
% The elements \texttt{caption} and \texttt{description} are imported
% from the original XML file from the Catalogue.
% The name of the XML file in the Catalogue is \xfile{colonequals.xml}.
%    \begin{macrocode}
%<*catalogue>
<?xml version='1.0' encoding='us-ascii'?>
<!DOCTYPE entry SYSTEM 'catalogue.dtd'>
<entry datestamp='$Date$' modifier='$Author$' id='colonequals'>
  <name>colonequals</name>
  <caption>Colon equals symbols.</caption>
  <authorref id='auth:oberdiek'/>
  <copyright owner='Heiko Oberdiek' year='2006'/>
  <license type='lppl1.3'/>
  <version number='1.0'/>
  <description>
    This package defines poor man&#x2018;s symbols for mathematical
    relation symbols such as &#x201C;colon equals&#x201D;.
    The colon is centered around the horizontal math axis.
    <p/>
    The package is part of the <xref refid='oberdiek'>oberdiek</xref>
    bundle.
  </description>
  <documentation details='Package documentation'
      href='ctan:/macros/latex/contrib/oberdiek/colonequals.pdf'/>
  <ctan file='true' path='/macros/latex/contrib/oberdiek/colonequals.dtx'/>
  <miktex location='oberdiek'/>
  <texlive location='oberdiek'/>
  <install path='/macros/latex/contrib/oberdiek/oberdiek.tds.zip'/>
</entry>
%</catalogue>
%    \end{macrocode}
%
% \begin{thebibliography}{9}
%
% \bibitem{txfonts}
%   Young Ryu: \textit{The TX Fonts};
%   2000/12/15;
%   \CTAN{fonts/txfonts/}.
%
% \bibitem{mathabx}
%   Anthony Phan: \textit{Mathabx font series};
%   2005/05/16;
%   \CTAN{fonts/mathabx/}.
%
% \end{thebibliography}
%
% \begin{History}
%   \begin{Version}{2006/08/01 v1.0}
%   \item
%     First version.
%   \end{Version}
% \end{History}
%
% \PrintIndex
%
% \Finale
\endinput

%        (quote the arguments according to the demands of your shell)
%
% Documentation:
%    (a) If colonequals.drv is present:
%           latex colonequals.drv
%    (b) Without colonequals.drv:
%           latex colonequals.dtx; ...
%    The class ltxdoc loads the configuration file ltxdoc.cfg
%    if available. Here you can specify further options, e.g.
%    use A4 as paper format:
%       \PassOptionsToClass{a4paper}{article}
%
%    Programm calls to get the documentation (example):
%       pdflatex colonequals.dtx
%       makeindex -s gind.ist colonequals.idx
%       pdflatex colonequals.dtx
%       makeindex -s gind.ist colonequals.idx
%       pdflatex colonequals.dtx
%
% Installation:
%    TDS:tex/latex/oberdiek/colonequals.sty
%    TDS:doc/latex/oberdiek/colonequals.pdf
%    TDS:source/latex/oberdiek/colonequals.dtx
%
%<*ignore>
\begingroup
  \catcode123=1 %
  \catcode125=2 %
  \def\x{LaTeX2e}%
\expandafter\endgroup
\ifcase 0\ifx\install y1\fi\expandafter
         \ifx\csname processbatchFile\endcsname\relax\else1\fi
         \ifx\fmtname\x\else 1\fi\relax
\else\csname fi\endcsname
%</ignore>
%<*install>
\input docstrip.tex
\Msg{************************************************************************}
\Msg{* Installation}
\Msg{* Package: colonequals 2006/08/01 v1.0 Colon equals symbols (HO)}
\Msg{************************************************************************}

\keepsilent
\askforoverwritefalse

\let\MetaPrefix\relax
\preamble

This is a generated file.

Project: colonequals
Version: 2006/08/01 v1.0

Copyright (C) 2006 by
   Heiko Oberdiek <heiko.oberdiek at googlemail.com>

This work may be distributed and/or modified under the
conditions of the LaTeX Project Public License, either
version 1.3c of this license or (at your option) any later
version. This version of this license is in
   http://www.latex-project.org/lppl/lppl-1-3c.txt
and the latest version of this license is in
   http://www.latex-project.org/lppl.txt
and version 1.3 or later is part of all distributions of
LaTeX version 2005/12/01 or later.

This work has the LPPL maintenance status "maintained".

This Current Maintainer of this work is Heiko Oberdiek.

This work consists of the main source file colonequals.dtx
and the derived files
   colonequals.sty, colonequals.pdf, colonequals.ins, colonequals.drv.

\endpreamble
\let\MetaPrefix\DoubleperCent

\generate{%
  \file{colonequals.ins}{\from{colonequals.dtx}{install}}%
  \file{colonequals.drv}{\from{colonequals.dtx}{driver}}%
  \usedir{tex/latex/oberdiek}%
  \file{colonequals.sty}{\from{colonequals.dtx}{package}}%
  \nopreamble
  \nopostamble
  \usedir{source/latex/oberdiek/catalogue}%
  \file{colonequals.xml}{\from{colonequals.dtx}{catalogue}}%
}

\catcode32=13\relax% active space
\let =\space%
\Msg{************************************************************************}
\Msg{*}
\Msg{* To finish the installation you have to move the following}
\Msg{* file into a directory searched by TeX:}
\Msg{*}
\Msg{*     colonequals.sty}
\Msg{*}
\Msg{* To produce the documentation run the file `colonequals.drv'}
\Msg{* through LaTeX.}
\Msg{*}
\Msg{* Happy TeXing!}
\Msg{*}
\Msg{************************************************************************}

\endbatchfile
%</install>
%<*ignore>
\fi
%</ignore>
%<*driver>
\NeedsTeXFormat{LaTeX2e}
\ProvidesFile{colonequals.drv}%
  [2006/08/01 v1.0 Colon equals symbols (HO)]%
\documentclass{ltxdoc}
\usepackage{holtxdoc}[2011/11/22]
\usepackage{colonequals}
\usepackage{array}
\usepackage{capt-of}
\usepackage{longtable}
\begin{document}
  \DocInput{colonequals.dtx}%
\end{document}
%</driver>
% \fi
%
% \CheckSum{92}
%
% \CharacterTable
%  {Upper-case    \A\B\C\D\E\F\G\H\I\J\K\L\M\N\O\P\Q\R\S\T\U\V\W\X\Y\Z
%   Lower-case    \a\b\c\d\e\f\g\h\i\j\k\l\m\n\o\p\q\r\s\t\u\v\w\x\y\z
%   Digits        \0\1\2\3\4\5\6\7\8\9
%   Exclamation   \!     Double quote  \"     Hash (number) \#
%   Dollar        \$     Percent       \%     Ampersand     \&
%   Acute accent  \'     Left paren    \(     Right paren   \)
%   Asterisk      \*     Plus          \+     Comma         \,
%   Minus         \-     Point         \.     Solidus       \/
%   Colon         \:     Semicolon     \;     Less than     \<
%   Equals        \=     Greater than  \>     Question mark \?
%   Commercial at \@     Left bracket  \[     Backslash     \\
%   Right bracket \]     Circumflex    \^     Underscore    \_
%   Grave accent  \`     Left brace    \{     Vertical bar  \|
%   Right brace   \}     Tilde         \~}
%
% \GetFileInfo{colonequals.drv}
%
% \title{The \xpackage{colonequals} package}
% \date{2006/08/01 v1.0}
% \author{Heiko Oberdiek\\\xemail{heiko.oberdiek at googlemail.com}}
%
% \maketitle
%
% \begin{abstract}
% Package \xpackage{colonequals} defines poor man's symbols
% for math relation symbols such as ``colon equals''.
% The colon is centered around the horizontal math axis.
% \end{abstract}
%
% \tableofcontents
%
% \section{User interface}
%
% \subsection{Introduction}
%
% Math symbols consisting of the colon character can be
% constructed with the colon text character, if the math font
% lacks of the complete symbol. Often, however, the colon text
% character is not centered around the math axis. Especially
% combined with the equals symbol the composed symbol does not
% look symmetrically. Thus this packages defines a colon
% math symbol \cs{ratio} that is centered around the horizontal
% math axis. Also math symbols are provided that consist of the
% colon symbol. The package is not necessary, if the math
% font contains the composed symbols. Examples are \textsf{txfonts}
% (\cite{txfonts}) or \textsf{mathabx} (\cite{mathabx}).
%
% \subsection{Symbols}
%
% All symbols of this package are relation symbols.
% The relation property can be changed by the appropriate
% \TeX\ command \cs{mathbin}, \cs{mathord}, \dots
%
% \begin{center}
% \captionof{table}{Unicode mathematical operators}
% \kern1ex
% \begin{tabular}{l>{\scshape}l>{$}l<{$}l}
%   U+2236 & ratio        & \ratio       & \cs{ratio}       \\
%   U+2237 & proportion   & \coloncolon  & \cs{coloncolon}  \\
%   U+2239 & excess       & \colonminus  & \cs{colonminus}  \\
%   U+2254 & colon equals & \colonequals & \cs{colonequals} \\
%   U+2255 & equals colon & \equalscolon & \cs{equalscolon} \\
% \end{tabular}
% \end{center}
%
% The following grammar generates all symbols that are supported by
% this package:
% \begin{center}
% \captionof{table}{Symbol grammar}
% \kern1ex
% \begin{tabular}{@{}l>{$}r<{$}l@{}}
%   symbols & \coloncolonequals & col \\
%           & \mid & col symbol \\
%           & \mid & symbol col \\
%           & ; & \\[1ex]
%   col     & \coloncolonequals & '\texttt{:}' \\
%           & \mid & '\texttt{::}' \\
%           & ; & \\[1ex]
%   symbol  & \coloncolonequals & '\texttt{=}' \\
%           & \mid & '\texttt{-}' \\
%           & \mid & '$\approx$' \\
%           & \mid & '$\sim$' \\
%           & ; &
% \end{tabular}
% \end{center}
%
% \def\entry#1{\csname #1\endcsname&\cs{#1}\\}
% \def\entryset#1{^^A
%    \entry{colon#1}^^A
%    \entry{coloncolon#1}^^A
%    \entry{#1colon}^^A
%    \entry{#1coloncolon}^^A
% }
% \begin{longtable}{>{$}l<{$}l}
%   \caption{All relation symbols}\\
%   \entry{ratio}
%   \entry{coloncolon}
%   \entryset{equals}
%   \entryset{minus}
%   \entryset{approx}
%   \entryset{sim}
% \end{longtable}
%
% \subsection{Fine tuning}
%
% The distances in composed symbols can be configured:
%
% \begin{declcs}{colonsep}
% \end{declcs}
% Macro \cs{colonsep} is executed between the colon and
% the other symbol.
%
% \begin{declcs}{doublecolonsep}
% \end{declcs}
% Macro \cs{doublecolonsep} controls the distance between
% two colons.
%
% \subsubsection{Example}
% \begin{quote}
%   \verb|\renewcommand*{\colonsep}{\mskip-.5\thinmuskip}|
% \end{quote}
%
%
% \StopEventually{
% }
%
% \section{Implementation}
%
% \subsection{Identification}
%
%    \begin{macrocode}
%<*package>
\NeedsTeXFormat{LaTeX2e}
\ProvidesPackage{colonequals}%
  [2006/08/01 v1.0 Colon equals symbols (HO)]%
%    \end{macrocode}
%
% \subsection{Distance control}
%
%    \begin{macro}{\colonsep}
%    \begin{macrocode}
\newcommand*{\colonsep}{}
%    \end{macrocode}
%    \end{macro}
%
%    \begin{macro}{\doublecolonsep}
%    \begin{macrocode}
\newcommand*{\doublecolonsep}{}
%    \end{macrocode}
%    \end{macro}
%
% \subsection{Centered colons}
%
%    \begin{macrocode}
\def\@center@colon{%
  \mathpalette\@center@math{:}%
}
\def\@center@math#1#2{%
  \vcenter{%
    \m@th
    \hbox{$#1#2$}%
  }%
}
%    \end{macrocode}
%
%    \begin{macro}{\ratio}
%    Because the name \cs{colon} is already in use, the Unicode name
%    \cs{ratio} is used for the centered colon relation symbol.
%    (The \cs{ratio} of package \textsf{calc} is not used outside
%    calc expressions.)
%    \begin{macrocode}
\newcommand*{\ratio}{%
  \ensuremath{%
    \mathrel{%
      \@center@colon
    }%
  }%
}
%    \end{macrocode}
%    \end{macro}
%
%    \begin{macro}{\coloncolon}
%    \begin{macrocode}
\newcommand*{\coloncolon}{%
  \ensuremath{%
    \mathrel{%
      \@center@colon
      \doublecolonsep
      \@center@colon
    }%
  }%
}
%    \end{macrocode}
%    \end{macro}
%
% \subsection{Combined symbols}
%
%    \begin{macrocode}
\def\@make@colon@set#1#2{%
  \begingroup
    \let\@center@colon\relax
    \let\newcommand\relax
    \let\ensuremath\relax
    \let\mathrel\relax
    \let\colonsep\relax
    \let\doublecolonsep\relax
    \def\csx##1{%
      \expandafter\noexpand\csname ##1\endcsname
    }%
    \edef\x{\endgroup
      \newcommand*{\csx{colon#1}}{%
        \ensuremath{%
          \mathrel{%
            \@center@colon
            \colonsep
            {#2}%
          }%
        }%
      }%
      \newcommand*{\csx{coloncolon#1}}{%
        \ensuremath{%
          \mathrel{%
            \@center@colon
            \doublecolonsep
            \@center@colon
            \colonsep
            {#2}%
          }%
        }%
      }%
      \newcommand*{\csx{#1colon}}{%
        \ensuremath{%
          \mathrel{%
            {#2}%
            \colonsep
            \@center@colon
          }%
        }%
      }%
      \newcommand*{\csx{#1coloncolon}}{%
        \ensuremath{%
          \mathrel{%
            {#2}%
            \colonsep
            \@center@colon
            \doublecolonsep
            \@center@colon
          }%
        }%
      }%
    }%
  \x
}
%    \end{macrocode}
%
%    \begin{macrocode}
\@make@colon@set{equals}{=}%
\@make@colon@set{minus}{-}%
\@make@colon@set{approx}{\approx}
\@make@colon@set{sim}{\sim}
%    \end{macrocode}
%
%    \begin{macrocode}
%</package>
%    \end{macrocode}
%
% \section{Installation}
%
% \subsection{Download}
%
% \paragraph{Package.} This package is available on
% CTAN\footnote{\url{ftp://ftp.ctan.org/tex-archive/}}:
% \begin{description}
% \item[\CTAN{macros/latex/contrib/oberdiek/colonequals.dtx}] The source file.
% \item[\CTAN{macros/latex/contrib/oberdiek/colonequals.pdf}] Documentation.
% \end{description}
%
%
% \paragraph{Bundle.} All the packages of the bundle `oberdiek'
% are also available in a TDS compliant ZIP archive. There
% the packages are already unpacked and the documentation files
% are generated. The files and directories obey the TDS standard.
% \begin{description}
% \item[\CTAN{install/macros/latex/contrib/oberdiek.tds.zip}]
% \end{description}
% \emph{TDS} refers to the standard ``A Directory Structure
% for \TeX\ Files'' (\CTAN{tds/tds.pdf}). Directories
% with \xfile{texmf} in their name are usually organized this way.
%
% \subsection{Bundle installation}
%
% \paragraph{Unpacking.} Unpack the \xfile{oberdiek.tds.zip} in the
% TDS tree (also known as \xfile{texmf} tree) of your choice.
% Example (linux):
% \begin{quote}
%   |unzip oberdiek.tds.zip -d ~/texmf|
% \end{quote}
%
% \paragraph{Script installation.}
% Check the directory \xfile{TDS:scripts/oberdiek/} for
% scripts that need further installation steps.
% Package \xpackage{attachfile2} comes with the Perl script
% \xfile{pdfatfi.pl} that should be installed in such a way
% that it can be called as \texttt{pdfatfi}.
% Example (linux):
% \begin{quote}
%   |chmod +x scripts/oberdiek/pdfatfi.pl|\\
%   |cp scripts/oberdiek/pdfatfi.pl /usr/local/bin/|
% \end{quote}
%
% \subsection{Package installation}
%
% \paragraph{Unpacking.} The \xfile{.dtx} file is a self-extracting
% \docstrip\ archive. The files are extracted by running the
% \xfile{.dtx} through \plainTeX:
% \begin{quote}
%   \verb|tex colonequals.dtx|
% \end{quote}
%
% \paragraph{TDS.} Now the different files must be moved into
% the different directories in your installation TDS tree
% (also known as \xfile{texmf} tree):
% \begin{quote}
% \def\t{^^A
% \begin{tabular}{@{}>{\ttfamily}l@{ $\rightarrow$ }>{\ttfamily}l@{}}
%   colonequals.sty & tex/latex/oberdiek/colonequals.sty\\
%   colonequals.pdf & doc/latex/oberdiek/colonequals.pdf\\
%   colonequals.dtx & source/latex/oberdiek/colonequals.dtx\\
% \end{tabular}^^A
% }^^A
% \sbox0{\t}^^A
% \ifdim\wd0>\linewidth
%   \begingroup
%     \advance\linewidth by\leftmargin
%     \advance\linewidth by\rightmargin
%   \edef\x{\endgroup
%     \def\noexpand\lw{\the\linewidth}^^A
%   }\x
%   \def\lwbox{^^A
%     \leavevmode
%     \hbox to \linewidth{^^A
%       \kern-\leftmargin\relax
%       \hss
%       \usebox0
%       \hss
%       \kern-\rightmargin\relax
%     }^^A
%   }^^A
%   \ifdim\wd0>\lw
%     \sbox0{\small\t}^^A
%     \ifdim\wd0>\linewidth
%       \ifdim\wd0>\lw
%         \sbox0{\footnotesize\t}^^A
%         \ifdim\wd0>\linewidth
%           \ifdim\wd0>\lw
%             \sbox0{\scriptsize\t}^^A
%             \ifdim\wd0>\linewidth
%               \ifdim\wd0>\lw
%                 \sbox0{\tiny\t}^^A
%                 \ifdim\wd0>\linewidth
%                   \lwbox
%                 \else
%                   \usebox0
%                 \fi
%               \else
%                 \lwbox
%               \fi
%             \else
%               \usebox0
%             \fi
%           \else
%             \lwbox
%           \fi
%         \else
%           \usebox0
%         \fi
%       \else
%         \lwbox
%       \fi
%     \else
%       \usebox0
%     \fi
%   \else
%     \lwbox
%   \fi
% \else
%   \usebox0
% \fi
% \end{quote}
% If you have a \xfile{docstrip.cfg} that configures and enables \docstrip's
% TDS installing feature, then some files can already be in the right
% place, see the documentation of \docstrip.
%
% \subsection{Refresh file name databases}
%
% If your \TeX~distribution
% (\teTeX, \mikTeX, \dots) relies on file name databases, you must refresh
% these. For example, \teTeX\ users run \verb|texhash| or
% \verb|mktexlsr|.
%
% \subsection{Some details for the interested}
%
% \paragraph{Attached source.}
%
% The PDF documentation on CTAN also includes the
% \xfile{.dtx} source file. It can be extracted by
% AcrobatReader 6 or higher. Another option is \textsf{pdftk},
% e.g. unpack the file into the current directory:
% \begin{quote}
%   \verb|pdftk colonequals.pdf unpack_files output .|
% \end{quote}
%
% \paragraph{Unpacking with \LaTeX.}
% The \xfile{.dtx} chooses its action depending on the format:
% \begin{description}
% \item[\plainTeX:] Run \docstrip\ and extract the files.
% \item[\LaTeX:] Generate the documentation.
% \end{description}
% If you insist on using \LaTeX\ for \docstrip\ (really,
% \docstrip\ does not need \LaTeX), then inform the autodetect routine
% about your intention:
% \begin{quote}
%   \verb|latex \let\install=y% \iffalse meta-comment
%
% File: colonequals.dtx
% Version: 2006/08/01 v1.0
% Info: Colon equals symbols
%
% Copyright (C) 2006 by
%    Heiko Oberdiek <heiko.oberdiek at googlemail.com>
%
% This work may be distributed and/or modified under the
% conditions of the LaTeX Project Public License, either
% version 1.3c of this license or (at your option) any later
% version. This version of this license is in
%    http://www.latex-project.org/lppl/lppl-1-3c.txt
% and the latest version of this license is in
%    http://www.latex-project.org/lppl.txt
% and version 1.3 or later is part of all distributions of
% LaTeX version 2005/12/01 or later.
%
% This work has the LPPL maintenance status "maintained".
%
% This Current Maintainer of this work is Heiko Oberdiek.
%
% This work consists of the main source file colonequals.dtx
% and the derived files
%    colonequals.sty, colonequals.pdf, colonequals.ins, colonequals.drv.
%
% Distribution:
%    CTAN:macros/latex/contrib/oberdiek/colonequals.dtx
%    CTAN:macros/latex/contrib/oberdiek/colonequals.pdf
%
% Unpacking:
%    (a) If colonequals.ins is present:
%           tex colonequals.ins
%    (b) Without colonequals.ins:
%           tex colonequals.dtx
%    (c) If you insist on using LaTeX
%           latex \let\install=y% \iffalse meta-comment
%
% File: colonequals.dtx
% Version: 2006/08/01 v1.0
% Info: Colon equals symbols
%
% Copyright (C) 2006 by
%    Heiko Oberdiek <heiko.oberdiek at googlemail.com>
%
% This work may be distributed and/or modified under the
% conditions of the LaTeX Project Public License, either
% version 1.3c of this license or (at your option) any later
% version. This version of this license is in
%    http://www.latex-project.org/lppl/lppl-1-3c.txt
% and the latest version of this license is in
%    http://www.latex-project.org/lppl.txt
% and version 1.3 or later is part of all distributions of
% LaTeX version 2005/12/01 or later.
%
% This work has the LPPL maintenance status "maintained".
%
% This Current Maintainer of this work is Heiko Oberdiek.
%
% This work consists of the main source file colonequals.dtx
% and the derived files
%    colonequals.sty, colonequals.pdf, colonequals.ins, colonequals.drv.
%
% Distribution:
%    CTAN:macros/latex/contrib/oberdiek/colonequals.dtx
%    CTAN:macros/latex/contrib/oberdiek/colonequals.pdf
%
% Unpacking:
%    (a) If colonequals.ins is present:
%           tex colonequals.ins
%    (b) Without colonequals.ins:
%           tex colonequals.dtx
%    (c) If you insist on using LaTeX
%           latex \let\install=y\input{colonequals.dtx}
%        (quote the arguments according to the demands of your shell)
%
% Documentation:
%    (a) If colonequals.drv is present:
%           latex colonequals.drv
%    (b) Without colonequals.drv:
%           latex colonequals.dtx; ...
%    The class ltxdoc loads the configuration file ltxdoc.cfg
%    if available. Here you can specify further options, e.g.
%    use A4 as paper format:
%       \PassOptionsToClass{a4paper}{article}
%
%    Programm calls to get the documentation (example):
%       pdflatex colonequals.dtx
%       makeindex -s gind.ist colonequals.idx
%       pdflatex colonequals.dtx
%       makeindex -s gind.ist colonequals.idx
%       pdflatex colonequals.dtx
%
% Installation:
%    TDS:tex/latex/oberdiek/colonequals.sty
%    TDS:doc/latex/oberdiek/colonequals.pdf
%    TDS:source/latex/oberdiek/colonequals.dtx
%
%<*ignore>
\begingroup
  \catcode123=1 %
  \catcode125=2 %
  \def\x{LaTeX2e}%
\expandafter\endgroup
\ifcase 0\ifx\install y1\fi\expandafter
         \ifx\csname processbatchFile\endcsname\relax\else1\fi
         \ifx\fmtname\x\else 1\fi\relax
\else\csname fi\endcsname
%</ignore>
%<*install>
\input docstrip.tex
\Msg{************************************************************************}
\Msg{* Installation}
\Msg{* Package: colonequals 2006/08/01 v1.0 Colon equals symbols (HO)}
\Msg{************************************************************************}

\keepsilent
\askforoverwritefalse

\let\MetaPrefix\relax
\preamble

This is a generated file.

Project: colonequals
Version: 2006/08/01 v1.0

Copyright (C) 2006 by
   Heiko Oberdiek <heiko.oberdiek at googlemail.com>

This work may be distributed and/or modified under the
conditions of the LaTeX Project Public License, either
version 1.3c of this license or (at your option) any later
version. This version of this license is in
   http://www.latex-project.org/lppl/lppl-1-3c.txt
and the latest version of this license is in
   http://www.latex-project.org/lppl.txt
and version 1.3 or later is part of all distributions of
LaTeX version 2005/12/01 or later.

This work has the LPPL maintenance status "maintained".

This Current Maintainer of this work is Heiko Oberdiek.

This work consists of the main source file colonequals.dtx
and the derived files
   colonequals.sty, colonequals.pdf, colonequals.ins, colonequals.drv.

\endpreamble
\let\MetaPrefix\DoubleperCent

\generate{%
  \file{colonequals.ins}{\from{colonequals.dtx}{install}}%
  \file{colonequals.drv}{\from{colonequals.dtx}{driver}}%
  \usedir{tex/latex/oberdiek}%
  \file{colonequals.sty}{\from{colonequals.dtx}{package}}%
  \nopreamble
  \nopostamble
  \usedir{source/latex/oberdiek/catalogue}%
  \file{colonequals.xml}{\from{colonequals.dtx}{catalogue}}%
}

\catcode32=13\relax% active space
\let =\space%
\Msg{************************************************************************}
\Msg{*}
\Msg{* To finish the installation you have to move the following}
\Msg{* file into a directory searched by TeX:}
\Msg{*}
\Msg{*     colonequals.sty}
\Msg{*}
\Msg{* To produce the documentation run the file `colonequals.drv'}
\Msg{* through LaTeX.}
\Msg{*}
\Msg{* Happy TeXing!}
\Msg{*}
\Msg{************************************************************************}

\endbatchfile
%</install>
%<*ignore>
\fi
%</ignore>
%<*driver>
\NeedsTeXFormat{LaTeX2e}
\ProvidesFile{colonequals.drv}%
  [2006/08/01 v1.0 Colon equals symbols (HO)]%
\documentclass{ltxdoc}
\usepackage{holtxdoc}[2011/11/22]
\usepackage{colonequals}
\usepackage{array}
\usepackage{capt-of}
\usepackage{longtable}
\begin{document}
  \DocInput{colonequals.dtx}%
\end{document}
%</driver>
% \fi
%
% \CheckSum{92}
%
% \CharacterTable
%  {Upper-case    \A\B\C\D\E\F\G\H\I\J\K\L\M\N\O\P\Q\R\S\T\U\V\W\X\Y\Z
%   Lower-case    \a\b\c\d\e\f\g\h\i\j\k\l\m\n\o\p\q\r\s\t\u\v\w\x\y\z
%   Digits        \0\1\2\3\4\5\6\7\8\9
%   Exclamation   \!     Double quote  \"     Hash (number) \#
%   Dollar        \$     Percent       \%     Ampersand     \&
%   Acute accent  \'     Left paren    \(     Right paren   \)
%   Asterisk      \*     Plus          \+     Comma         \,
%   Minus         \-     Point         \.     Solidus       \/
%   Colon         \:     Semicolon     \;     Less than     \<
%   Equals        \=     Greater than  \>     Question mark \?
%   Commercial at \@     Left bracket  \[     Backslash     \\
%   Right bracket \]     Circumflex    \^     Underscore    \_
%   Grave accent  \`     Left brace    \{     Vertical bar  \|
%   Right brace   \}     Tilde         \~}
%
% \GetFileInfo{colonequals.drv}
%
% \title{The \xpackage{colonequals} package}
% \date{2006/08/01 v1.0}
% \author{Heiko Oberdiek\\\xemail{heiko.oberdiek at googlemail.com}}
%
% \maketitle
%
% \begin{abstract}
% Package \xpackage{colonequals} defines poor man's symbols
% for math relation symbols such as ``colon equals''.
% The colon is centered around the horizontal math axis.
% \end{abstract}
%
% \tableofcontents
%
% \section{User interface}
%
% \subsection{Introduction}
%
% Math symbols consisting of the colon character can be
% constructed with the colon text character, if the math font
% lacks of the complete symbol. Often, however, the colon text
% character is not centered around the math axis. Especially
% combined with the equals symbol the composed symbol does not
% look symmetrically. Thus this packages defines a colon
% math symbol \cs{ratio} that is centered around the horizontal
% math axis. Also math symbols are provided that consist of the
% colon symbol. The package is not necessary, if the math
% font contains the composed symbols. Examples are \textsf{txfonts}
% (\cite{txfonts}) or \textsf{mathabx} (\cite{mathabx}).
%
% \subsection{Symbols}
%
% All symbols of this package are relation symbols.
% The relation property can be changed by the appropriate
% \TeX\ command \cs{mathbin}, \cs{mathord}, \dots
%
% \begin{center}
% \captionof{table}{Unicode mathematical operators}
% \kern1ex
% \begin{tabular}{l>{\scshape}l>{$}l<{$}l}
%   U+2236 & ratio        & \ratio       & \cs{ratio}       \\
%   U+2237 & proportion   & \coloncolon  & \cs{coloncolon}  \\
%   U+2239 & excess       & \colonminus  & \cs{colonminus}  \\
%   U+2254 & colon equals & \colonequals & \cs{colonequals} \\
%   U+2255 & equals colon & \equalscolon & \cs{equalscolon} \\
% \end{tabular}
% \end{center}
%
% The following grammar generates all symbols that are supported by
% this package:
% \begin{center}
% \captionof{table}{Symbol grammar}
% \kern1ex
% \begin{tabular}{@{}l>{$}r<{$}l@{}}
%   symbols & \coloncolonequals & col \\
%           & \mid & col symbol \\
%           & \mid & symbol col \\
%           & ; & \\[1ex]
%   col     & \coloncolonequals & '\texttt{:}' \\
%           & \mid & '\texttt{::}' \\
%           & ; & \\[1ex]
%   symbol  & \coloncolonequals & '\texttt{=}' \\
%           & \mid & '\texttt{-}' \\
%           & \mid & '$\approx$' \\
%           & \mid & '$\sim$' \\
%           & ; &
% \end{tabular}
% \end{center}
%
% \def\entry#1{\csname #1\endcsname&\cs{#1}\\}
% \def\entryset#1{^^A
%    \entry{colon#1}^^A
%    \entry{coloncolon#1}^^A
%    \entry{#1colon}^^A
%    \entry{#1coloncolon}^^A
% }
% \begin{longtable}{>{$}l<{$}l}
%   \caption{All relation symbols}\\
%   \entry{ratio}
%   \entry{coloncolon}
%   \entryset{equals}
%   \entryset{minus}
%   \entryset{approx}
%   \entryset{sim}
% \end{longtable}
%
% \subsection{Fine tuning}
%
% The distances in composed symbols can be configured:
%
% \begin{declcs}{colonsep}
% \end{declcs}
% Macro \cs{colonsep} is executed between the colon and
% the other symbol.
%
% \begin{declcs}{doublecolonsep}
% \end{declcs}
% Macro \cs{doublecolonsep} controls the distance between
% two colons.
%
% \subsubsection{Example}
% \begin{quote}
%   \verb|\renewcommand*{\colonsep}{\mskip-.5\thinmuskip}|
% \end{quote}
%
%
% \StopEventually{
% }
%
% \section{Implementation}
%
% \subsection{Identification}
%
%    \begin{macrocode}
%<*package>
\NeedsTeXFormat{LaTeX2e}
\ProvidesPackage{colonequals}%
  [2006/08/01 v1.0 Colon equals symbols (HO)]%
%    \end{macrocode}
%
% \subsection{Distance control}
%
%    \begin{macro}{\colonsep}
%    \begin{macrocode}
\newcommand*{\colonsep}{}
%    \end{macrocode}
%    \end{macro}
%
%    \begin{macro}{\doublecolonsep}
%    \begin{macrocode}
\newcommand*{\doublecolonsep}{}
%    \end{macrocode}
%    \end{macro}
%
% \subsection{Centered colons}
%
%    \begin{macrocode}
\def\@center@colon{%
  \mathpalette\@center@math{:}%
}
\def\@center@math#1#2{%
  \vcenter{%
    \m@th
    \hbox{$#1#2$}%
  }%
}
%    \end{macrocode}
%
%    \begin{macro}{\ratio}
%    Because the name \cs{colon} is already in use, the Unicode name
%    \cs{ratio} is used for the centered colon relation symbol.
%    (The \cs{ratio} of package \textsf{calc} is not used outside
%    calc expressions.)
%    \begin{macrocode}
\newcommand*{\ratio}{%
  \ensuremath{%
    \mathrel{%
      \@center@colon
    }%
  }%
}
%    \end{macrocode}
%    \end{macro}
%
%    \begin{macro}{\coloncolon}
%    \begin{macrocode}
\newcommand*{\coloncolon}{%
  \ensuremath{%
    \mathrel{%
      \@center@colon
      \doublecolonsep
      \@center@colon
    }%
  }%
}
%    \end{macrocode}
%    \end{macro}
%
% \subsection{Combined symbols}
%
%    \begin{macrocode}
\def\@make@colon@set#1#2{%
  \begingroup
    \let\@center@colon\relax
    \let\newcommand\relax
    \let\ensuremath\relax
    \let\mathrel\relax
    \let\colonsep\relax
    \let\doublecolonsep\relax
    \def\csx##1{%
      \expandafter\noexpand\csname ##1\endcsname
    }%
    \edef\x{\endgroup
      \newcommand*{\csx{colon#1}}{%
        \ensuremath{%
          \mathrel{%
            \@center@colon
            \colonsep
            {#2}%
          }%
        }%
      }%
      \newcommand*{\csx{coloncolon#1}}{%
        \ensuremath{%
          \mathrel{%
            \@center@colon
            \doublecolonsep
            \@center@colon
            \colonsep
            {#2}%
          }%
        }%
      }%
      \newcommand*{\csx{#1colon}}{%
        \ensuremath{%
          \mathrel{%
            {#2}%
            \colonsep
            \@center@colon
          }%
        }%
      }%
      \newcommand*{\csx{#1coloncolon}}{%
        \ensuremath{%
          \mathrel{%
            {#2}%
            \colonsep
            \@center@colon
            \doublecolonsep
            \@center@colon
          }%
        }%
      }%
    }%
  \x
}
%    \end{macrocode}
%
%    \begin{macrocode}
\@make@colon@set{equals}{=}%
\@make@colon@set{minus}{-}%
\@make@colon@set{approx}{\approx}
\@make@colon@set{sim}{\sim}
%    \end{macrocode}
%
%    \begin{macrocode}
%</package>
%    \end{macrocode}
%
% \section{Installation}
%
% \subsection{Download}
%
% \paragraph{Package.} This package is available on
% CTAN\footnote{\url{ftp://ftp.ctan.org/tex-archive/}}:
% \begin{description}
% \item[\CTAN{macros/latex/contrib/oberdiek/colonequals.dtx}] The source file.
% \item[\CTAN{macros/latex/contrib/oberdiek/colonequals.pdf}] Documentation.
% \end{description}
%
%
% \paragraph{Bundle.} All the packages of the bundle `oberdiek'
% are also available in a TDS compliant ZIP archive. There
% the packages are already unpacked and the documentation files
% are generated. The files and directories obey the TDS standard.
% \begin{description}
% \item[\CTAN{install/macros/latex/contrib/oberdiek.tds.zip}]
% \end{description}
% \emph{TDS} refers to the standard ``A Directory Structure
% for \TeX\ Files'' (\CTAN{tds/tds.pdf}). Directories
% with \xfile{texmf} in their name are usually organized this way.
%
% \subsection{Bundle installation}
%
% \paragraph{Unpacking.} Unpack the \xfile{oberdiek.tds.zip} in the
% TDS tree (also known as \xfile{texmf} tree) of your choice.
% Example (linux):
% \begin{quote}
%   |unzip oberdiek.tds.zip -d ~/texmf|
% \end{quote}
%
% \paragraph{Script installation.}
% Check the directory \xfile{TDS:scripts/oberdiek/} for
% scripts that need further installation steps.
% Package \xpackage{attachfile2} comes with the Perl script
% \xfile{pdfatfi.pl} that should be installed in such a way
% that it can be called as \texttt{pdfatfi}.
% Example (linux):
% \begin{quote}
%   |chmod +x scripts/oberdiek/pdfatfi.pl|\\
%   |cp scripts/oberdiek/pdfatfi.pl /usr/local/bin/|
% \end{quote}
%
% \subsection{Package installation}
%
% \paragraph{Unpacking.} The \xfile{.dtx} file is a self-extracting
% \docstrip\ archive. The files are extracted by running the
% \xfile{.dtx} through \plainTeX:
% \begin{quote}
%   \verb|tex colonequals.dtx|
% \end{quote}
%
% \paragraph{TDS.} Now the different files must be moved into
% the different directories in your installation TDS tree
% (also known as \xfile{texmf} tree):
% \begin{quote}
% \def\t{^^A
% \begin{tabular}{@{}>{\ttfamily}l@{ $\rightarrow$ }>{\ttfamily}l@{}}
%   colonequals.sty & tex/latex/oberdiek/colonequals.sty\\
%   colonequals.pdf & doc/latex/oberdiek/colonequals.pdf\\
%   colonequals.dtx & source/latex/oberdiek/colonequals.dtx\\
% \end{tabular}^^A
% }^^A
% \sbox0{\t}^^A
% \ifdim\wd0>\linewidth
%   \begingroup
%     \advance\linewidth by\leftmargin
%     \advance\linewidth by\rightmargin
%   \edef\x{\endgroup
%     \def\noexpand\lw{\the\linewidth}^^A
%   }\x
%   \def\lwbox{^^A
%     \leavevmode
%     \hbox to \linewidth{^^A
%       \kern-\leftmargin\relax
%       \hss
%       \usebox0
%       \hss
%       \kern-\rightmargin\relax
%     }^^A
%   }^^A
%   \ifdim\wd0>\lw
%     \sbox0{\small\t}^^A
%     \ifdim\wd0>\linewidth
%       \ifdim\wd0>\lw
%         \sbox0{\footnotesize\t}^^A
%         \ifdim\wd0>\linewidth
%           \ifdim\wd0>\lw
%             \sbox0{\scriptsize\t}^^A
%             \ifdim\wd0>\linewidth
%               \ifdim\wd0>\lw
%                 \sbox0{\tiny\t}^^A
%                 \ifdim\wd0>\linewidth
%                   \lwbox
%                 \else
%                   \usebox0
%                 \fi
%               \else
%                 \lwbox
%               \fi
%             \else
%               \usebox0
%             \fi
%           \else
%             \lwbox
%           \fi
%         \else
%           \usebox0
%         \fi
%       \else
%         \lwbox
%       \fi
%     \else
%       \usebox0
%     \fi
%   \else
%     \lwbox
%   \fi
% \else
%   \usebox0
% \fi
% \end{quote}
% If you have a \xfile{docstrip.cfg} that configures and enables \docstrip's
% TDS installing feature, then some files can already be in the right
% place, see the documentation of \docstrip.
%
% \subsection{Refresh file name databases}
%
% If your \TeX~distribution
% (\teTeX, \mikTeX, \dots) relies on file name databases, you must refresh
% these. For example, \teTeX\ users run \verb|texhash| or
% \verb|mktexlsr|.
%
% \subsection{Some details for the interested}
%
% \paragraph{Attached source.}
%
% The PDF documentation on CTAN also includes the
% \xfile{.dtx} source file. It can be extracted by
% AcrobatReader 6 or higher. Another option is \textsf{pdftk},
% e.g. unpack the file into the current directory:
% \begin{quote}
%   \verb|pdftk colonequals.pdf unpack_files output .|
% \end{quote}
%
% \paragraph{Unpacking with \LaTeX.}
% The \xfile{.dtx} chooses its action depending on the format:
% \begin{description}
% \item[\plainTeX:] Run \docstrip\ and extract the files.
% \item[\LaTeX:] Generate the documentation.
% \end{description}
% If you insist on using \LaTeX\ for \docstrip\ (really,
% \docstrip\ does not need \LaTeX), then inform the autodetect routine
% about your intention:
% \begin{quote}
%   \verb|latex \let\install=y\input{colonequals.dtx}|
% \end{quote}
% Do not forget to quote the argument according to the demands
% of your shell.
%
% \paragraph{Generating the documentation.}
% You can use both the \xfile{.dtx} or the \xfile{.drv} to generate
% the documentation. The process can be configured by the
% configuration file \xfile{ltxdoc.cfg}. For instance, put this
% line into this file, if you want to have A4 as paper format:
% \begin{quote}
%   \verb|\PassOptionsToClass{a4paper}{article}|
% \end{quote}
% An example follows how to generate the
% documentation with pdf\LaTeX:
% \begin{quote}
%\begin{verbatim}
%pdflatex colonequals.dtx
%makeindex -s gind.ist colonequals.idx
%pdflatex colonequals.dtx
%makeindex -s gind.ist colonequals.idx
%pdflatex colonequals.dtx
%\end{verbatim}
% \end{quote}
%
% \section{Catalogue}
%
% The following XML file can be used as source for the
% \href{http://mirror.ctan.org/help/Catalogue/catalogue.html}{\TeX\ Catalogue}.
% The elements \texttt{caption} and \texttt{description} are imported
% from the original XML file from the Catalogue.
% The name of the XML file in the Catalogue is \xfile{colonequals.xml}.
%    \begin{macrocode}
%<*catalogue>
<?xml version='1.0' encoding='us-ascii'?>
<!DOCTYPE entry SYSTEM 'catalogue.dtd'>
<entry datestamp='$Date$' modifier='$Author$' id='colonequals'>
  <name>colonequals</name>
  <caption>Colon equals symbols.</caption>
  <authorref id='auth:oberdiek'/>
  <copyright owner='Heiko Oberdiek' year='2006'/>
  <license type='lppl1.3'/>
  <version number='1.0'/>
  <description>
    This package defines poor man&#x2018;s symbols for mathematical
    relation symbols such as &#x201C;colon equals&#x201D;.
    The colon is centered around the horizontal math axis.
    <p/>
    The package is part of the <xref refid='oberdiek'>oberdiek</xref>
    bundle.
  </description>
  <documentation details='Package documentation'
      href='ctan:/macros/latex/contrib/oberdiek/colonequals.pdf'/>
  <ctan file='true' path='/macros/latex/contrib/oberdiek/colonequals.dtx'/>
  <miktex location='oberdiek'/>
  <texlive location='oberdiek'/>
  <install path='/macros/latex/contrib/oberdiek/oberdiek.tds.zip'/>
</entry>
%</catalogue>
%    \end{macrocode}
%
% \begin{thebibliography}{9}
%
% \bibitem{txfonts}
%   Young Ryu: \textit{The TX Fonts};
%   2000/12/15;
%   \CTAN{fonts/txfonts/}.
%
% \bibitem{mathabx}
%   Anthony Phan: \textit{Mathabx font series};
%   2005/05/16;
%   \CTAN{fonts/mathabx/}.
%
% \end{thebibliography}
%
% \begin{History}
%   \begin{Version}{2006/08/01 v1.0}
%   \item
%     First version.
%   \end{Version}
% \end{History}
%
% \PrintIndex
%
% \Finale
\endinput

%        (quote the arguments according to the demands of your shell)
%
% Documentation:
%    (a) If colonequals.drv is present:
%           latex colonequals.drv
%    (b) Without colonequals.drv:
%           latex colonequals.dtx; ...
%    The class ltxdoc loads the configuration file ltxdoc.cfg
%    if available. Here you can specify further options, e.g.
%    use A4 as paper format:
%       \PassOptionsToClass{a4paper}{article}
%
%    Programm calls to get the documentation (example):
%       pdflatex colonequals.dtx
%       makeindex -s gind.ist colonequals.idx
%       pdflatex colonequals.dtx
%       makeindex -s gind.ist colonequals.idx
%       pdflatex colonequals.dtx
%
% Installation:
%    TDS:tex/latex/oberdiek/colonequals.sty
%    TDS:doc/latex/oberdiek/colonequals.pdf
%    TDS:source/latex/oberdiek/colonequals.dtx
%
%<*ignore>
\begingroup
  \catcode123=1 %
  \catcode125=2 %
  \def\x{LaTeX2e}%
\expandafter\endgroup
\ifcase 0\ifx\install y1\fi\expandafter
         \ifx\csname processbatchFile\endcsname\relax\else1\fi
         \ifx\fmtname\x\else 1\fi\relax
\else\csname fi\endcsname
%</ignore>
%<*install>
\input docstrip.tex
\Msg{************************************************************************}
\Msg{* Installation}
\Msg{* Package: colonequals 2006/08/01 v1.0 Colon equals symbols (HO)}
\Msg{************************************************************************}

\keepsilent
\askforoverwritefalse

\let\MetaPrefix\relax
\preamble

This is a generated file.

Project: colonequals
Version: 2006/08/01 v1.0

Copyright (C) 2006 by
   Heiko Oberdiek <heiko.oberdiek at googlemail.com>

This work may be distributed and/or modified under the
conditions of the LaTeX Project Public License, either
version 1.3c of this license or (at your option) any later
version. This version of this license is in
   http://www.latex-project.org/lppl/lppl-1-3c.txt
and the latest version of this license is in
   http://www.latex-project.org/lppl.txt
and version 1.3 or later is part of all distributions of
LaTeX version 2005/12/01 or later.

This work has the LPPL maintenance status "maintained".

This Current Maintainer of this work is Heiko Oberdiek.

This work consists of the main source file colonequals.dtx
and the derived files
   colonequals.sty, colonequals.pdf, colonequals.ins, colonequals.drv.

\endpreamble
\let\MetaPrefix\DoubleperCent

\generate{%
  \file{colonequals.ins}{\from{colonequals.dtx}{install}}%
  \file{colonequals.drv}{\from{colonequals.dtx}{driver}}%
  \usedir{tex/latex/oberdiek}%
  \file{colonequals.sty}{\from{colonequals.dtx}{package}}%
  \nopreamble
  \nopostamble
  \usedir{source/latex/oberdiek/catalogue}%
  \file{colonequals.xml}{\from{colonequals.dtx}{catalogue}}%
}

\catcode32=13\relax% active space
\let =\space%
\Msg{************************************************************************}
\Msg{*}
\Msg{* To finish the installation you have to move the following}
\Msg{* file into a directory searched by TeX:}
\Msg{*}
\Msg{*     colonequals.sty}
\Msg{*}
\Msg{* To produce the documentation run the file `colonequals.drv'}
\Msg{* through LaTeX.}
\Msg{*}
\Msg{* Happy TeXing!}
\Msg{*}
\Msg{************************************************************************}

\endbatchfile
%</install>
%<*ignore>
\fi
%</ignore>
%<*driver>
\NeedsTeXFormat{LaTeX2e}
\ProvidesFile{colonequals.drv}%
  [2006/08/01 v1.0 Colon equals symbols (HO)]%
\documentclass{ltxdoc}
\usepackage{holtxdoc}[2011/11/22]
\usepackage{colonequals}
\usepackage{array}
\usepackage{capt-of}
\usepackage{longtable}
\begin{document}
  \DocInput{colonequals.dtx}%
\end{document}
%</driver>
% \fi
%
% \CheckSum{92}
%
% \CharacterTable
%  {Upper-case    \A\B\C\D\E\F\G\H\I\J\K\L\M\N\O\P\Q\R\S\T\U\V\W\X\Y\Z
%   Lower-case    \a\b\c\d\e\f\g\h\i\j\k\l\m\n\o\p\q\r\s\t\u\v\w\x\y\z
%   Digits        \0\1\2\3\4\5\6\7\8\9
%   Exclamation   \!     Double quote  \"     Hash (number) \#
%   Dollar        \$     Percent       \%     Ampersand     \&
%   Acute accent  \'     Left paren    \(     Right paren   \)
%   Asterisk      \*     Plus          \+     Comma         \,
%   Minus         \-     Point         \.     Solidus       \/
%   Colon         \:     Semicolon     \;     Less than     \<
%   Equals        \=     Greater than  \>     Question mark \?
%   Commercial at \@     Left bracket  \[     Backslash     \\
%   Right bracket \]     Circumflex    \^     Underscore    \_
%   Grave accent  \`     Left brace    \{     Vertical bar  \|
%   Right brace   \}     Tilde         \~}
%
% \GetFileInfo{colonequals.drv}
%
% \title{The \xpackage{colonequals} package}
% \date{2006/08/01 v1.0}
% \author{Heiko Oberdiek\\\xemail{heiko.oberdiek at googlemail.com}}
%
% \maketitle
%
% \begin{abstract}
% Package \xpackage{colonequals} defines poor man's symbols
% for math relation symbols such as ``colon equals''.
% The colon is centered around the horizontal math axis.
% \end{abstract}
%
% \tableofcontents
%
% \section{User interface}
%
% \subsection{Introduction}
%
% Math symbols consisting of the colon character can be
% constructed with the colon text character, if the math font
% lacks of the complete symbol. Often, however, the colon text
% character is not centered around the math axis. Especially
% combined with the equals symbol the composed symbol does not
% look symmetrically. Thus this packages defines a colon
% math symbol \cs{ratio} that is centered around the horizontal
% math axis. Also math symbols are provided that consist of the
% colon symbol. The package is not necessary, if the math
% font contains the composed symbols. Examples are \textsf{txfonts}
% (\cite{txfonts}) or \textsf{mathabx} (\cite{mathabx}).
%
% \subsection{Symbols}
%
% All symbols of this package are relation symbols.
% The relation property can be changed by the appropriate
% \TeX\ command \cs{mathbin}, \cs{mathord}, \dots
%
% \begin{center}
% \captionof{table}{Unicode mathematical operators}
% \kern1ex
% \begin{tabular}{l>{\scshape}l>{$}l<{$}l}
%   U+2236 & ratio        & \ratio       & \cs{ratio}       \\
%   U+2237 & proportion   & \coloncolon  & \cs{coloncolon}  \\
%   U+2239 & excess       & \colonminus  & \cs{colonminus}  \\
%   U+2254 & colon equals & \colonequals & \cs{colonequals} \\
%   U+2255 & equals colon & \equalscolon & \cs{equalscolon} \\
% \end{tabular}
% \end{center}
%
% The following grammar generates all symbols that are supported by
% this package:
% \begin{center}
% \captionof{table}{Symbol grammar}
% \kern1ex
% \begin{tabular}{@{}l>{$}r<{$}l@{}}
%   symbols & \coloncolonequals & col \\
%           & \mid & col symbol \\
%           & \mid & symbol col \\
%           & ; & \\[1ex]
%   col     & \coloncolonequals & '\texttt{:}' \\
%           & \mid & '\texttt{::}' \\
%           & ; & \\[1ex]
%   symbol  & \coloncolonequals & '\texttt{=}' \\
%           & \mid & '\texttt{-}' \\
%           & \mid & '$\approx$' \\
%           & \mid & '$\sim$' \\
%           & ; &
% \end{tabular}
% \end{center}
%
% \def\entry#1{\csname #1\endcsname&\cs{#1}\\}
% \def\entryset#1{^^A
%    \entry{colon#1}^^A
%    \entry{coloncolon#1}^^A
%    \entry{#1colon}^^A
%    \entry{#1coloncolon}^^A
% }
% \begin{longtable}{>{$}l<{$}l}
%   \caption{All relation symbols}\\
%   \entry{ratio}
%   \entry{coloncolon}
%   \entryset{equals}
%   \entryset{minus}
%   \entryset{approx}
%   \entryset{sim}
% \end{longtable}
%
% \subsection{Fine tuning}
%
% The distances in composed symbols can be configured:
%
% \begin{declcs}{colonsep}
% \end{declcs}
% Macro \cs{colonsep} is executed between the colon and
% the other symbol.
%
% \begin{declcs}{doublecolonsep}
% \end{declcs}
% Macro \cs{doublecolonsep} controls the distance between
% two colons.
%
% \subsubsection{Example}
% \begin{quote}
%   \verb|\renewcommand*{\colonsep}{\mskip-.5\thinmuskip}|
% \end{quote}
%
%
% \StopEventually{
% }
%
% \section{Implementation}
%
% \subsection{Identification}
%
%    \begin{macrocode}
%<*package>
\NeedsTeXFormat{LaTeX2e}
\ProvidesPackage{colonequals}%
  [2006/08/01 v1.0 Colon equals symbols (HO)]%
%    \end{macrocode}
%
% \subsection{Distance control}
%
%    \begin{macro}{\colonsep}
%    \begin{macrocode}
\newcommand*{\colonsep}{}
%    \end{macrocode}
%    \end{macro}
%
%    \begin{macro}{\doublecolonsep}
%    \begin{macrocode}
\newcommand*{\doublecolonsep}{}
%    \end{macrocode}
%    \end{macro}
%
% \subsection{Centered colons}
%
%    \begin{macrocode}
\def\@center@colon{%
  \mathpalette\@center@math{:}%
}
\def\@center@math#1#2{%
  \vcenter{%
    \m@th
    \hbox{$#1#2$}%
  }%
}
%    \end{macrocode}
%
%    \begin{macro}{\ratio}
%    Because the name \cs{colon} is already in use, the Unicode name
%    \cs{ratio} is used for the centered colon relation symbol.
%    (The \cs{ratio} of package \textsf{calc} is not used outside
%    calc expressions.)
%    \begin{macrocode}
\newcommand*{\ratio}{%
  \ensuremath{%
    \mathrel{%
      \@center@colon
    }%
  }%
}
%    \end{macrocode}
%    \end{macro}
%
%    \begin{macro}{\coloncolon}
%    \begin{macrocode}
\newcommand*{\coloncolon}{%
  \ensuremath{%
    \mathrel{%
      \@center@colon
      \doublecolonsep
      \@center@colon
    }%
  }%
}
%    \end{macrocode}
%    \end{macro}
%
% \subsection{Combined symbols}
%
%    \begin{macrocode}
\def\@make@colon@set#1#2{%
  \begingroup
    \let\@center@colon\relax
    \let\newcommand\relax
    \let\ensuremath\relax
    \let\mathrel\relax
    \let\colonsep\relax
    \let\doublecolonsep\relax
    \def\csx##1{%
      \expandafter\noexpand\csname ##1\endcsname
    }%
    \edef\x{\endgroup
      \newcommand*{\csx{colon#1}}{%
        \ensuremath{%
          \mathrel{%
            \@center@colon
            \colonsep
            {#2}%
          }%
        }%
      }%
      \newcommand*{\csx{coloncolon#1}}{%
        \ensuremath{%
          \mathrel{%
            \@center@colon
            \doublecolonsep
            \@center@colon
            \colonsep
            {#2}%
          }%
        }%
      }%
      \newcommand*{\csx{#1colon}}{%
        \ensuremath{%
          \mathrel{%
            {#2}%
            \colonsep
            \@center@colon
          }%
        }%
      }%
      \newcommand*{\csx{#1coloncolon}}{%
        \ensuremath{%
          \mathrel{%
            {#2}%
            \colonsep
            \@center@colon
            \doublecolonsep
            \@center@colon
          }%
        }%
      }%
    }%
  \x
}
%    \end{macrocode}
%
%    \begin{macrocode}
\@make@colon@set{equals}{=}%
\@make@colon@set{minus}{-}%
\@make@colon@set{approx}{\approx}
\@make@colon@set{sim}{\sim}
%    \end{macrocode}
%
%    \begin{macrocode}
%</package>
%    \end{macrocode}
%
% \section{Installation}
%
% \subsection{Download}
%
% \paragraph{Package.} This package is available on
% CTAN\footnote{\url{ftp://ftp.ctan.org/tex-archive/}}:
% \begin{description}
% \item[\CTAN{macros/latex/contrib/oberdiek/colonequals.dtx}] The source file.
% \item[\CTAN{macros/latex/contrib/oberdiek/colonequals.pdf}] Documentation.
% \end{description}
%
%
% \paragraph{Bundle.} All the packages of the bundle `oberdiek'
% are also available in a TDS compliant ZIP archive. There
% the packages are already unpacked and the documentation files
% are generated. The files and directories obey the TDS standard.
% \begin{description}
% \item[\CTAN{install/macros/latex/contrib/oberdiek.tds.zip}]
% \end{description}
% \emph{TDS} refers to the standard ``A Directory Structure
% for \TeX\ Files'' (\CTAN{tds/tds.pdf}). Directories
% with \xfile{texmf} in their name are usually organized this way.
%
% \subsection{Bundle installation}
%
% \paragraph{Unpacking.} Unpack the \xfile{oberdiek.tds.zip} in the
% TDS tree (also known as \xfile{texmf} tree) of your choice.
% Example (linux):
% \begin{quote}
%   |unzip oberdiek.tds.zip -d ~/texmf|
% \end{quote}
%
% \paragraph{Script installation.}
% Check the directory \xfile{TDS:scripts/oberdiek/} for
% scripts that need further installation steps.
% Package \xpackage{attachfile2} comes with the Perl script
% \xfile{pdfatfi.pl} that should be installed in such a way
% that it can be called as \texttt{pdfatfi}.
% Example (linux):
% \begin{quote}
%   |chmod +x scripts/oberdiek/pdfatfi.pl|\\
%   |cp scripts/oberdiek/pdfatfi.pl /usr/local/bin/|
% \end{quote}
%
% \subsection{Package installation}
%
% \paragraph{Unpacking.} The \xfile{.dtx} file is a self-extracting
% \docstrip\ archive. The files are extracted by running the
% \xfile{.dtx} through \plainTeX:
% \begin{quote}
%   \verb|tex colonequals.dtx|
% \end{quote}
%
% \paragraph{TDS.} Now the different files must be moved into
% the different directories in your installation TDS tree
% (also known as \xfile{texmf} tree):
% \begin{quote}
% \def\t{^^A
% \begin{tabular}{@{}>{\ttfamily}l@{ $\rightarrow$ }>{\ttfamily}l@{}}
%   colonequals.sty & tex/latex/oberdiek/colonequals.sty\\
%   colonequals.pdf & doc/latex/oberdiek/colonequals.pdf\\
%   colonequals.dtx & source/latex/oberdiek/colonequals.dtx\\
% \end{tabular}^^A
% }^^A
% \sbox0{\t}^^A
% \ifdim\wd0>\linewidth
%   \begingroup
%     \advance\linewidth by\leftmargin
%     \advance\linewidth by\rightmargin
%   \edef\x{\endgroup
%     \def\noexpand\lw{\the\linewidth}^^A
%   }\x
%   \def\lwbox{^^A
%     \leavevmode
%     \hbox to \linewidth{^^A
%       \kern-\leftmargin\relax
%       \hss
%       \usebox0
%       \hss
%       \kern-\rightmargin\relax
%     }^^A
%   }^^A
%   \ifdim\wd0>\lw
%     \sbox0{\small\t}^^A
%     \ifdim\wd0>\linewidth
%       \ifdim\wd0>\lw
%         \sbox0{\footnotesize\t}^^A
%         \ifdim\wd0>\linewidth
%           \ifdim\wd0>\lw
%             \sbox0{\scriptsize\t}^^A
%             \ifdim\wd0>\linewidth
%               \ifdim\wd0>\lw
%                 \sbox0{\tiny\t}^^A
%                 \ifdim\wd0>\linewidth
%                   \lwbox
%                 \else
%                   \usebox0
%                 \fi
%               \else
%                 \lwbox
%               \fi
%             \else
%               \usebox0
%             \fi
%           \else
%             \lwbox
%           \fi
%         \else
%           \usebox0
%         \fi
%       \else
%         \lwbox
%       \fi
%     \else
%       \usebox0
%     \fi
%   \else
%     \lwbox
%   \fi
% \else
%   \usebox0
% \fi
% \end{quote}
% If you have a \xfile{docstrip.cfg} that configures and enables \docstrip's
% TDS installing feature, then some files can already be in the right
% place, see the documentation of \docstrip.
%
% \subsection{Refresh file name databases}
%
% If your \TeX~distribution
% (\teTeX, \mikTeX, \dots) relies on file name databases, you must refresh
% these. For example, \teTeX\ users run \verb|texhash| or
% \verb|mktexlsr|.
%
% \subsection{Some details for the interested}
%
% \paragraph{Attached source.}
%
% The PDF documentation on CTAN also includes the
% \xfile{.dtx} source file. It can be extracted by
% AcrobatReader 6 or higher. Another option is \textsf{pdftk},
% e.g. unpack the file into the current directory:
% \begin{quote}
%   \verb|pdftk colonequals.pdf unpack_files output .|
% \end{quote}
%
% \paragraph{Unpacking with \LaTeX.}
% The \xfile{.dtx} chooses its action depending on the format:
% \begin{description}
% \item[\plainTeX:] Run \docstrip\ and extract the files.
% \item[\LaTeX:] Generate the documentation.
% \end{description}
% If you insist on using \LaTeX\ for \docstrip\ (really,
% \docstrip\ does not need \LaTeX), then inform the autodetect routine
% about your intention:
% \begin{quote}
%   \verb|latex \let\install=y% \iffalse meta-comment
%
% File: colonequals.dtx
% Version: 2006/08/01 v1.0
% Info: Colon equals symbols
%
% Copyright (C) 2006 by
%    Heiko Oberdiek <heiko.oberdiek at googlemail.com>
%
% This work may be distributed and/or modified under the
% conditions of the LaTeX Project Public License, either
% version 1.3c of this license or (at your option) any later
% version. This version of this license is in
%    http://www.latex-project.org/lppl/lppl-1-3c.txt
% and the latest version of this license is in
%    http://www.latex-project.org/lppl.txt
% and version 1.3 or later is part of all distributions of
% LaTeX version 2005/12/01 or later.
%
% This work has the LPPL maintenance status "maintained".
%
% This Current Maintainer of this work is Heiko Oberdiek.
%
% This work consists of the main source file colonequals.dtx
% and the derived files
%    colonequals.sty, colonequals.pdf, colonequals.ins, colonequals.drv.
%
% Distribution:
%    CTAN:macros/latex/contrib/oberdiek/colonequals.dtx
%    CTAN:macros/latex/contrib/oberdiek/colonequals.pdf
%
% Unpacking:
%    (a) If colonequals.ins is present:
%           tex colonequals.ins
%    (b) Without colonequals.ins:
%           tex colonequals.dtx
%    (c) If you insist on using LaTeX
%           latex \let\install=y\input{colonequals.dtx}
%        (quote the arguments according to the demands of your shell)
%
% Documentation:
%    (a) If colonequals.drv is present:
%           latex colonequals.drv
%    (b) Without colonequals.drv:
%           latex colonequals.dtx; ...
%    The class ltxdoc loads the configuration file ltxdoc.cfg
%    if available. Here you can specify further options, e.g.
%    use A4 as paper format:
%       \PassOptionsToClass{a4paper}{article}
%
%    Programm calls to get the documentation (example):
%       pdflatex colonequals.dtx
%       makeindex -s gind.ist colonequals.idx
%       pdflatex colonequals.dtx
%       makeindex -s gind.ist colonequals.idx
%       pdflatex colonequals.dtx
%
% Installation:
%    TDS:tex/latex/oberdiek/colonequals.sty
%    TDS:doc/latex/oberdiek/colonequals.pdf
%    TDS:source/latex/oberdiek/colonequals.dtx
%
%<*ignore>
\begingroup
  \catcode123=1 %
  \catcode125=2 %
  \def\x{LaTeX2e}%
\expandafter\endgroup
\ifcase 0\ifx\install y1\fi\expandafter
         \ifx\csname processbatchFile\endcsname\relax\else1\fi
         \ifx\fmtname\x\else 1\fi\relax
\else\csname fi\endcsname
%</ignore>
%<*install>
\input docstrip.tex
\Msg{************************************************************************}
\Msg{* Installation}
\Msg{* Package: colonequals 2006/08/01 v1.0 Colon equals symbols (HO)}
\Msg{************************************************************************}

\keepsilent
\askforoverwritefalse

\let\MetaPrefix\relax
\preamble

This is a generated file.

Project: colonequals
Version: 2006/08/01 v1.0

Copyright (C) 2006 by
   Heiko Oberdiek <heiko.oberdiek at googlemail.com>

This work may be distributed and/or modified under the
conditions of the LaTeX Project Public License, either
version 1.3c of this license or (at your option) any later
version. This version of this license is in
   http://www.latex-project.org/lppl/lppl-1-3c.txt
and the latest version of this license is in
   http://www.latex-project.org/lppl.txt
and version 1.3 or later is part of all distributions of
LaTeX version 2005/12/01 or later.

This work has the LPPL maintenance status "maintained".

This Current Maintainer of this work is Heiko Oberdiek.

This work consists of the main source file colonequals.dtx
and the derived files
   colonequals.sty, colonequals.pdf, colonequals.ins, colonequals.drv.

\endpreamble
\let\MetaPrefix\DoubleperCent

\generate{%
  \file{colonequals.ins}{\from{colonequals.dtx}{install}}%
  \file{colonequals.drv}{\from{colonequals.dtx}{driver}}%
  \usedir{tex/latex/oberdiek}%
  \file{colonequals.sty}{\from{colonequals.dtx}{package}}%
  \nopreamble
  \nopostamble
  \usedir{source/latex/oberdiek/catalogue}%
  \file{colonequals.xml}{\from{colonequals.dtx}{catalogue}}%
}

\catcode32=13\relax% active space
\let =\space%
\Msg{************************************************************************}
\Msg{*}
\Msg{* To finish the installation you have to move the following}
\Msg{* file into a directory searched by TeX:}
\Msg{*}
\Msg{*     colonequals.sty}
\Msg{*}
\Msg{* To produce the documentation run the file `colonequals.drv'}
\Msg{* through LaTeX.}
\Msg{*}
\Msg{* Happy TeXing!}
\Msg{*}
\Msg{************************************************************************}

\endbatchfile
%</install>
%<*ignore>
\fi
%</ignore>
%<*driver>
\NeedsTeXFormat{LaTeX2e}
\ProvidesFile{colonequals.drv}%
  [2006/08/01 v1.0 Colon equals symbols (HO)]%
\documentclass{ltxdoc}
\usepackage{holtxdoc}[2011/11/22]
\usepackage{colonequals}
\usepackage{array}
\usepackage{capt-of}
\usepackage{longtable}
\begin{document}
  \DocInput{colonequals.dtx}%
\end{document}
%</driver>
% \fi
%
% \CheckSum{92}
%
% \CharacterTable
%  {Upper-case    \A\B\C\D\E\F\G\H\I\J\K\L\M\N\O\P\Q\R\S\T\U\V\W\X\Y\Z
%   Lower-case    \a\b\c\d\e\f\g\h\i\j\k\l\m\n\o\p\q\r\s\t\u\v\w\x\y\z
%   Digits        \0\1\2\3\4\5\6\7\8\9
%   Exclamation   \!     Double quote  \"     Hash (number) \#
%   Dollar        \$     Percent       \%     Ampersand     \&
%   Acute accent  \'     Left paren    \(     Right paren   \)
%   Asterisk      \*     Plus          \+     Comma         \,
%   Minus         \-     Point         \.     Solidus       \/
%   Colon         \:     Semicolon     \;     Less than     \<
%   Equals        \=     Greater than  \>     Question mark \?
%   Commercial at \@     Left bracket  \[     Backslash     \\
%   Right bracket \]     Circumflex    \^     Underscore    \_
%   Grave accent  \`     Left brace    \{     Vertical bar  \|
%   Right brace   \}     Tilde         \~}
%
% \GetFileInfo{colonequals.drv}
%
% \title{The \xpackage{colonequals} package}
% \date{2006/08/01 v1.0}
% \author{Heiko Oberdiek\\\xemail{heiko.oberdiek at googlemail.com}}
%
% \maketitle
%
% \begin{abstract}
% Package \xpackage{colonequals} defines poor man's symbols
% for math relation symbols such as ``colon equals''.
% The colon is centered around the horizontal math axis.
% \end{abstract}
%
% \tableofcontents
%
% \section{User interface}
%
% \subsection{Introduction}
%
% Math symbols consisting of the colon character can be
% constructed with the colon text character, if the math font
% lacks of the complete symbol. Often, however, the colon text
% character is not centered around the math axis. Especially
% combined with the equals symbol the composed symbol does not
% look symmetrically. Thus this packages defines a colon
% math symbol \cs{ratio} that is centered around the horizontal
% math axis. Also math symbols are provided that consist of the
% colon symbol. The package is not necessary, if the math
% font contains the composed symbols. Examples are \textsf{txfonts}
% (\cite{txfonts}) or \textsf{mathabx} (\cite{mathabx}).
%
% \subsection{Symbols}
%
% All symbols of this package are relation symbols.
% The relation property can be changed by the appropriate
% \TeX\ command \cs{mathbin}, \cs{mathord}, \dots
%
% \begin{center}
% \captionof{table}{Unicode mathematical operators}
% \kern1ex
% \begin{tabular}{l>{\scshape}l>{$}l<{$}l}
%   U+2236 & ratio        & \ratio       & \cs{ratio}       \\
%   U+2237 & proportion   & \coloncolon  & \cs{coloncolon}  \\
%   U+2239 & excess       & \colonminus  & \cs{colonminus}  \\
%   U+2254 & colon equals & \colonequals & \cs{colonequals} \\
%   U+2255 & equals colon & \equalscolon & \cs{equalscolon} \\
% \end{tabular}
% \end{center}
%
% The following grammar generates all symbols that are supported by
% this package:
% \begin{center}
% \captionof{table}{Symbol grammar}
% \kern1ex
% \begin{tabular}{@{}l>{$}r<{$}l@{}}
%   symbols & \coloncolonequals & col \\
%           & \mid & col symbol \\
%           & \mid & symbol col \\
%           & ; & \\[1ex]
%   col     & \coloncolonequals & '\texttt{:}' \\
%           & \mid & '\texttt{::}' \\
%           & ; & \\[1ex]
%   symbol  & \coloncolonequals & '\texttt{=}' \\
%           & \mid & '\texttt{-}' \\
%           & \mid & '$\approx$' \\
%           & \mid & '$\sim$' \\
%           & ; &
% \end{tabular}
% \end{center}
%
% \def\entry#1{\csname #1\endcsname&\cs{#1}\\}
% \def\entryset#1{^^A
%    \entry{colon#1}^^A
%    \entry{coloncolon#1}^^A
%    \entry{#1colon}^^A
%    \entry{#1coloncolon}^^A
% }
% \begin{longtable}{>{$}l<{$}l}
%   \caption{All relation symbols}\\
%   \entry{ratio}
%   \entry{coloncolon}
%   \entryset{equals}
%   \entryset{minus}
%   \entryset{approx}
%   \entryset{sim}
% \end{longtable}
%
% \subsection{Fine tuning}
%
% The distances in composed symbols can be configured:
%
% \begin{declcs}{colonsep}
% \end{declcs}
% Macro \cs{colonsep} is executed between the colon and
% the other symbol.
%
% \begin{declcs}{doublecolonsep}
% \end{declcs}
% Macro \cs{doublecolonsep} controls the distance between
% two colons.
%
% \subsubsection{Example}
% \begin{quote}
%   \verb|\renewcommand*{\colonsep}{\mskip-.5\thinmuskip}|
% \end{quote}
%
%
% \StopEventually{
% }
%
% \section{Implementation}
%
% \subsection{Identification}
%
%    \begin{macrocode}
%<*package>
\NeedsTeXFormat{LaTeX2e}
\ProvidesPackage{colonequals}%
  [2006/08/01 v1.0 Colon equals symbols (HO)]%
%    \end{macrocode}
%
% \subsection{Distance control}
%
%    \begin{macro}{\colonsep}
%    \begin{macrocode}
\newcommand*{\colonsep}{}
%    \end{macrocode}
%    \end{macro}
%
%    \begin{macro}{\doublecolonsep}
%    \begin{macrocode}
\newcommand*{\doublecolonsep}{}
%    \end{macrocode}
%    \end{macro}
%
% \subsection{Centered colons}
%
%    \begin{macrocode}
\def\@center@colon{%
  \mathpalette\@center@math{:}%
}
\def\@center@math#1#2{%
  \vcenter{%
    \m@th
    \hbox{$#1#2$}%
  }%
}
%    \end{macrocode}
%
%    \begin{macro}{\ratio}
%    Because the name \cs{colon} is already in use, the Unicode name
%    \cs{ratio} is used for the centered colon relation symbol.
%    (The \cs{ratio} of package \textsf{calc} is not used outside
%    calc expressions.)
%    \begin{macrocode}
\newcommand*{\ratio}{%
  \ensuremath{%
    \mathrel{%
      \@center@colon
    }%
  }%
}
%    \end{macrocode}
%    \end{macro}
%
%    \begin{macro}{\coloncolon}
%    \begin{macrocode}
\newcommand*{\coloncolon}{%
  \ensuremath{%
    \mathrel{%
      \@center@colon
      \doublecolonsep
      \@center@colon
    }%
  }%
}
%    \end{macrocode}
%    \end{macro}
%
% \subsection{Combined symbols}
%
%    \begin{macrocode}
\def\@make@colon@set#1#2{%
  \begingroup
    \let\@center@colon\relax
    \let\newcommand\relax
    \let\ensuremath\relax
    \let\mathrel\relax
    \let\colonsep\relax
    \let\doublecolonsep\relax
    \def\csx##1{%
      \expandafter\noexpand\csname ##1\endcsname
    }%
    \edef\x{\endgroup
      \newcommand*{\csx{colon#1}}{%
        \ensuremath{%
          \mathrel{%
            \@center@colon
            \colonsep
            {#2}%
          }%
        }%
      }%
      \newcommand*{\csx{coloncolon#1}}{%
        \ensuremath{%
          \mathrel{%
            \@center@colon
            \doublecolonsep
            \@center@colon
            \colonsep
            {#2}%
          }%
        }%
      }%
      \newcommand*{\csx{#1colon}}{%
        \ensuremath{%
          \mathrel{%
            {#2}%
            \colonsep
            \@center@colon
          }%
        }%
      }%
      \newcommand*{\csx{#1coloncolon}}{%
        \ensuremath{%
          \mathrel{%
            {#2}%
            \colonsep
            \@center@colon
            \doublecolonsep
            \@center@colon
          }%
        }%
      }%
    }%
  \x
}
%    \end{macrocode}
%
%    \begin{macrocode}
\@make@colon@set{equals}{=}%
\@make@colon@set{minus}{-}%
\@make@colon@set{approx}{\approx}
\@make@colon@set{sim}{\sim}
%    \end{macrocode}
%
%    \begin{macrocode}
%</package>
%    \end{macrocode}
%
% \section{Installation}
%
% \subsection{Download}
%
% \paragraph{Package.} This package is available on
% CTAN\footnote{\url{ftp://ftp.ctan.org/tex-archive/}}:
% \begin{description}
% \item[\CTAN{macros/latex/contrib/oberdiek/colonequals.dtx}] The source file.
% \item[\CTAN{macros/latex/contrib/oberdiek/colonequals.pdf}] Documentation.
% \end{description}
%
%
% \paragraph{Bundle.} All the packages of the bundle `oberdiek'
% are also available in a TDS compliant ZIP archive. There
% the packages are already unpacked and the documentation files
% are generated. The files and directories obey the TDS standard.
% \begin{description}
% \item[\CTAN{install/macros/latex/contrib/oberdiek.tds.zip}]
% \end{description}
% \emph{TDS} refers to the standard ``A Directory Structure
% for \TeX\ Files'' (\CTAN{tds/tds.pdf}). Directories
% with \xfile{texmf} in their name are usually organized this way.
%
% \subsection{Bundle installation}
%
% \paragraph{Unpacking.} Unpack the \xfile{oberdiek.tds.zip} in the
% TDS tree (also known as \xfile{texmf} tree) of your choice.
% Example (linux):
% \begin{quote}
%   |unzip oberdiek.tds.zip -d ~/texmf|
% \end{quote}
%
% \paragraph{Script installation.}
% Check the directory \xfile{TDS:scripts/oberdiek/} for
% scripts that need further installation steps.
% Package \xpackage{attachfile2} comes with the Perl script
% \xfile{pdfatfi.pl} that should be installed in such a way
% that it can be called as \texttt{pdfatfi}.
% Example (linux):
% \begin{quote}
%   |chmod +x scripts/oberdiek/pdfatfi.pl|\\
%   |cp scripts/oberdiek/pdfatfi.pl /usr/local/bin/|
% \end{quote}
%
% \subsection{Package installation}
%
% \paragraph{Unpacking.} The \xfile{.dtx} file is a self-extracting
% \docstrip\ archive. The files are extracted by running the
% \xfile{.dtx} through \plainTeX:
% \begin{quote}
%   \verb|tex colonequals.dtx|
% \end{quote}
%
% \paragraph{TDS.} Now the different files must be moved into
% the different directories in your installation TDS tree
% (also known as \xfile{texmf} tree):
% \begin{quote}
% \def\t{^^A
% \begin{tabular}{@{}>{\ttfamily}l@{ $\rightarrow$ }>{\ttfamily}l@{}}
%   colonequals.sty & tex/latex/oberdiek/colonequals.sty\\
%   colonequals.pdf & doc/latex/oberdiek/colonequals.pdf\\
%   colonequals.dtx & source/latex/oberdiek/colonequals.dtx\\
% \end{tabular}^^A
% }^^A
% \sbox0{\t}^^A
% \ifdim\wd0>\linewidth
%   \begingroup
%     \advance\linewidth by\leftmargin
%     \advance\linewidth by\rightmargin
%   \edef\x{\endgroup
%     \def\noexpand\lw{\the\linewidth}^^A
%   }\x
%   \def\lwbox{^^A
%     \leavevmode
%     \hbox to \linewidth{^^A
%       \kern-\leftmargin\relax
%       \hss
%       \usebox0
%       \hss
%       \kern-\rightmargin\relax
%     }^^A
%   }^^A
%   \ifdim\wd0>\lw
%     \sbox0{\small\t}^^A
%     \ifdim\wd0>\linewidth
%       \ifdim\wd0>\lw
%         \sbox0{\footnotesize\t}^^A
%         \ifdim\wd0>\linewidth
%           \ifdim\wd0>\lw
%             \sbox0{\scriptsize\t}^^A
%             \ifdim\wd0>\linewidth
%               \ifdim\wd0>\lw
%                 \sbox0{\tiny\t}^^A
%                 \ifdim\wd0>\linewidth
%                   \lwbox
%                 \else
%                   \usebox0
%                 \fi
%               \else
%                 \lwbox
%               \fi
%             \else
%               \usebox0
%             \fi
%           \else
%             \lwbox
%           \fi
%         \else
%           \usebox0
%         \fi
%       \else
%         \lwbox
%       \fi
%     \else
%       \usebox0
%     \fi
%   \else
%     \lwbox
%   \fi
% \else
%   \usebox0
% \fi
% \end{quote}
% If you have a \xfile{docstrip.cfg} that configures and enables \docstrip's
% TDS installing feature, then some files can already be in the right
% place, see the documentation of \docstrip.
%
% \subsection{Refresh file name databases}
%
% If your \TeX~distribution
% (\teTeX, \mikTeX, \dots) relies on file name databases, you must refresh
% these. For example, \teTeX\ users run \verb|texhash| or
% \verb|mktexlsr|.
%
% \subsection{Some details for the interested}
%
% \paragraph{Attached source.}
%
% The PDF documentation on CTAN also includes the
% \xfile{.dtx} source file. It can be extracted by
% AcrobatReader 6 or higher. Another option is \textsf{pdftk},
% e.g. unpack the file into the current directory:
% \begin{quote}
%   \verb|pdftk colonequals.pdf unpack_files output .|
% \end{quote}
%
% \paragraph{Unpacking with \LaTeX.}
% The \xfile{.dtx} chooses its action depending on the format:
% \begin{description}
% \item[\plainTeX:] Run \docstrip\ and extract the files.
% \item[\LaTeX:] Generate the documentation.
% \end{description}
% If you insist on using \LaTeX\ for \docstrip\ (really,
% \docstrip\ does not need \LaTeX), then inform the autodetect routine
% about your intention:
% \begin{quote}
%   \verb|latex \let\install=y\input{colonequals.dtx}|
% \end{quote}
% Do not forget to quote the argument according to the demands
% of your shell.
%
% \paragraph{Generating the documentation.}
% You can use both the \xfile{.dtx} or the \xfile{.drv} to generate
% the documentation. The process can be configured by the
% configuration file \xfile{ltxdoc.cfg}. For instance, put this
% line into this file, if you want to have A4 as paper format:
% \begin{quote}
%   \verb|\PassOptionsToClass{a4paper}{article}|
% \end{quote}
% An example follows how to generate the
% documentation with pdf\LaTeX:
% \begin{quote}
%\begin{verbatim}
%pdflatex colonequals.dtx
%makeindex -s gind.ist colonequals.idx
%pdflatex colonequals.dtx
%makeindex -s gind.ist colonequals.idx
%pdflatex colonequals.dtx
%\end{verbatim}
% \end{quote}
%
% \section{Catalogue}
%
% The following XML file can be used as source for the
% \href{http://mirror.ctan.org/help/Catalogue/catalogue.html}{\TeX\ Catalogue}.
% The elements \texttt{caption} and \texttt{description} are imported
% from the original XML file from the Catalogue.
% The name of the XML file in the Catalogue is \xfile{colonequals.xml}.
%    \begin{macrocode}
%<*catalogue>
<?xml version='1.0' encoding='us-ascii'?>
<!DOCTYPE entry SYSTEM 'catalogue.dtd'>
<entry datestamp='$Date$' modifier='$Author$' id='colonequals'>
  <name>colonequals</name>
  <caption>Colon equals symbols.</caption>
  <authorref id='auth:oberdiek'/>
  <copyright owner='Heiko Oberdiek' year='2006'/>
  <license type='lppl1.3'/>
  <version number='1.0'/>
  <description>
    This package defines poor man&#x2018;s symbols for mathematical
    relation symbols such as &#x201C;colon equals&#x201D;.
    The colon is centered around the horizontal math axis.
    <p/>
    The package is part of the <xref refid='oberdiek'>oberdiek</xref>
    bundle.
  </description>
  <documentation details='Package documentation'
      href='ctan:/macros/latex/contrib/oberdiek/colonequals.pdf'/>
  <ctan file='true' path='/macros/latex/contrib/oberdiek/colonequals.dtx'/>
  <miktex location='oberdiek'/>
  <texlive location='oberdiek'/>
  <install path='/macros/latex/contrib/oberdiek/oberdiek.tds.zip'/>
</entry>
%</catalogue>
%    \end{macrocode}
%
% \begin{thebibliography}{9}
%
% \bibitem{txfonts}
%   Young Ryu: \textit{The TX Fonts};
%   2000/12/15;
%   \CTAN{fonts/txfonts/}.
%
% \bibitem{mathabx}
%   Anthony Phan: \textit{Mathabx font series};
%   2005/05/16;
%   \CTAN{fonts/mathabx/}.
%
% \end{thebibliography}
%
% \begin{History}
%   \begin{Version}{2006/08/01 v1.0}
%   \item
%     First version.
%   \end{Version}
% \end{History}
%
% \PrintIndex
%
% \Finale
\endinput
|
% \end{quote}
% Do not forget to quote the argument according to the demands
% of your shell.
%
% \paragraph{Generating the documentation.}
% You can use both the \xfile{.dtx} or the \xfile{.drv} to generate
% the documentation. The process can be configured by the
% configuration file \xfile{ltxdoc.cfg}. For instance, put this
% line into this file, if you want to have A4 as paper format:
% \begin{quote}
%   \verb|\PassOptionsToClass{a4paper}{article}|
% \end{quote}
% An example follows how to generate the
% documentation with pdf\LaTeX:
% \begin{quote}
%\begin{verbatim}
%pdflatex colonequals.dtx
%makeindex -s gind.ist colonequals.idx
%pdflatex colonequals.dtx
%makeindex -s gind.ist colonequals.idx
%pdflatex colonequals.dtx
%\end{verbatim}
% \end{quote}
%
% \section{Catalogue}
%
% The following XML file can be used as source for the
% \href{http://mirror.ctan.org/help/Catalogue/catalogue.html}{\TeX\ Catalogue}.
% The elements \texttt{caption} and \texttt{description} are imported
% from the original XML file from the Catalogue.
% The name of the XML file in the Catalogue is \xfile{colonequals.xml}.
%    \begin{macrocode}
%<*catalogue>
<?xml version='1.0' encoding='us-ascii'?>
<!DOCTYPE entry SYSTEM 'catalogue.dtd'>
<entry datestamp='$Date$' modifier='$Author$' id='colonequals'>
  <name>colonequals</name>
  <caption>Colon equals symbols.</caption>
  <authorref id='auth:oberdiek'/>
  <copyright owner='Heiko Oberdiek' year='2006'/>
  <license type='lppl1.3'/>
  <version number='1.0'/>
  <description>
    This package defines poor man&#x2018;s symbols for mathematical
    relation symbols such as &#x201C;colon equals&#x201D;.
    The colon is centered around the horizontal math axis.
    <p/>
    The package is part of the <xref refid='oberdiek'>oberdiek</xref>
    bundle.
  </description>
  <documentation details='Package documentation'
      href='ctan:/macros/latex/contrib/oberdiek/colonequals.pdf'/>
  <ctan file='true' path='/macros/latex/contrib/oberdiek/colonequals.dtx'/>
  <miktex location='oberdiek'/>
  <texlive location='oberdiek'/>
  <install path='/macros/latex/contrib/oberdiek/oberdiek.tds.zip'/>
</entry>
%</catalogue>
%    \end{macrocode}
%
% \begin{thebibliography}{9}
%
% \bibitem{txfonts}
%   Young Ryu: \textit{The TX Fonts};
%   2000/12/15;
%   \CTAN{fonts/txfonts/}.
%
% \bibitem{mathabx}
%   Anthony Phan: \textit{Mathabx font series};
%   2005/05/16;
%   \CTAN{fonts/mathabx/}.
%
% \end{thebibliography}
%
% \begin{History}
%   \begin{Version}{2006/08/01 v1.0}
%   \item
%     First version.
%   \end{Version}
% \end{History}
%
% \PrintIndex
%
% \Finale
\endinput
|
% \end{quote}
% Do not forget to quote the argument according to the demands
% of your shell.
%
% \paragraph{Generating the documentation.}
% You can use both the \xfile{.dtx} or the \xfile{.drv} to generate
% the documentation. The process can be configured by the
% configuration file \xfile{ltxdoc.cfg}. For instance, put this
% line into this file, if you want to have A4 as paper format:
% \begin{quote}
%   \verb|\PassOptionsToClass{a4paper}{article}|
% \end{quote}
% An example follows how to generate the
% documentation with pdf\LaTeX:
% \begin{quote}
%\begin{verbatim}
%pdflatex colonequals.dtx
%makeindex -s gind.ist colonequals.idx
%pdflatex colonequals.dtx
%makeindex -s gind.ist colonequals.idx
%pdflatex colonequals.dtx
%\end{verbatim}
% \end{quote}
%
% \section{Catalogue}
%
% The following XML file can be used as source for the
% \href{http://mirror.ctan.org/help/Catalogue/catalogue.html}{\TeX\ Catalogue}.
% The elements \texttt{caption} and \texttt{description} are imported
% from the original XML file from the Catalogue.
% The name of the XML file in the Catalogue is \xfile{colonequals.xml}.
%    \begin{macrocode}
%<*catalogue>
<?xml version='1.0' encoding='us-ascii'?>
<!DOCTYPE entry SYSTEM 'catalogue.dtd'>
<entry datestamp='$Date$' modifier='$Author$' id='colonequals'>
  <name>colonequals</name>
  <caption>Colon equals symbols.</caption>
  <authorref id='auth:oberdiek'/>
  <copyright owner='Heiko Oberdiek' year='2006'/>
  <license type='lppl1.3'/>
  <version number='1.0'/>
  <description>
    This package defines poor man&#x2018;s symbols for mathematical
    relation symbols such as &#x201C;colon equals&#x201D;.
    The colon is centered around the horizontal math axis.
    <p/>
    The package is part of the <xref refid='oberdiek'>oberdiek</xref>
    bundle.
  </description>
  <documentation details='Package documentation'
      href='ctan:/macros/latex/contrib/oberdiek/colonequals.pdf'/>
  <ctan file='true' path='/macros/latex/contrib/oberdiek/colonequals.dtx'/>
  <miktex location='oberdiek'/>
  <texlive location='oberdiek'/>
  <install path='/macros/latex/contrib/oberdiek/oberdiek.tds.zip'/>
</entry>
%</catalogue>
%    \end{macrocode}
%
% \begin{thebibliography}{9}
%
% \bibitem{txfonts}
%   Young Ryu: \textit{The TX Fonts};
%   2000/12/15;
%   \CTAN{fonts/txfonts/}.
%
% \bibitem{mathabx}
%   Anthony Phan: \textit{Mathabx font series};
%   2005/05/16;
%   \CTAN{fonts/mathabx/}.
%
% \end{thebibliography}
%
% \begin{History}
%   \begin{Version}{2006/08/01 v1.0}
%   \item
%     First version.
%   \end{Version}
% \end{History}
%
% \PrintIndex
%
% \Finale
\endinput
|
% \end{quote}
% Do not forget to quote the argument according to the demands
% of your shell.
%
% \paragraph{Generating the documentation.}
% You can use both the \xfile{.dtx} or the \xfile{.drv} to generate
% the documentation. The process can be configured by the
% configuration file \xfile{ltxdoc.cfg}. For instance, put this
% line into this file, if you want to have A4 as paper format:
% \begin{quote}
%   \verb|\PassOptionsToClass{a4paper}{article}|
% \end{quote}
% An example follows how to generate the
% documentation with pdf\LaTeX:
% \begin{quote}
%\begin{verbatim}
%pdflatex colonequals.dtx
%makeindex -s gind.ist colonequals.idx
%pdflatex colonequals.dtx
%makeindex -s gind.ist colonequals.idx
%pdflatex colonequals.dtx
%\end{verbatim}
% \end{quote}
%
% \section{Catalogue}
%
% The following XML file can be used as source for the
% \href{http://mirror.ctan.org/help/Catalogue/catalogue.html}{\TeX\ Catalogue}.
% The elements \texttt{caption} and \texttt{description} are imported
% from the original XML file from the Catalogue.
% The name of the XML file in the Catalogue is \xfile{colonequals.xml}.
%    \begin{macrocode}
%<*catalogue>
<?xml version='1.0' encoding='us-ascii'?>
<!DOCTYPE entry SYSTEM 'catalogue.dtd'>
<entry datestamp='$Date$' modifier='$Author$' id='colonequals'>
  <name>colonequals</name>
  <caption>Colon equals symbols.</caption>
  <authorref id='auth:oberdiek'/>
  <copyright owner='Heiko Oberdiek' year='2006'/>
  <license type='lppl1.3'/>
  <version number='1.0'/>
  <description>
    This package defines poor man&#x2018;s symbols for mathematical
    relation symbols such as &#x201C;colon equals&#x201D;.
    The colon is centered around the horizontal math axis.
    <p/>
    The package is part of the <xref refid='oberdiek'>oberdiek</xref>
    bundle.
  </description>
  <documentation details='Package documentation'
      href='ctan:/macros/latex/contrib/oberdiek/colonequals.pdf'/>
  <ctan file='true' path='/macros/latex/contrib/oberdiek/colonequals.dtx'/>
  <miktex location='oberdiek'/>
  <texlive location='oberdiek'/>
  <install path='/macros/latex/contrib/oberdiek/oberdiek.tds.zip'/>
</entry>
%</catalogue>
%    \end{macrocode}
%
% \begin{thebibliography}{9}
%
% \bibitem{txfonts}
%   Young Ryu: \textit{The TX Fonts};
%   2000/12/15;
%   \CTAN{fonts/txfonts/}.
%
% \bibitem{mathabx}
%   Anthony Phan: \textit{Mathabx font series};
%   2005/05/16;
%   \CTAN{fonts/mathabx/}.
%
% \end{thebibliography}
%
% \begin{History}
%   \begin{Version}{2006/08/01 v1.0}
%   \item
%     First version.
%   \end{Version}
% \end{History}
%
% \PrintIndex
%
% \Finale
\endinput
